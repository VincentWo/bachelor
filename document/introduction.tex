\chapter*{Introduction}
\addcontentsline{toc}{chapter}{Introduction}
Upon the first introduction to derivatives one learns quite early that a function
$f: \RR \to \RR$ is constant if and only if $f'(x) = 0$ for every $x \in \RR$. This statement is useful,
but also contains a subtle implicit assumption: It depends not only on the value of $f'$, but also on
the domain of $f$. In fact, take:
\begin{align*}
	f: \RR \setminus \{0\} &\to \RR \\
		x &\mapsto \begin{cases}
			-1 \quad&\text{if } x < 0 \\
			 1 \quad&\text{if } x > 0
		\end{cases}
\end{align*}
this functions unsurprisingly has derivative zero, but is not globally constant, only locally i.e. every
point has a neighborhood on which the restriction of $f$ is constant.

In fact it holds in general:
\begin{lemma}
	Let $\U \subseteq \RR$ open and $f: \U \to \RR$ a smooth function. $f$ is locally constant if and only if
	$f' = 0$.
\end{lemma}
As it usually goes this statement is more general, but also weaker: There are obviously sets (like $\RR$ itself) on
which a vanishing derivative implies constness. Recovering the original statement from our general version requires
detecting these cases. To figure out the distinguish property of these sets, we follow the proof of the statement on
$\RR$ to see where it breaks apart for general sets:

On $\RR$ this implication is easily proven: Let $f: \RR \to \RR$ be locally constant. For every $x \in \RR$
pick an open neighborhood $\V_x$ of $x$ on which $f$ is constant. Pick any $y \in \RR$ and
define:
\begin{align*}
	\V^=      &\coloneqq \bigcup_{x \in \RR\colon f(x) = f(y)} \V_x \\
	\V^{\neq} &\coloneqq \bigcup_{x \in \RR\colon f(x) \neq f(y)} \V_x
\end{align*}
Both of these sets are open and disjoint and $\RR = \V^= \cup \V^{\neq}$, hence also closed.
But the only sets on $\RR$ that are open and closed are the empty set and $\RR$ itself and since
$\V^=$ is non-empty by construction it has to be $\RR$, therefore $f$ is constant.

This non-existence of sets that are open and closed (except $\RR$ and the empty set) follows from a
geometric intuitive topological property:
\begin{definition}
	Let $X$ be a topological space. We call X \textbf{connected} if there are no non-empty
	disjoint open sets $\V, \O \subseteq X$ such that $\V \cup \O = X$.
\end{definition}
Hence if $\U$ is connected $f$ is constant and the connected subsets of $\RR$ are exactly
these on which $f' = 0$ implies constness of $f$.

Keeping this example in the back of the mind, we are going to take a look at a seemingly unrelated
question. In Physics one often deals with forces that are dependent on the position of an object (e.g. gravity
acting on a spaceship as a function of distance to earth). The usual way to model these are vector fields:
\begin{definition}
Let $\U \subseteq \RR^n$. A \textbf{vector field on $\U$} is a smooth map $f: \U \to \RR^n$
\end{definition}

Forces are a useful concept, but they have the problem of being proportional to the acceleration
of an object, turning any equation about the position of an object into a second order differential
equation. Because of this (and for unrelated reasons that we can not get into) it is often more convenient
to work with the potential energy of an object, defined as the antiderivative of the force field. That is,
given a  force field $f: \RR^n \to \RR^n$ we define its potential function to be a smooth function $u: \RR^n \to \RR$
such that $\nabla u = f$ (note that the potential is only unique up to a added constant).

Given an arbitrary vector field $f: \RR^n \to \RR^n$ there is no guarantee that it has a potential function, hence we would like to
single out these that do. There is a special term for these fields, coming from physics:
\begin{definition}
A vector field $f: \RR^n \supseteq \U \to \RR^n$ is called \textbf{conservative} if there is a potential $u: \U \to \RR$
such that $\nabla u = f$.
\end{definition}
Calling such a vector field conservative stems from the fact that these are exactly the vector
fields that conserve energy i.e. every line integral (and the line integral of a force is
energy) only depends on the difference of the potential at their endpoints.
In particular the line integral over any closed loop is zero.

This makes a question arise: Given an vector field $f: \U \to \RR^n$, how can one check that it
is conservative? It turns out that for $n = 1$ every vector field is conservative: This is
a direct consequence of the fundamental theorem of calculus. But for $n \geq 2$ non-conservative vector
fields exist! So how, given any vector field, can we decide whether it is conservative? One possibility would
be to "just" find a potential function or a closed loop with non-zero integral, but this is not always a simple
or even feasible task.

At this point in a physics class one would usually learn of a simpler method to establish whether a vector field is
conservative by defining the curl of a vector field:
\begin{definition}
	Let $\U \subseteq \RR^2$ be open and $f: \U \to \RR^2$ a vector field. Define the \textbf{curl of $f$} as
	the smooth function defined by:
	\begin{align*}
		\curl f: \RR^2 &\to \RR \\
		 (x_1, x_2)
			&\mapsto \frac{\partial}{\partial x_1} f_2(x_1, x_2) - \frac{\partial}{\partial x_2} f_1(x_1, x_2)
	\end{align*}
	With $f_1, f_2$ being the components of $f$ i.e. $f = (f_1, f_2)$
\end{definition}
Being based on derivatives, the curl is a property that can be calculated rather easily. And luckily one
can reduce the question whether a vector field is conservative to a statement about its curl:
\begin{lemma}
Let $f: \RR^2 \to \RR^2$ be a vector field. Then $f$ is conservative if and only if its curl is zero.
\end{lemma}

This is wonderful! This Lemma reduces a hard question to an easy one and we are now able to easily calculate whether
a given vector field is conservative. But, as the attentive reader might have already noticed, we did so at the cost of generality.
Similar as with our previous statement, we are again limited to vector fields defined on the whole of $\RR^2$.
And not without reason, either: There are subsets of $\RR^2$ on which curl-free vector fields are not necessary conservative.
As an example take $\U = \RR^2 \setminus \{0\}$ and define:
\begin{align*}
	f: \U &\to \RR \\
		(x, y) &\mapsto \frac{1}{x^2 + y^2}(-y, x)
\end{align*}
Its curl vanishes:
\begin{align*}
	\curl f = \frac{y^2 - x^2}{(x^2 + y^2)^2} - \frac{y^2 - x^2}{(x^2 + y^2)^2} = 0
\end{align*}
But this vector field is clearly not conservative: Take e.g. the integral on the unit circle (obviously a closed
loop), it is not zero!

Our argument fails for a similar reason as in the first example: It includes a assumption on the domain of
the vector field, since curl-free vector fields are in general not (globally) conservative, but only
locally, i.e. every point has a neighborhood on which the vector field is conservative.

That curl-free vector fields are locally conservative can intuitively be seen by approximating small
loops using small squares. Imagine one such square of side length $2\Delta > 0$ with mid-point
$(x_1, x_2)$. The (counter-clockwise) integral of its boundary over a vector field $f$ can then be
approximated as:
\begin{align*}
	\Delta \cdot & (  f_1(x_1, x_2 - \Delta) + f_2(x_1 + \Delta, x_2) - f_1(x_1, x_2 + \Delta) - f_2(x_1 - \Delta, x_2) \\
	&\approx \Delta \Big( f_1(x_1, x_2) - \Delta \frac{\partial f_1}{\partial x_2}(x_1, x_2) + f_2(x_1, x_2) + \Delta \frac{\partial f_2}{\partial x_1}(x_1, x_2) \\
	       & \phantom{\approx \Delta (}		   - f_1(x_1, x_2) - \Delta \frac{\partial f_1}{\partial x_2} (x_1, x_2) - f_2(x_1, x_2) + \Delta \frac{\partial f_2}{\partial x_1}(x_1, x_2) \Big) \\
	      &= 2\Delta^2 \Big( \frac{\partial f_2}{\partial x_1}(x_1, x_2) - \frac{\partial f_1}{\partial x_2}(x_1, x_2)\Big) \\
	&= 2 \Delta^2 \curl f
\end{align*}
Hence vector fields are locally conservative if and only if they are curl-free.

Thus the natural question arises: On which domains is a locally conservative vector field conservative? 

In the familiar case of $\RR^2$ this implications holds since every closed loop can be filled up
by a disk. This disk can then be subdivided into small enough pieces such that $f$ is conservative on them. The integral around
the closed loop is then equal to the sum of the integrals of $f$ over the boundaries of the pieces, all
of which are zero. Hence the integral is zero and $f$ is conservative.

This argument only depends on the ability to fill every closed loop and can thus be extended to every space on which this is
possible. This property also has a nice geometrical interpretation of being equivalent to a space not having any "holes" and
is also a famous topological invariant:
\begin{definition}
We call a topological space $X$ \textbf{simply connected} if it is connected and every closed loop i.e. map $f: S^1 \to X$ admits a extension to $\mathbb{D}^2$.
\end{definition}

Similar as in the previous case, we found a connection between a topological property of a space and properties of smooth
objects with vanishing "derivative". It turns out that there is a general statement called \textit{de Rhams theorem}, a deep
statement about the connection between the "globalization" of local properties of smooth functions on a space and topological
properties of the space itself, which includes these examples as a special case and is the subject of our thesis.

Properly stating de Rhams theorem requires the development of de Rham Cohomology: The concept of smooth
functions and vector fields can be generalized into \textbf{differential forms} with the gradient/curl then
becoming the \textbf{exterior derivative $\mathrm{d}$}, mapping $k$-forms to $k+1$-forms. Thus the objects
of our interest are the forms with zero exterior exterior derivative (called \textit{closed}) modulo the ones that are
the exterior derivative of another form (called \textit{exact}), or in algebraic terms:
\begin{align*}
	H^* \coloneqq \ker(\mathrm{d}) / \Img(\mathrm{d})
\end{align*}

\begin{remark}
The attentive reader might notice that this is only sensible if $\mathrm{d}$ is nil-potent, that
is $\mathrm{d} \circ \mathrm{d} = 0$, but that is indeed the case, as we are going to proof.
\end{remark}

$H^*$ is called the de Rham cohomology of a space and de Rhams theorem states that this is
isomorphic to a topological invariant: Čech cohomology, which can be understood as measuring the "holes"
in a space by detecting certain objects that have empty boundary (called cycles) modulo objects that are the
boundary of something. (This can be seen as the generalization of a closed loop that cannot be filled in by a disk)

That these two invariants are isomorphic is a deep fact, first conjectured by E. Cartan in 1928 and
proven in 1931 by George de Rham in his dissertation \cite{de_rham_sur_1931} using integration
of differential $k$-forms. While giving an intuitive interpretation of de Rham cohomology being dual to the
(oriented and compact) submanifolds
\footnote{
	Sadly it is more complicated than this, because one needs to consider more than just these submanifolds.
	Properly doing this leads to the theory of \textit{currents} for a detailed review see \cite{de_rham_differentiable_1984}
}
it also requires the machinery of integration of differential forms and gets quite technical in the details.

On the other extreme it is possible to prove this statement using \textit{sheaf cohomology}, which is a category theoric tool
that generalizes the description of obstructions to the extension of local solutions of geometric problems to global ones.
Such a sheaf-theoric proof is elegant, but it also lacks the geometric intuition of de Rhams approach and requires heavy
algebraic machinery

In this thesis we are going to try to find a middle ground between these extremes, using a method, that was premiered in 1952
by André Weil in the (French) paper \cite{weil_sur_1952}
(with \cite{tu_differential_1982} being a good (non-French) modern account). While being primarily algebraic it foregoes too
heavy machinery, does not require integration and offers a geometric explanation (though not one as obvious).

Out contribution lies in supplying details of the proof that were left out of \cite{tu_differential_1982}, combining
it with a self-contained
\footnote{
	Not self-contained in the sense of proving every statement, but self-contained in the sense that it can be
	(hopefully) understood on its own
}
introduction to all the needed materials and motivations, trying to make this thesis accessible to anybody with a
knowledge of linear algebra, analysis and topology at an undergraduate level without requiring any further topic-specific
knowledge.
 
