\chapter*{Introduction}
\addcontentsline{toc}{chapter}{Introduction}
Upon the first introduction to derivatives one learns quite early that a function
$f: \RR \to \RR$ is constant if and only if $f'(x) = 0$ for every $x \in \RR$. This statement is quite usefull,
but it subtle implicit assumption, depending not only on the value of $f'$, but also on the domain
of $f$. In fact, take:
\begin{align*}
	f: \RR \setminus \{0\} &\to \RR \\
		x &\mapsto \begin{cases}
			-1 \quad&\text{if } x < 0 \\
			 1 \quad&\text{if } x > 0
		\end{cases}
\end{align*}
this functions has derivative zero, but is not constant!

This does not come as a surprise and $f$ is of course locally constant,
i.e. every $\varepsilon$-ball can be shrunk to make $f$ constant on it, or properly:
\begin{definition}[Locally constant]
	Let $\U \subseteq \RR$ and $f: \U \to \RR$. We call $f$ \textbf{locally constant} if for every $x \in \U$
	there is a open neighborhood $\V \subseteq \U$ of $x$ such that $f$ is constant on $\V$.
\end{definition}

Using this we get the general version of our theorem:
\begin{lemma}
	Let $\U \subseteq \RR$ and $f: \U \to \RR$ a smooth function. $f$ is locally constant if and only if
	$f' = 0$.
\end{lemma}
As it usually goes, this statement is more general, but weaker: There are obviously sets (like $\RR$ itself) on
which a vanishing derivative implies constness e.g. $\RR$ itself – what sets them apart?

On $\RR$ this implication is easily proven: Let $f: \RR \to \RR$ be locally constant. For every $x \in \RR$
pick an open neighborhood $\V_x$ of $x$ on which $f$ is constant. Now pick any $y \in \RR$ and
define:
\begin{align*}
	\V^=      &\coloneqq \bigcup_{x \in \RR\colon f(x) = f(y)} \V_x \\
	\V^{\neq} &\coloneqq \bigcup_{x \in \RR\colon f(x) \neq f(y)} \V_x
\end{align*}
Both of these sets are open and disjoint and $\RR = \V^= \cup \V^{\neq}$, hence also closed.
But the only sets on $\RR$ that are open and closed are the empty set and $\RR$ itself and since
$\V^=$ is non-empty by construction it has to be $\RR$, therefore $f$ is constant.

This non-existence of sets that are open and closed (except $\RR$ and the empty set) follows from a
geometric intuitive topological property:
\begin{definition}
	Let $X$ be a topological space. We call X \textbf{connected} if there are no non-empty
	disjoint open sets $\V, \O \subseteq X$ such that $\V \cup \O = X$.
\end{definition}
Hence if $\U$ is connected $f$ is constant and the connected subsets of $\RR$ are exactly
these on which $f' = 0$ implies constness of $f$.

Keeping this example in the back of the mind, we are going to take a look at a seemingly unrelated
question. In Physics one often deals with forces that are dependent on the position of an object (e.g. gravity
acting on a spaceship as a function of distance to earth). The usual way to model these are vector fields:
\begin{definition}
Let $\U \subseteq \RR^n$. A \textbf{vector field on $\U$} is a smooth map $f: \U \to \RR^n$
\end{definition}

Forces are a usefull concept, but they have the problem of being proportional to the acceleration
of an object, turning any equation about the position of an object into a second order differential
equation. Because of this (and for unrelated reasons that we can not get into) it is often more conventient
to work with the potential energy of an object, defined as the antideritative of the force field. That is,
given a  force field $f: \RR^n \to \RR^n$ we define its potential function to be a smooth function $u: \RR^n \to \RR$
such that $\nabla u = f$ (note that the potential is only unique up to a added constant). But given a force field $f: \RR^n \to \RR^n$
there is no guarentee that it has a potential, hence we would like to limit ourselves to the ones that do.
\begin{definition}
A vector field $f: \RR^n \supseteq \U \to \RR^n$ is called \textbf{conservative} if there is a potential $u: \U \to \RR$
such that $\nabla u = f$.
\end{definition}
Calling such a vector field conservative stems from the fact that these are exactly the vector
fields that conserve energy i.e. every line integral (and the line integral of a force is
energy) only depends on its endpoints. In particular the line integral over any closed loop is
zero.

This rises the question: Given an vector field $f: \U \to \RR^n$, how can one check that it
is conservative? It turns out that for $n = 1$ every vector field is conservative: This is
a direct consequence of the fundamental theorem of calculus. But for $n \geq 2$ non-conservative vector
fields exist! So how, given any vector field, can we decide whether it is conservative? One possibility would
be to "just" find a potential function or a closed loop with non-zero integral, but this isn't always a simple
or even feasible task. Luckily one usually learns of a simpler method using the curl of a vector field:
\begin{definition}
	Let $\U \subseteq \RR^2$ be open and $f: \U \to \RR^2$ a vector field. Define the \textbf{curl of $f$} as
	the smooth function defined by:
	\begin{align*}
		\curl f: \RR^2 &\to \RR \\
		 (x_1, x_2)
			&\mapsto \frac{\partial}{\partial x_1} f_2(x_1, x_2) - \frac{\partial}{\partial x_2} f_1(x_1, x_2)
	\end{align*}
	With $f_1, f_2$ being the components of $f$ i.e. $f = (f_1, f_2)$
\end{definition}
Being based on derivatives, the curl is a property that can be calculated rather easily. And luckily one
can reduce the question whether a vector field is conservative to a statement about its curl:
\begin{lemma}
Let $f: \RR^2 \to \RR^2$ be a vector field. Then $f$ is conservative if and only if its curl is zero.
\end{lemma}

This is wonderfull! We reduced a hard question to an easy one, turning conservativity into an easily calculated
invariant. But, as the attentive reader might have already noticed, we did so at the cost of generality.
Similiar to our first statement, we again reduced the statement to vector fields defined on the whole of
$\RR^2$. And not without reason, either: There are subsets of $\RR^2$ on which curl-free vector fields are
not nescecarrily conservative. As an example take $\U = \RR^2 \setminus \{0\}$ and define:
\begin{align*}
	f: \U &\to \RR \\
		(x, y) &\mapsto \frac{1}{x^2 + y^2}(-y, x)
\end{align*}
Its curl is:
\begin{align*}
	\curl f = \frac{y^2 - x^2}{(x^2 + y^2)^2} - \frac{y^2 - x^2}{(x^2 + y^2)^2} = 0
\end{align*}
But this vector field is clear not conservative: Take e.g. the integral on the unit circle (a trivially closed
loop), it is not zero!

Our argument fails for a similiar reason as in the first example: It includes a assumption on the domain of
the vector field, since curl-free vector fields are not conservative in general, but something weaker:
\begin{definition}
A vector field $f: \RR^n \supseteq  \U \to \RR^n$ is called \textbf{locally conservative} if every point $x \in \U$ has an
open neighborhood $\V \subseteq \U$ such that $f|_\V$ is conservative.
\end{definition}

That curl-free vector fields are locally conservative can intuitively be seen by approximating small loops using small squares.
Imagine one such square of side length $2\Delta > 0$ with mid-point $(x_1, x_2)$. The (counter-clockwise) integral of its boundary over
a vector field $f$ can then be approximated as:
\begin{align*}
	\Delta \cdot & (  f_1(x_1, x_2 - \Delta) + f_2(x_1 + \Delta, x_2) - f_1(x_1, x_2 + \Delta) - f_2(x_1 - \Delta, x_2) \\
	&\approx \Delta \Big( f_1(x_1, x_2) - \Delta \frac{\partial f_1}{\partial x_2}(x_1, x_2) + f_2(x_1, x_2) + \Delta \frac{\partial f_2}{\partial x_1}(x_1, x_2) \\
	       & \phantom{\approx \Delta (}		   - f_1(x_1, x_2) - \Delta \frac{\partial f_1}{\partial x_2} (x_1, x_2) - f_2(x_1, x_2) + \Delta \frac{\partial f_2}{\partial x_1}(x_1, x_2) \Big) \\
	      &= 2\Delta^2 \Big( \frac{\partial f_2}{\partial x_1}(x_1, x_2) - \frac{\partial f_1}{\partial x_2}(x_1, x_2)\Big) \\
	&= 2 \Delta^2 \curl f
\end{align*}
Hence vector fields are locally conservative if and only if they are curl-free. But again: There are sets on which
every locally conservative vector field is conservative – why? Without giving a rigorous proof:
As an approximate answer imagine a closed loop on $\RR^2$, to keep it simple again as a square.
This square can be split into 4 smaller squares and then the integral over the boundary of the larger square is equal to the sum
of the integrals over the boundaries of the smaller ones. By doing this until all squares
are small enough such that the integral over their boundary is zero shows that their sum and hence the integral over the boundary of
the large square is zero.

But in the case of $\RR^2 \setminus \{0\}$ this is not possible: Any loop/square that contains the origin can not be split into smaller squares.
This property can now again be phrased in topological terms:
\begin{definition}
We call a topological space $X$ \textbf{simply connected} if every closed loop i.e. map $f: S^1 \to X$ admits a extension to $\mathbb{D}^2$.
\end{definition}
Hence on simply connected spaces locally conservative vector fields are conservative!

Both of these examples can be seen as instances of \textbf{de Rhams theorem}, a deep statement about the connection between the "globalization"
of local properties of smooth functions on a space and topological properties of the space itself,
and the subject of this thesis.

Properly stating de Rhams theorem requires the development of de Rham Cohomology: The concept of smooth
functions and vector fields can be generalized into \textbf{differential forms} with the gradient/curl then
becoming the \textbf{exterior derivative $\mathrm{d}$}, that maps $k$-forms to $k+1$-forms. Thus the objects
of our interest are the forms with zero exterior exterior derivative (called closed) modulo the ones that are
the exterior derivative of another form (called exact), or in algebraic terms:
\begin{align*}
	H^* \coloneqq \ker(\mathrm{d}) / \Img(\mathrm{d})
\end{align*}

\begin{remark}
The attentive reader might notice that this is only sensible if $\mathrm{d}$ is nil-potent, that
is $\mathrm{d} \circ \mathrm{d} = 0$, but that is indeed the case, as we are going to proof.
\end{remark}

$H^*$ is called the de Rham cohomology of a space and de Rhams theorem now states that this is
isomorphic to a topological invariant: Čech cohomology, which can be understood as measuring the "holes"
in a space by detecting certain objects that have empty boundary (caled cycles) modulo objects that are the
boundary of something.

That these two invariants are isomorphic is a surprising fact, first conjectured by E. Cartan in 1928 and
proven in 1931 by George de Rham in his dissertation \cite{de_rham_sur_1931} using the theory of integration
of $k$-forms.

The proof that we are going to use was premiered 1952 by Weil in the (french) paper \cite{weil_sur_1952},
with \cite{tu_differential_1982} being a good modern account. Our contributing lies in extending details
that were left out of \cite{tu_differential_1982} and combining it with a self-contained \footnote{
Not self-contained in the sense of proving every statement, but self-contained in the sense that it can be
(hopefully) understood on its own
} introductiong to all the needed materials \& motivations, trying to make this thesis accessible to anybody
with linear algebra, analysis and basic topology without requiring any topic-specific knowledge.
 
