\chapter{De Rham Cohomology}
Developing de Rham cohomology begins with generalizing smooth functions to domains
that aren't open subsets of $\RR^n$. This leads to the ubiquitous concept of smooth
manifolds, the setting of the rest of this thesis, which then allows us to generalize
the concept of differentiation using the concept of a tangent space – and its dual
\textit{integration} using the cotangent space, culminating into the theory of differential forms
and de Rham cohomology.

\begin{remark}
While we are going to aim for a self-contained introduction to all the needed differential geometric
background, it is impossible to do so without skipping quite some detail. If one desires a full fledged
introduction (and the topic is definitly more than interesting enough to deserve one), then one should read
one of the plethora of well written books about it e.g. \cite{lee_introduction_2013}, \cite{spivak_comprehensive_1979}
or \cite{wendl_differential_2022}
\end{remark}

\section{Smooth Manifolds}
As should be not too suprising this is in general not possible, there isn't any sensible notion
of a map $f: X \to \RR$ being smooth for an arbitrary topological space. To define smoothness
we need both some additional structure and some "well-behavedness" i.e. we are going to have to
limit ourselves to a certain class of "good" spaces.

We are going to start with the "well-behavedness". All throughout calculus the real numbers play a
quite important role, hence to even have a shot at defining differential structures we are going
to require our space to be "locally euclidean", that is, we want every point to have a neighborhood that
is homeomorphic to $\RR^n$. This leads us to the definition of a topological manifold:
\begin{definition}[Topological Manifold]
A \underline{topological $n$-manifold} is a second countable and Hausdorff topological space $M$ such
that every $x \in M$ has a neighborhood $\U \ni x$ that is homeomorphic to $\RR^n$.
\end{definition}


% TODO: rewrite this
This manifold concept is already a quite powerfull and versatile one, but it is still a purely topological
concept. But by adding a bit more structure we are going to develop from it the fundamental setting of differential geometry:
The smooth manifold. And after introducing just enough differential geometric terminology we are going
to introduce the invariant that interests us most: de Rham cohomology, the multi dimensional analog of
the question whether local constness of smooth functions implies constness.

But first, we are going to have to develop smooth manifolds:

\section{Smooth Manifolds}
A smooth manifold is first and foremost a space that locally looks like $\RR^n$:
But one might note that 
% TODO: Correct definitions etc.
\begin{definition}[Coordinate Transformation]
A \underline{coordinate transformation} from $U_\alpha$ to $U_\beta$ is the map
\[
	\phi_{\alpha \beta}: 
		\phi^{-1}_\alpha(\mathcal{U}_\alpha \cap \mathcal{U}_\beta)
		\to
		\phi^{-1}_\beta(\mathcal{U}_\alpha \cap \mathcal{U}_\beta) \
		x \mapsto (\phi^{-1}_\beta \circ \phi_\alpha)(x)
\]
for charts $(\mathcal{U}_\alpha, \phi_\alpha), (\mathcal{U}_\beta, \phi_\beta)$
of a $n$-manifold $M$.
\end{definition}
Though quite wordy it is a natural concept: Given a point
$p \in \mathcal{U}_\alpha \cap \mathcal{U}_\beta$ there are two unique
coordinates $\mathbf{x}, \mathbf{y} \in \RR^n$ of $p$ such that
$\phi_\alpha(\mathbf{x}) = \phi_\beta(\mathbf{x}) = p$ and the coordinate
transformation is just the way to do that. The problem: If we find a way to
define differentiation on a $n$-manifold $M$ we would want to make sure that the
differentiability of a function $f: M \to \RR$ at a point $p \in M$ (whatever
that means) does not depend on our chosen coordinates $\mathbf{x} \in \RR^n$ for
$p$, but while the coordinate transformations are required to be continous,
they aren't required to uphold any kind of differentiability conditions we
invent. Since these are maps from $\RR^n$ to $\RR^n$ the most natural requirement
would be smoothness \footnote{
	Instead of smoothness one can of course also require differentiability,
	but this soon gets messy since some concepts require differentiating more
	than once, so we will be practical and just say smooth, leaving the
	detailed requirements of theorems and definitions as an (not recommened
	unless nescessary) exercise for the reader.
} and this also turns out to be exactly the right thing
\begin{definition}
A \underline{smooth manifold} is a manifold
$M$ together with an atlas $\{\mathcal{U}_\alpha, \phi_\alpha\})$  such that
every coordinate transformation is smooth.
\end{definition}
\begin{remark}
Note that a manifold is only a topological space while a smooth manifold is
a topological space together with a choice of atlas. This choice is actually
important since there are manifolds which are homeomorphic,
but not diffeomorphic, the differential geometric equivalent of homeomorphic.
\end{remark}
This requirement allows us to now define differentiability in a natural way
\begin{definition}
A function $f: M \to N$ from a $m$-manifold $M$ to a $n$-manifold $N$ is
\underline{smooth} at the point $p \in M$ if there are open sets
$p \in \mathcal{U} \subseteq M, f(p) \in \mathcal{V} \subseteq N$ together
with coordinate charts $\phi: \mathbb{R}^m \to \mathcal{U},
\psi: \mathbb{R}^m \to \mathcal{V}$ such that
\[
	\psi^{-1} \circ f \circ \phi
\]
is smooth at $\phi^{-1}(p)$. $f$ is smooth if it is smooth at every $p \in M$.
\end{definition}

In higher dimensional real analysis one has to replace the derivative $f'$ by the linear
map $Df$ i.e. the differential of $f$, which can be interpreted as the best linear
approximation of $f$ in the sense of $Df|_p(v)$ being the best linear (in $v$)
approximation for $f(p + v)$. We would love to do the same thing, but we can't since a
smooth manifold in general does not have any vector space structure, making it impossible
to talk about directions or linearity.

The remedy for this is going to be the tangent space: For every point $p \in M$ one
defines a local tangent space $T_p M$ that is a $n$-dimensional vector space and can
be thought of as either a linear approximation of $M$ at $p$, the $n$-dimensional vector
space of "directions" at $p$ or as the directional derivatives at $p$ seen as functionals
on $C^\infty(M)$ fullfilling a "product-rule" TODO.

We sadly don't have the time to fully develop the theory of tangent spaces, so we have
to take the things we need as black boxes. $TM$ is the tangent space of $M$, taken as the
"disjoint union" of all local tangent spaces, it can also be equipped with a topology and
an atlas such that it is a $2n$-dimensional smooth manifold. A smooth map $f: M \to N$
induces a map $Df: TM \to TN$ that intuitively maps "changes" on $M$ to "changes" on $N$.

Although tangent spaces don't have a canonical basis, once we have chosen coordinates
$\mathbf{x} = (x_1, \dots, x_n)$ for $p \in M$ there is a standard choice of basis of
$T_pM$ denoted as 
\begin{align*}
	\frac{\partial}{\partial \mathbf{x}} = 
		\left\{\frac{\partial}{\partial x_1}, \dots, \frac{\partial}{\partial x_n}\right\}
\end{align*}
% TODO: M \to N or only M \to \RR
that acts on a smooth $f: M \to \RR$ as
\begin{align*}
	\frac{\partial}{\partial x_1} f = (f \circ \mathbf{x}^{-1}(\mathbf{x}(p) + t e_1))'(0)
\end{align*}

\section{Differential Forms}
While we established a way to define differentiation of smooth functions on manifolds,
it remains to define its counterpart: Integration. We delayed this for a reason: While
the classic concept of differentiation can be unambigously generalized to manifolds and
multiple dimensions, the concept of integration can be generalized in quite different ways.
And this is not by coincidence: The 1-d Integral actually fullfills three distinct roles,
which grow distinct when taken to multiple dimensions (while still being interconnected).

In the case of $\RR$ Integration is a means of measuring area (this is generalized by the familiar
Lebesgue Integral TODO: Riemann Curvature Integral), of representing antiderivates, that is
a function $F$ defined by the equation $F' = f$ (this is generalized by (partial) differential
equations) and last but not least a means of calculating a signed area/"work" (which is
finally generalized by differential forms)

The difference between signed and unsigned area is substanstial, even it might seem minute,
since it requires one to integrate over orientated areas. In the easiest 1d-case this is intuitivly
expressed by exchanging the limits:
\[
	\int_a^b = -\int_b^a	
\]
And it makes sense: The work required of following a path from a to b should be the opposite of
following the same path backwards. (While one would want the length of the path to be independent
from the direction taken).

Conviently this concept of signed area is not a new thing, one recalls from linear algebra that
the determinant of a $n \times n$ matrix can be interpreted as the signed change of area of a
$n$ parallelopiped under the linear transformation. And indeed, our theory of differential
forms is going to start with its linear algebraic predecessor: Forms, which we are going
to use to define the concept of a "weighted" $n$-volume of a $n$-parallelopiped. This concept
can then by generalized to smooth manifolds by using tangent spaces as local linearizations
of a smooth manifolds, allowing us to talk rigourisly about "infinitisimal" volumes without
having to take limits. After that we finally are going to be able to enjoy the fruits of our
labor by defining de Rham cohomology and developing its basic computational tools.

\begin{remark}
The power of differential geometry often lies in using the (hopefully)
familiar techniques of linear algebra to analyse smooth functions
% 
\end{remark}

% TODO: The upper section is kind of ugly
\subsection*{Forms}
When defining integration one usually starts to talk about "infinitisimal"
displacements (or in higher dimensions parallelepipeds), making everything
precise using limits. But we already have a way to talk about an "infinititisimal"
change in a direction: The tangent space at a point precisely presents that.

Because of this logic it makes sense to start the definition of differential forms
as a locally defined function on the tangent space. Also: One important property of
smooth objects is that they are "locally linear", hence it might even be enough to
limit ourselves to linear maps of the tangent space.

We therefore define a differential 1-form at a point $p$ to be a linear map from
$T_pM$ to $\RR$.

This logic upholds for line-integrals quite well, but is there any way to use the
tangent space to represent higher dimensional shapes? Since we can't assume there
to be any canonical definition of matrix, we have to limit ourselves to shapes that
can be defined just using the linear algebraic properties of the tangent space. The
most obvious shape is therefore the $k$-parallelepiped, that we can represent using
an $k$-tuple of vectors $\xi_1, \dots, \xi_k \in T_pM$. Following the linearity of 
the 1-dimensional case, we also require a differential $k$-form to be linear in all
its arguments. For $k \geq 2$ there is also a special class of parallelepipeds,
the degenerate ones which have two or more linear dependent sides. These don't have
any $k$-volume and hence we require our differential forms to map these to zero.
One consequence of this is that for any $\xi_1, \xi_2 \in \RR^n$ we have:
\begin{align*}
&&	 0 &= \omega(\xi + \eta, \xi + \eta) \\
&&	   &= \omega(\xi, \xi + \eta) + \omega(\eta, \xi + \eta) \\
&&	   &= \underbrace{\omega(\xi, \xi)}_{= 0} + \omega(\xi, \eta) + \omega(\eta, \xi) + \underbrace{\omega(\eta, \eta)}_{= 0} \\
&&	   &= \omega(\xi, \eta) + \omega(\eta, \xi) \\
&\iff& \omega(\xi,\eta) &= - \omega(\eta,\xi)
\end{align*}
hence we require differential forms to be skew-symmetric. In fact skew-symmetry is
not only a necessary, but also a sufficient condition for forms to map linear
dependent vectors to zero.
\begin{lemma}
	Let $\omega$ be a skew-symmetric $k$-form on $\RR^n$ and $\xi_1, \dots, \xi_k$
	be lineary dependent. Then $\omega(\xi_1, \dots, \xi_k) = 0$
\end{lemma}
\begin{proof}
	Without loss of generality we assume that $\xi_1 = \lambda_2 \xi_2 + \dots + \lambda_k \xi_k$
	for some $\lambda_2, \dots, \lambda_k \in \RR$. Then:
	\begin{align*}
		\omega(\xi_1, \dots, \xi_n)
		&=
		\omega(\lambda_2 \xi_2 + \dots + \lambda_k \xi_k, \xi_2, \dots, \xi_k) \\
		&=
		\lambda_2 \omega(\xi_2, \xi_2, \dots, \xi_k) + \dots + \lambda_k \omega(\xi_k, \xi_2, \dots, \xi_k)
	\end{align*}
	All of these terms have at least two equal vectors and hence by skew-symmetry are equal to their
	negative value, which is only possible if they are zero. Therefore the sum is zero.
\end{proof}
This effort culminates in our local definition of a differential form:
\begin{definition}
Let $M$ be a smooth $n$-manifold and $p \in M$. A $k$-differential form (with $k \leq n$)
at $p$ is a skew-symmetric multi-linear map from $(T_p M)^k$ to $\RR$.
\end{definition}

Now some notation. Given a basis $\frac{\partial}{\partial x_1}, \dots, \frac{\partial}{\partial x_n}$
of a tangent space $T_p M$, we denote the dual basis as $\mathrm{d}x_1, \dots, \mathrm{d}x_n$, i.e.:
\[
	\mathrm{d}x_i(\frac{\partial}{\partial x_j}) = \delta_{i,j}
\]
We then recall from linear algebra that every linear map $\omega$ can be uniquely written as:
\[
	\omega = c^i \mathrm{d}x_i	
\]
for $c^1, \dots, c^n \in \RR$, so we usually denote differential forms like that
(but always remember that this is a non-canonical, coordinate dependent notation).

We can add more structure to this. 


Let's imagine that we have two differential forms $\mathrm{d}x_1$ and $\mathrm{d}x_2$.
Both of these assign a "weight" to the $\frac{\partial}{\partial  x_1}$/$\frac{\partial}{\partial x_2}$ direction,
so there should be a way to assign a "matching" weight to the parallelogram spanned by these vectors.
In fact we will denote this 2-form as $\mathrm{d}x_1 \wedge \mathrm{d}x_2$

TODO: k-forms representation, smoothness, exterior product, definition of integration

\subsection{The exterior deriative}
One of the most important aspects of integration on $\RR$ is the
fundamental theorem of calculus:
\[
	\int_a^b f'(t)\mathrm{d}t = f(b) - f(a)	
\]
To generalize this to higher dimensions we first note that the points $a,b$ are the
boundary of the path from $a$ to $b$. Points can be taken as 0-dimensional manifolds
and hence the concept of induced boundary notation can be extended to them, in absence
of different orders of a single point we take the order of a point to be either a plus
or minus sign. Hence the path from $a$ to $b$ induces a positive orientation on $b$
and a negative one on $a$, equal to the signs in the fundamental theorem. Even
going further we can define the integration of a smooth function $f$ on a 0-dimensional
oriented manifold $M$ as:
$
	\int_M f = \sum_{p \in M} \sign(p) f(p)
$
where $\sign(p)$ represents the orientation of $p$. This allows us to rephrase the
fundamental theorem as a statement purely about integration:
\[
	\int_a^b f'(t)\mathrm{d}t = \int_{\{-a, b\}} f
\]
since a differential $k$-form can be integrated on $k$-manifolds, we are going to
call smooth functions differential $0$-forms.

The fundamental theorem of calculus in 1-dimension fits in the picture of differential
forms as a measurement of change. Given a $0$-form $f$ on points, the integral of the
one form $f'$ on a path $\gamma$ from $a$ to $b$ measures the total change of $f$
on that path. Generalizing this we ask whether for a given $k$-form $\omega$ there
is a $k + 1$-form $\eta$ such that for every $k+1$-manifold $M$ we have:
\[
	\int_M \eta = \int_{\partial M} \omega
\]
\section{De-Rham Cohomology}
After spending some time to get our definitions in order we want to talk about
our original problem: Detecing holes. Our most basic definition of an hole is
in the $0$-dimensional case: The number of connected components of a manifold.
Since this is the most basic case we would hope to classify $0$-forms in some
way to detect this. One special property of a connected manifold is that every
locally constant function is globally constant. A function being locally constant
is something that we can easliy measure using differential forms: A function $f$ is
locally constant iff $\mathrm{d}f = 0$. Since $\mathrm{d}$ is a ring homomorphism
on $\Omega^0(M)$ its kernel $\ker(\mathrm{d})$ is a subring and we can define
the 0-th De-Rham Cohomology as:
\[
	H^0(M) \coloneqq \ker(\mathrm{d}) \cap \Omega^0(M)
\]
We find that the dimension of this is exactly equal to the number of connected components
of $M$, since every locally constant function is defined by it's value on one point of a
connected component.

It is a priori not clear how to generalize the concept of "local constness" to higher differential
forms, but we can find a generalization by changing how we phrase "local constness".
A function is locally const iff for a given point $p$ there is a neighborhood $U$ of $p$ such that for
every $q \in U$:
\[
	f(p) = f(q) \iff 0 = f(p) - f(q) = \int_{\{-q, p\}} f = \int_\gamma \mathrm{d} f
\]
where $\gamma$ is any loop from $p$ to $q$. The "interesting" forms (forms that only appear
if the manifold is not connected) are now the ones for which this is true locally, but not globally
i.e. there are points $p, q \in M$ such that:
\[
	\int_{-q, p} \omega \neq 0
\]

Taking this into higher dimensions, we can look at the differential $k$-forms in the kernel
of $\mathrm{d}$. Given a differential $k$-form $\omega$ with $\mathrm{d}\omega$, its local property
now is that for every point $p \in M$ there is neighborhood $\mathcal{U}$ such that the integral
of $\omega$ over any compact oriented $k$-submanifold $M$ without boundary is zero. This is because
locally every such manifold is the boundary of a compact oriented $k+1$-manifold $B$

TODO: This is not a trivial statement

and hence by Stokes:
\[
	\int_M \omega = \int_B \mathrm{d}M
\]
Since these are important, they have a special name
\begin{definition}
Let $M$ be a smooth manifold. A differential form $\omega$ on $M$ is called \textbf{closed}
if $\mathrm{d} \omega = 0$
\end{definition}

The interesting question now becomes for which closed forms this is \textbf{not} true
globally. It is easier to answer the converse: For which $k$-forms is this true globally.
One important fact about the exterior derivative is, that it is nilpotent, hence a large
class of closed forms are the forms that are the results of exterior differentiation and in
fact it holds for every $k$-form which can be written $\mathrm{d}\omega$ for some $k-1$ form $\omega$,
that for any compact oriented manifold $M$ without boundary:
\[
	\int_M \mathrm{d}\omega = \int_{\partial M} \omega = 0
\]
These are the "boring" forms, which we also denote by a special name.
\begin{definition}
Let $M$ be a smooth manifold, $\omega$ a differential form. We call $\omega$ exact if there is a differential
form $\eta$ such that $\omega = \mathrm{d}\eta$.
\end{definition}

We are thus interested in the closed forms modulo the exact ones, in general we define:
\begin{definition}
Let $M$ be a smooth manifold. We define the \textbf{de-Rham cohomology of $M$} as the graded
algebra:
\[
	H^*(M) \coloneqq \bigoplus_{k = 0}^\infty \ker(\mathrm{d}^k)/\Img(\mathrm{d}^{k - 1})
\]
\end{definition}
\begin{remark}
Elements of $H^*(M)$ are equivalence classes of elements of $\Omega^*(M)$, which we will
denote as $[c]$ for some $c \in \Omega^*(M)$ as a representative. But since we are going to
talk a lot about cohomology we are going to sometimes abuse notation and write $c$ to mean $[c]$
when the intent is clear. One just has to remember that the cohomology consists of equivalence classes
and that hence such a choice is not unique in general.
\end{remark}

% TODO: Graded Algebra? Exterior product?
Our previous discussion already allows us to calculate the zero-th de-Rham cohomology
\begin{lemma}
Let $M$ be a smooth manifold. Then the zeroth de-Rham cohomology is
\[
	H^0(M) \simeq \bigoplus_{c \in \pi_0(M)} \mathbb{R}
\]
where $\pi_0(M)$ are the connected components of $M$.

The exterior product corresponds to memberwise multiplication.
\end{lemma}
\begin{proof}
TODO
\end{proof}
%TODO: Sollte vermutlich erklärt werden
\begin{lemma}
Let $M$ be a smooth $n$-manifold. Then for every $k > n$:
\[
	H^k(M) = 0
\]
\end{lemma}
\begin{proof}
For $k > n$ the only skew-symmetric $k$-form on a $n$-dimensional vector space is $0$, hence
$\Omega^k(M) = 0$, which implies $H^k(M) = 0$.
\end{proof}.

% TODO: Pullback of forms

\subsection{The Mayer-Vietoris Sequence}
With our current knowledge it seems impossible to calculate the de-Rham cohomology for anything but the most simplest of
spaces, hence we have to develop more computational tools. We are going to start with the most important one: A way to
calculate the cohomology of more complicated spaces by splitting them into simpler ones, whose cohomology is known.
Concretely we are going to find a connection between the cohomology of a space $X$ and the cohomology of open subsets $\U, \V$ that cover $X$
i.e. $\U \cup \V = X$.

That connection is going to be the \textbf{Mayer-Vietoris-Sequence}, derived from the sequence of inclusions:
\begin{center}
% https://tikzcd.yichuanshen.de/#N4Igdg9gJgpgziAXAbVABwnAlgFyxMJZABgBpiBdUkANwEMAbAVxiRAFkQBfU9TXfIRQBGclVqMWbADrSAqgAJZAYwhoATtCXSAat14gM2PASIAmMdXrNWiELMUq6abXq7iYUAObwioAGaaALZIFiA4EEiiIAxYYLYgcBCxUCDUABYwdKmIYEwMDNQ4dFgMbJDxaYnpWP44SAC00dZSdlgA+rJuBoEQIYhhEVHUcDV1wzF0AEYwDAAK-CZCIOpYXun1VpIJHQ76AcETQ4hkEjZsAFbcFFxAA
\begin{tikzcd}
M & \U \coprod \V \arrow[l, "j"] & \U \cap \V \arrow[l, "i_\V", shift left] \arrow[l, "i_\U"', shift right]
\end{tikzcd}
\end{center}
With $\U \coprod \V$ being the disjoint union, $j$ the inclusion from $\U$ and $\V$ into $M$
and $i_\U$ being the inclusion of $\U \cap \V$ into $\U$ (and then into the disjoint union),
similiar with $i_\V$. All of these are smooth maps and hence this induces another sequence of
restrictions of differential forms:
\begin{equation}
% https://tikzcd.yichuanshen.de/#N4Igdg9gJgpgziAXAbVABwnAlgFyxMJZABgBpiBdUkANwEMAbAVxiRAB12B5AWxgHM6APQBUACgCyAShABfUuky58hFAEZyVWoxZtOvAcPGcAqlIAEnAEZZ+ENMziXufQaLGcAajPmLseAiIAJk1qemZWRA4XQ3dTZwBjOjRnbzktGCh+eCJQADMAJwgeJA0QHAgkEJAGLDBIkDgIWqgQagALGDpWxDAmBgZqHDosBjZIerbG9qw8nFKwnQasAH0vUSmGOisYBgAFJQDVEALbdvnfEELihfLKxGq4GbmkAFoyrZ39w5U2BhgXosImxVqYNpdriVEGQ7rdwroogArcEUWRAA
\begin{tikzcd}
\Omega^*(M) \arrow[r, "j^*"] & \Omega^*(\U) \bigoplus \Omega^*(\V) \arrow[r, "i_\V^*"', shift right] \arrow[r, "i_\U^*", shift left] & \Omega^*(\U \cap \V)
\end{tikzcd}
\label{cd:de-rham:mv-first}
\end{equation}
This does not quite get us anywhere, but we recall from (linear) Algebra that there are special
kind of sequences that provide a lot of information:
\begin{definition}
Let $A_i$ a sequence of vector spaces with homomorphism $f_i: A_i \to A_{i + 1}$:
\begin{center}
	% https://tikzcd.yichuanshen.de/#N4Igdg9gJgpgziAXAbVABwnAlgFyxMJZABgBpiBdUkANwEMAbAVxiRAB12oIcEBfUuky58hFAEZyVWoxZsAggH1gAawAEAWjXi+IAUOx4CRAExTq9Zq0Qglq3fpAZDoogGZzMqwuXqA1NoOgk7CRmLIACyelnI2nNy8etIwUADm8ESgAGYAThAAtkiSIDgQSCaOuQXl1KVIbpV5hYgeJWWIEY3ViGRtRXwUfEA
	\begin{tikzcd}
	\dots \arrow[r] & A_{k - 1} \arrow[r] & A_{k} \arrow[r] & A_{k + 1} \arrow[r] & \dots
	\end{tikzcd}
\end{center}
We call this sequence \textbf{exact} if $\Img(f_i) = \ker(f_{i + 1})$ for every $i \in \NN$.
If every $A_i$ but three are zero we call it a \textbf{short exact sequence}.
\end{definition}
Now \ref{cd:de-rham:mv-first} being exact can be seen as the statement that
the cohomology of a space is the cohomology of its parts, modulo their overlap.

This statement seems reasonable, but we have to modify it a bit, since the default restrictions would map a form $\omega$ on $M$
to the form $2\omega$ on $\U \cap \V$ which of course does not have to be zero. But we can make this nilpotent by arrigning some signs to arrive at the 
\textbf{Short Mayer Vietoris Sequence}:
\begin{equation}
	% https://tikzcd.yichuanshen.de/#N4Igdg9gJgpgziAXAbVABwnAlgFyxMJZARgBoAGAXVJADcBDAGwFcYkQAdDgeQFsYA5vQB6AKgAUAWQCUIAL6l0mXPkIoATBWp0mrdlz6CRErgFVpAAi4AjLAIhoWcKz35Cx4rgDVZCpdjwCIgBmLRoGFjZETlcjDzMXAGN6NBcfeUUQDADVInIwnUj2cgz-FSCUABYCiL1okr8s5UC1ZE1ibVqokE8OCDd6UhccemZfTOzy1tCO8N1urhHmCwBaF36jeW0YKAF4IlAAMwAnfqQyEBwIJE0QRiwwbrgIe6gQGgALGHo3xDBmRiMGgjLCMdiQR7vEBwD5YQ44JD5Qp1EBYMQAfW8qwsaNEmI4pihjHo1hgjAACs1ctFGDB4aUQCczogkVdznMitEAFZiBlM3hIUKXa4sxr8m7AkWVMWnAWIACskqQADZgfRQexeCk4Gy5JQ5EA
	\begin{tikzcd}
	0 \arrow[r] & \Omega^*(M) \arrow[r, "j^*"] & \Omega^*(\U) \bigoplus \Omega^*(\V) \arrow[r, "i^*_\V - i^*_\U"] & \Omega^*(\U \cap \V) \arrow[r] & 0 \\
	            &                              & {(\omega, \tau)} \arrow[r, maps to]                              & \tau - \omega                  &  
	\end{tikzcd}
	\label{cd:de-rham:mv-ses}
\end{equation}
Now the important statement becomes:
\begin{theorem}
The Short Mayer-Vietoris sequence (\ref{cd:de-rham:mv-ses}) is exact.
\end{theorem}
%TODO: I need to talk about partitions of unity somewhere
\begin{proof}
We already concluded that $j^*$ is injective and the kernel of $i^*_\V - i^*_\U$ consists
exactly of the differential forms that agree on $\U \cap \V$, hence
$\Img(j^*) = \ker(i^*_\V - i^*_\U)$.

It remains to show that $i^*_\V - i^*_\U$ is surjective, which we are going to do by
constructing a right-inverse. Let $\omega \in \Omega^*(\U \cap \V)$ and $\rho_\U, \rho_V$ a
partition of unity subordinate to $\U, \V$. Now $\rho_\mathcal{V} \omega$ (taken as pointwise
multiplication) can be zero-extended to $\U$, as can $\rho_\mathcal{U} \omega$ to $\V$. We then
have:
\begin{equation*}
	(i^*_\V - i^*_U)(-\rho_\V \omega, \rho_\U \omega) = (\rho_\U + \rho_\V)\omega = \omega
\end{equation*}
Therefore this map is surjective and the whole sequence is exact.
%TODO: Picture
\end{proof}
This is still a bit pointless since it is only a statement of differential complexes, not of
cohomologies. But all of these maps descend to maps on cohomlogies
\begin{theorem}[Mayer-Vietoris sequence]
	Let $M$ be a smooth manifold and $\U, \V$ open subsets, such that $M = \U \cup \V$.

	The short Mayer-Vietoris sequence
	\begin{equation*}
	% https://tikzcd.yichuanshen.de/#N4Igdg9gJgpgziAXAbVABwnAlgFyxMJZABgBpiBdUkANwEMAbAVxiRAB12B5AWxgHM6APQBUACgCyAShABfUuky58hFAEZyVWoxZtOvAcPGcAqlIAEnAEZZ+ENMziXufQaLGcAajPmLseAiIAJk1qemZWRA4XQ3dTZwBjOjRnbzktGCh+eCJQADMAJwgeJA0QHAgkEJAGLDBIkDgIWqgQagALGDpWxDAmBgZqHDosBjZIerbG9qw8nFKwnQasAH0vUSmGOisYBgAFJQDVEALbdvnfEELihfLKxGq4GbmkAFoyrZ39w5U2BhgXosImxVqYNpdriVEGQ7rdwroogArcEUWRAA
	\begin{tikzcd}
	\Omega^*(M) \arrow[r, "j^*"] & \Omega^*(\U) \bigoplus \Omega^*(\V) \arrow[r, "i_\V^*"', shift right] \arrow[r, "i_\U^*", shift left] & \Omega^*(\U \cap \V)
	\end{tikzcd}
	\end{equation*}
	induces a long exact sequence sequence in cohomology, called \textbf{Mayer-Vietoris} sequence:
	\begin{equation}
		% https://tikzcd.yichuanshen.de/#N4Igdg9gJgpgziAXAbVABwnAlgFyxMJZARgBoAGAXVJADcBDAGwFcYkQAJAPQGsAKALIBKEAF9S6TLnyEUAJgrU6TVu278AOhoCqQgARaARlgDmENCzh71fLQDUR4ydjwEiAZkU0GLNok68tjoGGgDG9GghDmISIBguMkTkXsq+7FpQEDgITnFSrrIkpMRKPqr+3MA8egDUesSigo6x8dJu8sWlKn4BVbX1jVq6IcZmFsxWldV1DUHRua0FHp3e3WpcfTODwVrhkfbNzm2FACwrqeUgGVk5SjBQJvBEoABmAE4QALZIZCA4EEgFCBGFgwD04BAQVAQDQABYwejQxBgZiMRg0HD0LCMdiQMEwkBwWFYF44JDJYH0QwwRgABXyiX8jBgpJirw+30QFP+P1y7y+SE8fwBXL5HKQADYMSKAOxigWIM7CpAAVnlnJV0slokooiAA
		\begin{tikzcd}[column sep=small]
			\usetikzlibrary{fadings}
			&\arrow[
				d,
				"\dots"{description},
				dash pattern=
					on 0.5pt
					off 8pt
					on 1pt
					off 7pt
					on 2pt
					off 6pt
					on 4pt
					off 5pt
					on 6pt
					off 4pt
					% on 8pt
					% off 3pt
					% on 10pt
					% off 1pt
					on 100pt,
				rounded corners,
				to path={
					([xshift=3ex]\tikztostart.west)
					-- (\tikztostart.west)
					-- ([xshift=-3ex]\tikztostart.west)
					-| ([xshift=-3ex]\tikztotarget.west)
					-- (\tikztotarget)
				}
				]
			&\\[-1.6em]
			& H^k(M)
				\arrow[r]
			& H^k(\U) \bigoplus H^k(\V)
				\arrow[r]
				% \arrow[u, ""{coordinate, name=U}]
				\arrow[d, phantom, ""{coordinate, name=Z}]
			& H^k(\U \cap \V)
				\arrow[
					dll,
					"d^*"{pos=1.0, description},
					rounded corners,
					to path={
						-- ([xshift=2ex]\tikztostart.east)
						|- (Z) [near end]\tikztonodes
						-| ([xshift=-2ex]\tikztotarget.west)
						-- (\tikztotarget)
					}]
			&\\
            & H^{k + 1}(M)
				\arrow[r]
			& H^{k + 1}(\U) \bigoplus H^{k + 1}(\V)
				\arrow[d, phantom, ""{coordinate, name=D}]
				\arrow[r]
			& H^{k + 1}(\U \cap \V)
				\arrow[
					d,
					""{description},
					dash,
					dash pattern=
						on 0.5pt
						off 8pt
						on 1pt
						off 7pt
						on 2pt
						off 6pt
						on 4pt
						off 5pt
						on 6pt
						off 4pt
						% on 8pt
						% off 3pt
						% on 10pt
						% off 1pt
						on 100pt,
					rounded corners,
					to path={
						([xshift=-6ex]\tikztotarget.east)
						-| ([xshift=2ex]\tikztostart.east)
						-- (\tikztostart.east)
					}
				]
			&\\[-1.3em]
			& {} & {} & {}
		\end{tikzcd}
	\end{equation}
\end{theorem}
The easiest way to prove this is to actually first prove a much more abstract algebraic result,
which is going to include our theorem as a special case. But before we can do this we have to
define some algebraic basics.
\begin{definition}[Cochain Complex]
A \textbf{cochain complex} is a graded vector space $A^*$ together with maps
\begin{align*}
	\rmd^k: A^k \to A^{k + 1}
\end{align*}
which together define a map $\rmd: A^* \to A^*$ in the obvious way, that fullfills:
\begin{align*}
	\rmd \circ \rmd = 0
\end{align*}
The map $\mathrm{d}$ will usually be called \textbf{differential} or
\textbf{coboundary operator} of $A^*$.
\end{definition}
The de Rham complex is our most important example of cochain complex. We can even phrase
the concept of Cohomology in purely algebraic terms:
\begin{definition}[Cohomology]
Let $A^*$ be a cochain complex. We define the \textbf{cohomology} of $A^*$ as the graded
vector space:
\begin{align*}
H^k(A^*) \coloneqq \ker(\rmd^k) / \Img(\rmd^{k - 1})
\end{align*}
We will call $\ker(\rmd^k)$ $k$-cocyles of $A^*$ and $\Img(\rmd^{k - 1})$
$k$-coboundaries of $A^*$.
\end{definition}
The only missing ingredients remaining to give the more general result is a general way to
talk about maps between cochain complexes
\begin{definition}[cochain map]
Let $(A^*, \rmd_A^*), (B^*, \rmd_B^*)$ be cochain complexes and $f^k: A^k \to B^k$ a vector
space homomorphism for every $k$. We call $f^k$ \textbf{cochain map} if it commutes with
$\rmd_A^*$/$\rmd_B^*$ i.e. if the following diagram commutes:
\[\begin{tikzcd}
	{A^*} & {A^{* + 1}} \\
	{B^*} & {B^{* + 1}}
	\arrow["{\mathrm{d}_A}", from=1-1, to=1-2]
	\arrow["{f^*}"', from=1-1, to=2-1]
	\arrow["{f^{* + 1}}"', from=1-2, to=2-2]
	\arrow["{\mathrm{d}_B}"', from=2-1, to=2-2]
\end{tikzcd}\]
(this introduces a common abuse of notation: Writing $A^*, A^{* + 1}$ to mean that these can be
replaced by $A^k, A^{k + 1}$ for any $k$.
)
\end{definition}
Cochain maps are so important because they induce well-defined maps in cohomology:
\begin{lemma}
Let $(A^*, \rmd_A), (B^*, \rmd_B)$ be two cochain complexes and $f^*: A^* \to B^*$ be a chain
map. Then the homomorphism $\tilde{f}^*: H^*(A^*) \to H^*(B^*)$ is well-defined.
\end{lemma}
\begin{proof}
$f^*$ descends to a map $\tilde{f}^*$ on $H^*(A^*)$ since this is just a quotient of a
subspace of $A^*$. It remains to check that $\Img(\tilde{f}^*) \subseteq H^*(B^*)$. Let
$a \in \ker(\rmd_A)$, then:
\begin{align*}
\rmd_B(\tilde{f}^*(a)) = \tilde{f}^*(\rmd_A a) = 0
\end{align*}
Hence $\Img(\tilde{f}^*) \subseteq H^*(B^*)$.
\end{proof}
By abuse of notation we will usually refer to $f^*$ and $\tilde{f}^*$ as just $f^*$.

By replacing vector spaces with cochain complexes and maps with cochain complexes we can
talk about a short exact sequence of cochain complexes and thus finally phrase our general
statement:
\begin{theorem}
Let
% https://q.uiver.app/#q=WzAsNSxbMSwwLCJBXioiXSxbMCwwLCIwIl0sWzIsMCwiQl4qIl0sWzMsMCwiQ14qIl0sWzQsMCwiMCJdLFsxLDBdLFswLDIsImZeKiJdLFsyLDMsImdeKiJdLFszLDRdXQ==
\[\begin{tikzcd}
	0 & {A^*} & {B^*} & {C^*} & 0
	\arrow[from=1-1, to=1-2]
	\arrow["{f^*}", from=1-2, to=1-3]
	\arrow["{g^*}", from=1-3, to=1-4]
	\arrow[from=1-4, to=1-5]
\end{tikzcd}\]
be a short exact sequence of chain complexes. Then there exists a connecting morphism
$\rmd^*: C^* \to A^{* + 1}$ such that the sequence
% https://q.uiver.app/#q=WzAsNixbMSwwLCJIXiooQV4qKSJdLFsyLDAsIkheKihCXiopIl0sWzMsMCwiSF4qKENeKikiXSxbNCwwLCJIXnsqICsgMX0oQV4qKSJdLFs1LDAsIlxcZG90cyJdLFswLDAsIlxcZG90cyJdLFswLDEsImZeKiJdLFsxLDIsImdeKiJdLFsyLDMsIlxcbWF0aHJte2R9XioiXSxbMyw0XSxbNSwwXV0=
\[\begin{tikzcd}[column sep=small]
	\dots & {H^*(A^*)} & {H^*(B^*)} & {H^*(C^*)} & {H^{* + 1}(A^*)} & \dots
	\arrow[from=1-1, to=1-2]
	\arrow["{f^*}", from=1-2, to=1-3]
	\arrow["{g^*}", from=1-3, to=1-4]
	\arrow["{\mathrm{d}^*}", from=1-4, to=1-5]
	\arrow[from=1-5, to=1-6]
\end{tikzcd}\]
is exact.
\end{theorem}
\begin{proof}
We first define a candidate for the morphism, then prove that it is well-defined and then,
that it indeed makes the sequence exact. Let $[c] \in H^k(C^*)$. $g^k$ is surjective, hence
we can pick a $b \in B^k$ such that $g^k(b) = c$. Since $g^*$ is a cochain map we can put
these in a commutative diagram together with $\mathrm{d}_B b$ and $\rmd_C c$ (which is by
definition zero):
% https://q.uiver.app/#q=WzAsNCxbMCwwLCJiIl0sWzAsMSwiXFxtYXRocm17ZH1fQiBiIl0sWzEsMSwiMCJdLFsxLDAsImMiXSxbMCwxLCJcXG1hdGhybXtkfV9CIiwyXSxbMSwyLCJnXntrICsgMX0iXSxbMCwzLCJnXmsiLDJdLFszLDIsIlxcbWF0aHJte2R9X0MiXV0=
\[\begin{tikzcd}[column sep=small]
	b & c \\
	{\mathrm{d}_B b} & 0
	\arrow["{g^k}"', from=1-1, to=1-2]
	\arrow["{\mathrm{d}_B}"', from=1-1, to=2-1]
	\arrow["{\mathrm{d}_C}", from=1-2, to=2-2]
	\arrow["{g^{k + 1}}", from=2-1, to=2-2]
\end{tikzcd}\]
Thus $g(\mathrm{d}_B b) = 0$ and by exactness there is a $a \in A^{k + 1}$ such that 
$f(a) = \rmd_B b$. We define the connecting map as the map from $[c]$ to $[a]$.

There was one choice we made in this construction: Picking a specific $b$ (picking $a$ did not
involve a choice since $f^*$ is injective on the cochain complexes). To prove that that $[a]$
is independent of this choice let $\tilde{b} \in B^k$ be another element such that
$g^k(\tilde{b})=c$. Then $g^k(b - \tilde{b}) = 0$ and thus their exists a $\hat{a} \in A^k$
such that $f^k(a) = b - \tilde{b}$ and thus by commutatitivy:
% https://q.uiver.app/#q=WzAsNixbMSwwLCJiIC0gXFx0aWxkZXtifSJdLFsxLDEsIlxcbWF0aHJte2R9X0IoYiAtIFxcdGlsZGV7Yn0pIl0sWzIsMSwiMCJdLFsyLDAsIjAiXSxbMCwwLCJcXGhhdHthfSJdLFswLDEsIlxcbWF0aHJte2R9X0FcXGhhdHthfSJdLFswLDEsIlxcbWF0aHJte2R9X0IiLDJdLFsxLDIsImdee2sgKyAxfSIsMl0sWzAsMywiZ15rIl0sWzMsMiwiXFxtYXRocm17ZH1fQyJdLFs0LDAsImZeayJdLFs0LDUsIlxcbWF0aHJte2R9X0EiLDJdLFs1LDEsImZee2sgKyAxfSIsMl1d
\[\begin{tikzcd}[column sep=small]
	{\hat{a}} & {b - \tilde{b}} & 0 \\
	{\mathrm{d}_A\hat{a}} & {\mathrm{d}_B(b - \tilde{b})} & 0
	\arrow["{f^k}", from=1-1, to=1-2]
	\arrow["{\mathrm{d}_A}"', from=1-1, to=2-1]
	\arrow["{g^k}", from=1-2, to=1-3]
	\arrow["{\mathrm{d}_B}"', from=1-2, to=2-2]
	\arrow["{\mathrm{d}_C}", from=1-3, to=2-3]
	\arrow["{f^{k + 1}}"', from=2-1, to=2-2]
	\arrow["{g^{k + 1}}"', from=2-2, to=2-3]
\end{tikzcd}\]
And since $[\rmd_A] = 0$ by definition, both $b$ and $\tilde{b}$ lead to the same $[a]$, hence
$\mathrm{d}^*$ is well-defined.

Proof of exactness starts with proving that $\Img(f^k) = \ker(g^k)$ in cohomology. One part of
this is trivial:
\begin{align*}
	(g \circ f)([a]) = [(g \circ f)(a)] = [0] = 0
\end{align*}
For the reverse let $[b] \in H^k(B^*)$ such that $g^k([b]) = [0]$. That is the case if and only if
$g^k(b) = \rmd_C c$ for some $c \in C^{k - 1}$. By surjectivity there is then a
$\hat{b} \in B^{k - 1}$ such that $g^{k - 1}(\hat{b}) = c$. By commutativity then
$g(\rmd_B \hat{b}) = \rmd_C g(\hat{b}) = \rmd_C c$ and thus $g^k(b - \rmd_B \hat{b}) = 0$.
By exactness of the original sequence there exists therefore a $a \in A^k$ such that
$f^k(a) = b - \rmd_B \hat{b}$ and:
\begin{align*}
	0 = \rmd_B (b - \rmd_B \hat{b}) = \rmd_B f^k(a) = f^{k + 1}(\rmd_A a)
\end{align*}
Hence by injectivity $\rmd_A a = 0$ and thus $[a] \in H^k(A^*)$. Especially:
\begin{align*}
	f^k([a]) = [b - \rmd_B \hat{b}] = [b] - [\rmd_B \hat{b}] = [b]
\end{align*}
Hence $\Img(f^k) = \ker(g^k)$.

Next we prove that $\Img(g^k) = \ker(\rmd^*)$. Let $[b] \in H^k(B^*)$, looking at our
construction of $\rmd^*$ we see that $\rmd_B b = 0$ and hence $a = 0$ by injectivity
and hence $\rmd^* [g^k(b)] = [a] = 0$ and $\ker(\rmd^*) \subseteq \Img(g^*)$. Also, if $\mathrm{d}^* [c] = [0]$, then our
construction becomes:
% https://q.uiver.app/#q=WzAsNSxbMCwxLCJcXG1hdGhybXtkfV9BIGEiXSxbMCwwLCJhIl0sWzEsMSwiXFxtYXRocm17ZH1fQiBiIl0sWzEsMCwiYiJdLFsyLDAsImMiXSxbMSwwLCJcXG1hdGhybXtkfV9BIiwyXSxbMCwyLCJmXntrICsgMX0iLDJdLFszLDIsIlxcbWF0aHJte2R9X0IiXSxbMyw0LCJnXmsiXV0=
\[\begin{tikzcd}
	a & b & c \\
	{\mathrm{d}_A a} & {\mathrm{d}_B b}
	\arrow["{\mathrm{d}_A}"', from=1-1, to=2-1]
	\arrow["{g^k}", from=1-2, to=1-3]
	\arrow["{\mathrm{d}_B}", from=1-2, to=2-2]
	\arrow["{f^{k + 1}}"', from=2-1, to=2-2]
\end{tikzcd}\]
Since $g^k \circ f^k = 0$ it follows that $g^k(b - f^k(a)) = c$ and since $f^*$ is a chain
map $\rmd_B f^k(a) = \rmd_B b$, thus:
\begin{align*}
\rmd_B(b - f^k(a)) = 0
\end{align*}
Hence $[c] \in \Img(g^k)$ and especially $\Img(g^*) = \ker (\rmd^*)$.

The last step is to proof that $\Img(\mathrm{d}^*) = \ker(f^*)$. Let $[c] \in H^k(C^*)$, then
$f^{k + 1}(\rmd^* c) = [\rmd_B b]$ for some $b \in H^k(B^*)$ by construction, hence
$\Img(\mathrm{d}^*) \subseteq \ker(f^*)$. The reverse works similiar, if
$[a] \in \ker(f^{k + 1})$ then there is a $b \in B^k$ such that
$f^{k + 1}(a) = \rmd_B^{k + 1}b$, then by our construction $\rmd^*([g^{k}(b)]) = [a]$ and
therefore $\Img(\rmd^*) = \ker(f^*)$.
\end{proof}

It is constructive to translate this result back into the language of de Rham cohomology
using a concrete example. Let us look at the circle $S^1$ and compute its cohomology
using this sequence. We can cover $S^1$ by two open sets $\U, \V$ as seen in TODO, then
$H^0(\U \cap \V) \simeq \RR^2$ by TODO and $H^k(\U) = H^k(\V) = H^k(\U \cap \V) = 0$ for
every $k \geq 1$ by TODO. Thus we get the sequence:
% https://q.uiver.app/#q=WzAsNixbMCwwLCIwIl0sWzEsMCwiSF4wKFNeMSkiXSxbMiwwLCJIXjAoXFxVKSBcXG9wbHVzIEheMChcXFYpIl0sWzMsMCwiSF4wKFxcVSBcXGNhcCBcXFYpIl0sWzQsMCwiSF4xKFNeMSkiXSxbNSwwLCIwIl0sWzAsMV0sWzEsMl0sWzIsM10sWzMsNF0sWzQsNV1d
\[\begin{tikzcd}[column sep=small]
	0 & {H^0(S^1)} & {H^0(\U) \oplus H^0(\V)} & {H^0(\U \cap \V)} & {H^1(S^1)} & 0
	\arrow[from=1-1, to=1-2]
	\arrow[from=1-2, to=1-3]
	\arrow[from=1-3, to=1-4]
	\arrow[from=1-4, to=1-5]
	\arrow[from=1-5, to=1-6]
\end{tikzcd}\]
It is already known that $H^0(S^1)$ consists of the const functions on $S^1$ (since $S^1$
is connected) and the restriction of these functions are obviously const themselves
% TODO: First algebraic, then geometric
% TODO: Some computational examples

