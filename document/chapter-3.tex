\chapter{The Čech-de Rham Double Complex}
\epigraph{
  Algebra is the offer made by the devil to the mathematician. The devil says: I will give you
  this powerful machine, it will answer any question you like. All you need to do is give me
  your soul: give up geometry and you will have this marvellous machine.
}{Sir Michael Atiyah}
So far Čech and de Rham Cohomology are similiar in name only: They are defined on different
spaces using different  machinery and seem to measure different things: Čech cohomology the
"holes" of a space and de Rham cohomology the solution spaces of differential equations modulo
trivial solutions.

But at the same time there is a good reason to hope for a connection betweem them both: We
already sketched in the introduction that the zeroth and first de Rham Cohomology vanish if
and only if the respective Čech Cohomologies vanish and in all of the examples we have seen
so far both cohomologies gave the same results. That this wasn't just a happy accident is
captured by de Rhams theorem:
\begin{theorem}[De Rhams theorem]
Let $M$ be a smooth manifold. There is a natural isomorphism between the de Rham cohomology of
$M$ and the Čech cohomology of $M$.
\end{theorem}
The proof of this statement is going to occupy most of the rest of this thesis. This is not
only a remarkable fact, but is also going to solve some non-trivial remaining issues we had
with both cohomologies. Our main problem with Čech cohomology was, that while the cohomology
of a given cover is easily calculatable, its limit is not. But a corollary of our proof is
going to be that the Čech cohomologies of good covers are isomorphic. Since good covers are
cofinal, their cohomology is therefore also isomorphic to the Čech cohomology of $M$ itself,
allowing an easy calculation of de Rham and Čech cohomology based on combinatorial data.

Another question that this theorem answers is whether de Rham cohomology is a topological
invariant i.e. whether different non-diffeomorphic choices of a smooth structure on a manifold
may lead to different cohomologies. Since Čech cohomology by definition does not depend on any
smooth structure, de Rhams theorem implies:
\begin{corolarry}
Let $M, N$ be two homeomorphic smooth manifolds. Then the de Rham cohomology of $M$ is
naturally isomorphic to the de Rham cohomology of $N$.
\end{corolarry}

\section{Mayer Vietoris to the infinite}
To prove de Rhams theorem we are going to start with expressing the Čech Cohomology of a given
cover in terms of the de Rham cohomologies of the sets. Fix $M$ to be a smooth manifold with
good cover $\mathfrak{U} = \{\U_\alpha\}_{\alpha \in I}$ ($I$ being some totally ordered index
set). Then we denote the intersection of $k$ sets as
\[
	\U_{\alpha_1, \dots, \alpha_k} \coloneqq \U_{\alpha_1} \cap \dots \cap \U_{\alpha_k}
\]

$\hat{C}^k(\mathfrak{U})$ consists of the maps from $\hat{C}_k(\mathfrak{U})$ to $\RR$
i.e. of a product of copies of $\RR$ for every non-empty intersection $\U_{\alpha_0, \dots,
\alpha_k}$. Now denote the constant functions on a set $\U$ as $C_c(\U)$. This is a vector
space of dimension 1 if $\U$ is non-empty and of dimension $0$ if it is, therefore there is
an isomorphism:
\[
	\hat{C}^k
		\cong
			\prod_{
				\substack{
					\alpha_0 < \dots < \alpha_k \\
					\U_{\alpha_0, \dots, \alpha_k} \neq \emptyset
				}
			}
			\RR
		\cong
			\prod_{\alpha_0 < \dots < \alpha_k}
				C_c(\U_{\alpha_0, \dots, \alpha_k})
\]
Since $\mathfrak{U}$ is a good cover all $\U_{\alpha_0, \dots, \alpha_k}$ are connected and
thus the constant functions are exactly the locally constant ones. Given that locally constant
functions are just the $0$-th de Rham cohomology of a space, this induces another isomorphism:
\[
	\hat{C}^k(\U)
	\cong
	\prod_{\alpha_0 < \dots < \alpha_k} H_\dr^0(\U_{\alpha_0, \dots, \alpha_k})
\]
This already gives us a description of the Čech cocomplex using de Rham cohomologies, but
making use of this information requires finding tools that create a connection between the
cohomologies of open covers of smooth manifolds and the cohomology of the manifold itself.

Actually we already have such a tool, in form of the Mayer-Vietoris sequence, albeit only for
the special case of covers consisting of 2 open sets. Recall that this sequence built upon the
fact that
% https://q.uiver.app/#q=WzAsNSxbMCwwLCIwIl0sWzEsMCwiXFxPbWVnYV4qKE0pIl0sWzIsMCwiXFxPbWVnYV4qKFxcVSkgXFxiaWdvcGx1cyBcXE9tZWdhXiooXFxWKSJdLFszLDAsIlxcT21lZ2FeKihcXFUgXFxjYXAgXFxWKSJdLFs0LDAsIjAiXSxbMCwxXSxbMSwyLCJqXioiXSxbMiwzLCJpXipfXFxWIC0gaV4qX1xcVSJdLFszLDRdXQ==
\[\begin{tikzcd}
	0 & {\Omega^*(M)} & {\Omega^*(\U) \bigoplus \Omega^*(\V)} & {\Omega^*(\U \cap \V)} & 0
	\arrow[from=1-1, to=1-2]
	\arrow["{j^*}", from=1-2, to=1-3]
	\arrow["{i^*_\V - i^*_\U}", from=1-3, to=1-4]
	\arrow[from=1-4, to=1-5]
\end{tikzcd}\]
is a short exact sequence. We can rewrite this to be more compatible with arbitrary covers,
by writing $\U_1, \U_2$ instead of $\U, \V$ and replacing the sum of vector spaces with their
products (in the finite case these are equivalent anyway):
% https://q.uiver.app/#q=WzAsNSxbMCwwLCIwIl0sWzEsMCwiXFxPbWVnYV4qKE0pIl0sWzIsMCwiXFxjb3Byb2Rfe2sgPSAxfV4yIFxcT21lZ2FeKihcXFVfaykiXSxbMywwLCJcXE9tZWdhXiooXFxVX3sxMn0pIl0sWzQsMCwiMCJdLFswLDFdLFsxLDIsImpeKiJdLFsyLDMsImleKiJdLFszLDRdXQ==
\[\begin{tikzcd}
	0 & {\Omega^*(M)} & {\coprod_{k = 1}^2 \Omega^*(\U_k)} & {\Omega^*(\U_{12})} & 0
	\arrow[from=1-1, to=1-2]
	\arrow["{j^*}", from=1-2, to=1-3]
	\arrow["{i^*}", from=1-3, to=1-4]
	\arrow[from=1-4, to=1-5]
\end{tikzcd}\]
Where we are using our convention of $\U_{1,2} = \U_1 \cap \U_2$ and define
\[
	i^* \coloneqq \sum_{k = 1}^2 (-1)^k i^*_{k}
\]
with $i^*_k$ being the map induced by the inclusion $\U_{12} \hookrightarrow \U_1
\hookrightarrow \U \amalg \V$. While this is a quite convoluted way to write down this
sequence, it might make its generalization to arbitrary covers seem more natural:
% https://q.uiver.app/#q=WzAsNSxbMiwwLCJcXHByb2Rfe1xcYWxwaGF9IFxcT21lZ2FeKihcXFVfXFxhbHBoYSkiXSxbMywwLCJcXHByb2Rfe1xcYWxwaGFfMCA8IFxcYWxwaGFfMX0gXFxPbWVnYV4qKFxcVV97XFxhbHBoYV8wIFxcYWxwaGFfMX0pIl0sWzEsMCwiXFxPbWVnYV4qKE0pIl0sWzAsMCwiMCJdLFs0LDAsIlxcZG90cyJdLFswLDEsIlxcZGVsdGFeezAsKn0iXSxbMiwwLCJqXioiXSxbMywyXSxbMSw0XV0=
\[\begin{tikzcd}
	0 & {\Omega^*(M)} & {\prod_{\alpha} \Omega^*(\U_\alpha)} & {\prod_{\alpha_0 < \alpha_1} \Omega^*(\U_{\alpha_0 \alpha_1})} & \dots
	\arrow[from=1-1, to=1-2]
	\arrow["{j^*}", from=1-2, to=1-3]
	\arrow["{\delta^{0,*}}", from=1-3, to=1-4]
	\arrow[from=1-4, to=1-5]
\end{tikzcd}\]
where, as in the original, the morphisms are sums of restrictions with alternating signs,
though defining them in an elegant way requires developing some additional notation, starting
with a convenient shorthand:
\[
	\Omega^{k, \ell}(\mathfrak{U})
		\coloneqq \prod_{\alpha_0 < \dots < \alpha_k} \Omega^\ell(\U_{\alpha_0 < \dots < \alpha_k})
\]
which extends to $\Omega^{k, *}$ in the obvious way. To describe an $\omega \in \Omega^{k,
\ell} (\mathfrak{U})$ itself, we note that this is an element of a product space and thus
has "components", which we denote as $\omega_{\alpha_0 \dots \alpha_k}$. Taken as actual
components this notation only makes sense for increasing indices $\alpha_0 < \dots <
\alpha_k$, but we can extend this notation to all indices by defining for any $\sigma \in S_{k
+ 1}$:
\[
	\omega_{\alpha_{\sigma(0)} \dots \alpha_{\sigma(k)}}
		= (-1)^{|\sigma|} \omega_{\alpha_0 \dots \alpha_k}
\]
and $\omega_{\beta_0 \dots \beta_k} = 0$ if any indices are duplicate.

Now we can define the maps $\delta^k$ component-wise:
\[
	(\delta^k \omega)_{\alpha_0 \dots \alpha_k}
		\coloneqq \sum_{i = 0}^k (-1)^i \omega_{\alpha_0 \dots \hat{\alpha}_i \dots \alpha_k}
\]
where $\hat{\alpha}_i$ denotes the omission of that index. $\delta^k$ are thus the restriction
maps with an alternating sign, as in the original Mayer-Vietoris sequence. Note that we still
have to check that this definition of $\delta^k$ is compatible with our index convention,
for which it suffices to show this for a swap of the first two indices, i.e. that $(\delta^k
\omega)_{\alpha_0 \dots \alpha_k} = (\delta^k \omega)_{\alpha_1 \alpha_0 \alpha_2 \dots
\alpha_k}$. By a simple calculation:
\begin{align*}
	(\delta^k \omega)_{\alpha_1 \alpha_0 \alpha_2 \dots \alpha_k}	
		&= \omega_{\alpha_0 \alpha_2 \dots \alpha_k} - \omega_{\alpha_1 \dots \alpha_k}
			+ \sum_{i = 2} (-1)^i \omega_{\alpha_1 \alpha_0 \alpha_2 \dots \hat{\alpha}_i \dots \alpha_k } \\
		&= - \omega_{\alpha_1 \dots \alpha_k} + \omega_{\alpha_0 \alpha_2 \dots \alpha_k}
			- \sum_{i = 2} (-1)^i \omega_{\alpha_0 \dots \hat{\alpha}_i \dots \alpha_k} \\
		&= - \sum_{i = 0} (-1)^i \omega_{\alpha_0 \dots \hat{\alpha}_i \dots \alpha_k} \\
		&= - (\delta^k \omega)_{\alpha_0 \dots \alpha_k}
\end{align*}
The important statement is now that this sequence is exact:
\begin{lemma}
The sequence
% https://q.uiver.app/#q=WzAsNixbMiwwLCJcXE9tZWdhXnswLCAqfShcXG1hdGhmcmFre1V9KSJdLFszLDAsIlxcT21lZ2FeezEsICp9KFxcbWF0aGZyYWt7VX0pIl0sWzQsMCwiXFxPbWVnYV57MiwqfShcXG1hdGhmcmFre1V9KSJdLFs1LDAsIlxcY2RvdHMiXSxbMSwwLCJcXE9tZWdhXiooTSkiXSxbMCwwLCIwIl0sWzAsMSwiXFxkZWx0YV57MCwqfSJdLFsxLDIsIlxcZGVsdGFeezEsICp9Il0sWzIsMywiXFxkZWx0YV57MiwqfSJdLFs0LDAsImleKiJdLFs1LDRdXQ==
\[\begin{tikzcd}
	0 & {\Omega^*(M)} & {\Omega^{0, *}(\mathfrak{U})} & {\Omega^{1, *}(\mathfrak{U})} & {\Omega^{2,*}(\mathfrak{U})} & \cdots
	\arrow[from=1-1, to=1-2]
	\arrow["{i^*}", from=1-2, to=1-3]
	\arrow["{\delta^{0,*}}", from=1-3, to=1-4]
	\arrow["{\delta^{1, *}}", from=1-4, to=1-5]
	\arrow["{\delta^{2,*}}", from=1-5, to=1-6]
\end{tikzcd}\]
is exact.
\end{lemma}
\begin{proof}
We start with the map $i^*$.  This is just the restriction of differential forms to all the sets of $\mathfrak{U}$
and since $\mathfrak{U}$ covers $M$ this is injective. Since $\delta^{0, *}$ consist of two restriction with
opposite signs, its kernel are exactly the differential forms that agree on all intersections of sets in
$\mathfrak{U}$. The differential forms coming from the restriction of a form on $M$ are exactly those that
agree on all intersections, therefore $\ker(\delta^{0, *}) = \Img(i^*)$.

Continuing with the rest of the sequence, we show that $\delta^{k, *}$ is nilpotent. Let
$\omega \in \Omega^{k - 1, *}(\mathfrak{U})$, then
\begin{align*}
	(\delta^{k, *} \delta^{k - 1, *} \omega)_{\alpha_0 \dots \alpha_{k + 1}}
		&= \sum_{i = 0}^k (-1)^i (\delta^{k, *} \omega)_{\alpha_0 \dots \hat{\alpha}_i \dots \alpha_{k + 1}} \\
		&= \sum_{i = 0}^k
			\Bigg( 
				\sum_{j = 0}^{i - 1}
					(-1)^{i + j} \omega_{\alpha_0 \dots \hat{\alpha}_i \dots \hat{\alpha}_j \dots \alpha_{k + 1}} \\
		& \phantom{= \sum_{}^{} } \, {} +   \sum_{j = i + 1}^{k + 1}
					(-1)^{i + j - 1} \omega_{\alpha_0 \dots \hat{\alpha}_i \dots \hat{\alpha}_j \dots \alpha_{k + 1}}
			\Bigg)
\end{align*}
Every term appears once in the upper sum and once in the lower, with opposing signs, thus
$\delta^{k + 1, *} \circ \delta^{k, *} = 0$ for every $k$.

It remains to show that $\ker(\delta^{k + 1, *}) \subseteq \Img(\delta^{k, *})$, which we are going to do by
constructing an explicit right inverse for the forms in the kernel. Let $\{\rho_\alpha\}_{\alpha \in I}$ be
a partition of unity subordinate to $\mathfrak{U}$ and define $r: \ker(\delta^{k + 1, *}) \to \Omega^{k, *}$
component-wise:
\[
	r(\omega)_{\alpha_0 \dots \alpha_{k}}
		= \sum_{\beta \in I} \rho_\beta \omega_{\beta \omega_0 \dots \alpha_{k}}
\]
(Note that this is an infinite sum, but since the partition of unity is locally finite everything still makes
sense). This definition obviously matches our index convention, we just have to show that this is indeed a
right inverse on the kernel, so let $\omega \in \ker(\delta^{k, *})$:
\begin{align*}
	(\delta^{k - 1, *} r(\omega))_{\alpha_0 \dots \alpha_k}
		&= \sum_{i = 0}^k (-1)^i r(\omega)_{\alpha_0 \dots \hat{\alpha}_i \dots \alpha_k} \\
		&= \sum_{i = 0}^k (-1)^i \sum_{\beta \in I} \rho_\beta \omega_{\beta \omega_0 \dots \hat{\alpha}_i \dots \alpha_k} \\
		&= \sum_{\beta \in I} \rho_\beta \sum_{i = 0}^k (-1)^i \omega_{\beta \omega_0 \dots \hat{\alpha}_i \dots \alpha_k}
\end{align*}
Since $\omega \in \ker(\delta^{k, *})$ we have:
\begin{align*}
	&&(\delta^{k, *}(\omega))_{\beta \alpha_0 \dots \alpha_k}
		&= \omega_{\alpha_0 \dots \alpha_k} - \sum_{i = 0}^k (-1)^i \omega_{\beta \alpha_0 \dots \hat{\alpha}_i \dots \alpha_k}
		 = 0 \\
	\iff && \omega_{\alpha_0 \dots \alpha_k }&= \sum_{i = 0}^k (-1)^i \omega_{\beta \alpha_0 \dots \hat{\alpha}_i \dots \alpha_k} \\
\end{align*}
Therefore
\begin{align*}
	(\delta^{k - 1, *} r(\omega))_{\alpha_0 \dots \alpha_k}
		&= \sum_{\beta \in I} \rho_\beta \sum_{i = 0}^k (-1)^i \omega_{\beta \omega_0 \dots \hat{\alpha}_i \dots \alpha_k} \\
		&= \sum_{\beta \in I} \rho_\beta \omega_{\alpha_0 \dots \alpha_k}
		 = \omega_{\alpha_0 \dots \alpha_k}
\end{align*}
Hence the sequence is exact.
\end{proof}
Making use of this exactness requires still a bit of work, starting with giving $\Omega^{k, *}(\mathfrak{U})$
the structure of a cochain complex. We define its differential as applying the exterior derivative to all its
components and denote it, with a bit of abuse of notation, also as $\rmd$. This makes the question arise
whether the $\delta^{k, *}$ maps are also cochain maps:
\begin{align*}
	(\rmd \delta^{k, *} (\omega))_{\alpha_0 \dots \alpha_{k + 1}}
		&= \rmd \delta^{k, *}(\omega)_{\alpha_0 \dots \alpha_{k + 1}} \\
		&= \rmd \left( \sum_{i = 0}^{k + 1} (-1)^i \omega_{\alpha_0 \dots \hat{\alpha}_i \dots \alpha_{k + 1}} \right) \\
		&= \sum_{i = 0}^{k + 1} (-1)^i \rmd \omega_{\alpha_0 \dots \hat{\alpha}_i \dots \alpha_{k + 1}} \\
		&= \sum_{i = 0}^{k + 1} (-1)^i (\rmd \omega)_{\alpha_0 \dots \hat{\alpha}_i \dots \alpha_{k + 1}} \\
		&= \delta^{k, *}(\rmd \omega)_{\alpha_0 \dots \alpha_{k + 1}}
\end{align*}
They are! All of this information can be put together in one big commutative diagram:
% https://q.uiver.app/#q=WzAsMjIsWzIsMywiXFxPbWVnYV57MCwgMH0oXFxtYXRoZnJha3tVfSkiXSxbMywzLCJcXE9tZWdhXnsxLCAwfShcXG1hdGhmcmFre1V9KSJdLFs0LDMsIlxcT21lZ2FeezIsMH0oXFxtYXRoZnJha3tVfSkiXSxbMiwyLCJcXE9tZWdhXnswLCAxfShcXG1hdGhmcmFre1V9KSJdLFszLDIsIlxcT21lZ2FeezEsIDF9KFxcbWF0aGZyYWt7VX0pIl0sWzIsMSwiXFxPbWVnYV57MCwgMn0oXFxtYXRoZnJha3tVfSkiXSxbMywxLCJcXE9tZWdhXnsxLDJ9KFxcbWF0aGZyYWt7VX0pIl0sWzQsMSwiXFxPbWVnYV57MiwyfShcXG1hdGhmcmFre1V9KSJdLFsyLDAsIlxcdmRvdHMiXSxbMywwLCJcXHZkb3RzIl0sWzEsMSwiXFxPbWVnYV4yKE0pIl0sWzEsMiwiXFxPbWVnYV4xKE0pIl0sWzEsMywiXFxPbWVnYV4wKE0pIl0sWzEsMCwiXFx2ZG90cyJdLFswLDEsIjAiXSxbMCwyLCIwIl0sWzAsMywiMCJdLFs1LDMsIlxcY2RvdHMiXSxbNSwyLCJcXGRvdHMiXSxbNCwyLCJcXE9tZWdhXnsyLCAxfShcXG1hdGhmcmFre1V9KSJdLFs1LDEsIlxcZG90cyJdLFs0LDAsIlxcdmRvdHMiXSxbMCwxLCJcXGRlbHRhXnswLDB9Il0sWzEsMiwiXFxkZWx0YV57MSwgMH0iXSxbMCwzLCJcXG1hdGhybXtkfSIsMV0sWzMsNCwiXFxkZWx0YV57MCwxfSJdLFsxLDQsIlxcbWF0aHJte2R9IiwxXSxbMyw1LCJcXG1hdGhybXtkfSIsMV0sWzUsNiwiXFxkZWx0YV57MCwgMn0iXSxbNCw2LCJcXG1hdGhybXtkfSIsMV0sWzYsNywiXFxkZWx0YV57MSwgMn0iXSxbNSw4XSxbNiw5XSxbMTAsNSwiaV4yIl0sWzExLDMsImleMSJdLFsxMiwwLCJpXjAiXSxbMTIsMTEsIlxcbWF0aHJte2R9IiwxXSxbMTEsMTAsIlxcbWF0aHJte2R9IiwxXSxbMTAsMTNdLFsxNCwxMF0sWzE1LDExXSxbMTYsMTJdLFsyLDE3XSxbNCwxOSwiXFxkZWx0YV57MSwgMX0iXSxbMiwxOSwiXFxtYXRocm17ZH0iLDFdLFsxOSwxOF0sWzE5LDcsIlxcbWF0aHJte2R9IiwxXSxbNywyMF0sWzcsMjFdXQ==
\[\begin{tikzcd}
	& \vdots & \vdots & \vdots & \vdots \\
	0 & {\Omega^2(M)} & {\Omega^{0, 2}(\mathfrak{U})} & {\Omega^{1,2}(\mathfrak{U})} & {\Omega^{2,2}(\mathfrak{U})} & \dots \\
	0 & {\Omega^1(M)} & {\Omega^{0, 1}(\mathfrak{U})} & {\Omega^{1, 1}(\mathfrak{U})} & {\Omega^{2, 1}(\mathfrak{U})} & \dots \\
	0 & {\Omega^0(M)} & {\Omega^{0, 0}(\mathfrak{U})} & {\Omega^{1, 0}(\mathfrak{U})} & {\Omega^{2,0}(\mathfrak{U})} & \cdots
	\arrow[from=2-1, to=2-2]
	\arrow[from=2-2, to=1-2]
	\arrow["{i^2}", from=2-2, to=2-3]
	\arrow[from=2-3, to=1-3]
	\arrow["{\delta^{0, 2}}", from=2-3, to=2-4]
	\arrow[from=2-4, to=1-4]
	\arrow["{\delta^{1, 2}}", from=2-4, to=2-5]
	\arrow[from=2-5, to=1-5]
	\arrow[from=2-5, to=2-6]
	\arrow[from=3-1, to=3-2]
	\arrow["{\mathrm{d}}"{description}, from=3-2, to=2-2]
	\arrow["{i^1}", from=3-2, to=3-3]
	\arrow["{\mathrm{d}}"{description}, from=3-3, to=2-3]
	\arrow["{\delta^{0,1}}", from=3-3, to=3-4]
	\arrow["{\mathrm{d}}"{description}, from=3-4, to=2-4]
	\arrow["{\delta^{1, 1}}", from=3-4, to=3-5]
	\arrow["{\mathrm{d}}"{description}, from=3-5, to=2-5]
	\arrow[from=3-5, to=3-6]
	\arrow[from=4-1, to=4-2]
	\arrow["{\mathrm{d}}"{description}, from=4-2, to=3-2]
	\arrow["{i^0}", from=4-2, to=4-3]
	\arrow["{\mathrm{d}}"{description}, from=4-3, to=3-3]
	\arrow["{\delta^{0,0}}", from=4-3, to=4-4]
	\arrow["{\mathrm{d}}"{description}, from=4-4, to=3-4]
	\arrow["{\delta^{1, 0}}", from=4-4, to=4-5]
	\arrow["{\mathrm{d}}"{description}, from=4-5, to=3-5]
	\arrow[from=4-5, to=4-6]
\end{tikzcd}\]
Not only does this diagram commute, all its rows are exact as proven above and our assumption that
$\mathfrak{U}$ is a good cover ensures the exactness of all columns, but the first. These exactness
properties also give us a perfect candidate to extend this diagram downwards: The kernel of $\rmd$, which we
already identified as the Čech cocomplex! Doing this produces the following diagram, called the
\textbf{Čech-de Rham double complex}:
% https://q.uiver.app/#q=WzAsMzAsWzIsMywiXFxPbWVnYV57MCwgMH0oXFxtYXRoZnJha3tVfSkiXSxbMywzLCJcXE9tZWdhXnsxLCAwfShcXG1hdGhmcmFre1V9KSJdLFs0LDMsIlxcT21lZ2FeezIsMH0oXFxtYXRoZnJha3tVfSkiXSxbMiwyLCJcXE9tZWdhXnswLCAxfShcXG1hdGhmcmFre1V9KSJdLFszLDIsIlxcT21lZ2FeezEsIDF9KFxcbWF0aGZyYWt7VX0pIl0sWzIsMSwiXFxPbWVnYV57MCwgMn0oXFxtYXRoZnJha3tVfSkiXSxbMywxLCJcXE9tZWdhXnsxLDJ9KFxcbWF0aGZyYWt7VX0pIl0sWzQsMSwiXFxPbWVnYV57MiwyfShcXG1hdGhmcmFre1V9KSJdLFsyLDAsIlxcdmRvdHMiXSxbMywwLCJcXHZkb3RzIl0sWzEsMSwiXFxPbWVnYV4yKE0pIl0sWzEsMiwiXFxPbWVnYV4xKE0pIl0sWzEsMywiXFxPbWVnYV4wKE0pIl0sWzEsMCwiXFx2ZG90cyJdLFswLDEsIjAiXSxbMCwyLCIwIl0sWzAsMywiMCJdLFs1LDMsIlxcY2RvdHMiXSxbNSwyLCJcXGRvdHMiXSxbNCwyLCJcXE9tZWdhXnsyLCAxfShcXG1hdGhmcmFre1V9KSJdLFs1LDEsIlxcZG90cyJdLFs0LDAsIlxcdmRvdHMiXSxbMSw0XSxbMiw0LCJcXGhhdHtDfV4wKFxcbWF0aGZyYWt7VX0pIl0sWzMsNCwiXFxoYXR7Q31eMShcXG1hdGhmcmFre1V9KSJdLFs0LDQsIlxcaGF0e0N9XjIoXFxtYXRoZnJha3tVfSkiXSxbNSw0LCJcXGRvdHMiXSxbMiw1LCIwIl0sWzMsNSwiMCJdLFs0LDUsIjAiXSxbMCwxLCJcXGRlbHRhXnswLDB9Il0sWzEsMiwiXFxkZWx0YV57MSwgMH0iXSxbMCwzLCJcXG1hdGhybXtkfSIsMV0sWzMsNCwiXFxkZWx0YV57MCwxfSJdLFsxLDQsIlxcbWF0aHJte2R9IiwxXSxbMyw1LCJcXG1hdGhybXtkfSIsMV0sWzUsNiwiXFxkZWx0YV57MCwgMn0iXSxbNCw2LCJcXG1hdGhybXtkfSIsMV0sWzYsNywiXFxkZWx0YV57MSwgMn0iXSxbNSw4XSxbNiw5XSxbMTAsNSwiaV4yIl0sWzExLDMsImleMSJdLFsxMiwwLCJpXjAiXSxbMTIsMTEsIlxcbWF0aHJte2R9IiwxXSxbMTEsMTAsIlxcbWF0aHJte2R9IiwxXSxbMTAsMTNdLFsxNCwxMF0sWzE1LDExXSxbMTYsMTJdLFsyLDE3XSxbNCwxOSwiXFxkZWx0YV57MSwgMX0iXSxbMiwxOSwiXFxtYXRocm17ZH0iLDFdLFsxOSwxOF0sWzE5LDcsIlxcbWF0aHJte2R9IiwxXSxbNywyMF0sWzcsMjFdLFsyMywwLCJcXG1hdGhybXtkfSIsMV0sWzIzLDI0LCJcXGRlbHRhIl0sWzI0LDEsIlxcbWF0aHJte2R9IiwxXSxbMjQsMjUsIlxcZGVsdGEiXSxbMjUsMiwiXFxtYXRocm17ZH0iLDFdLFsyNSwyNl0sWzI3LDIzXSxbMjgsMjRdLFsyOSwyNV1d
\[\begin{tikzcd}
	& \vdots & \vdots & \vdots & \vdots \\
	0 & {\Omega^2(M)} & {\Omega^{0, 2}(\mathfrak{U})} & {\Omega^{1,2}(\mathfrak{U})} & {\Omega^{2,2}(\mathfrak{U})} & \dots \\
	0 & {\Omega^1(M)} & {\Omega^{0, 1}(\mathfrak{U})} & {\Omega^{1, 1}(\mathfrak{U})} & {\Omega^{2, 1}(\mathfrak{U})} & \dots \\
	0 & {\Omega^0(M)} & {\Omega^{0, 0}(\mathfrak{U})} & {\Omega^{1, 0}(\mathfrak{U})} & {\Omega^{2,0}(\mathfrak{U})} & \cdots \\
	& {} & {\hat{C}^0(\mathfrak{U})} & {\hat{C}^1(\mathfrak{U})} & {\hat{C}^2(\mathfrak{U})} & \dots \\
	&& 0 & 0 & 0
	\arrow[from=2-1, to=2-2]
	\arrow[from=2-2, to=1-2]
	\arrow["{i^2}", from=2-2, to=2-3]
	\arrow[from=2-3, to=1-3]
	\arrow["{\delta^{0, 2}}", from=2-3, to=2-4]
	\arrow[from=2-4, to=1-4]
	\arrow["{\delta^{1, 2}}", from=2-4, to=2-5]
	\arrow[from=2-5, to=1-5]
	\arrow[from=2-5, to=2-6]
	\arrow[from=3-1, to=3-2]
	\arrow["{\mathrm{d}}"{description}, from=3-2, to=2-2]
	\arrow["{i^1}", from=3-2, to=3-3]
	\arrow["{\mathrm{d}}"{description}, from=3-3, to=2-3]
	\arrow["{\delta^{0,1}}", from=3-3, to=3-4]
	\arrow["{\mathrm{d}}"{description}, from=3-4, to=2-4]
	\arrow["{\delta^{1, 1}}", from=3-4, to=3-5]
	\arrow["{\mathrm{d}}"{description}, from=3-5, to=2-5]
	\arrow[from=3-5, to=3-6]
	\arrow[from=4-1, to=4-2]
	\arrow["{\mathrm{d}}"{description}, from=4-2, to=3-2]
	\arrow["{i^0}", from=4-2, to=4-3]
	\arrow["{\mathrm{d}}"{description}, from=4-3, to=3-3]
	\arrow["{\delta^{0,0}}", from=4-3, to=4-4]
	\arrow["{\mathrm{d}}"{description}, from=4-4, to=3-4]
	\arrow["{\delta^{1, 0}}", from=4-4, to=4-5]
	\arrow["{\mathrm{d}}"{description}, from=4-5, to=3-5]
	\arrow[from=4-5, to=4-6]
	\arrow["{\mathrm{d}}"{description}, from=5-3, to=4-3]
	\arrow["\delta", from=5-3, to=5-4]
	\arrow["{\mathrm{d}}"{description}, from=5-4, to=4-4]
	\arrow["\delta", from=5-4, to=5-5]
	\arrow["{\mathrm{d}}"{description}, from=5-5, to=4-5]
	\arrow[from=5-5, to=5-6]
	\arrow[from=6-3, to=5-3]
	\arrow[from=6-4, to=5-4]
	\arrow[from=6-5, to=5-5]
\end{tikzcd}\]
We also included the Čech differential $\delta$ in this diagram, which we want to commute with $\rmd$:
\[
	\text{TODO}	
\]
Having united the Čech cocomplex with the de Rham cocomplex in a single commutative diagram, all that's left is
using this diagram to prove the existence of the actual isomorphism. As with the Mayer-Vietoris sequence, this
is best done purely algebraicly and requires a bit of a detour into homological algebra.
\section{Double Complexes}
The diagram we constructed in the last paragraph is an instance of a more abstract phenomenom:
\begin{definition}[Double Complex]
	A \underline{double complex} is a $\ZZ \times \ZZ$-graded vector space
	$C^{*, *} := \{C^{p,q}\}_{p,q \in \ZZ}$ together with linear
	maps
	\[
		d^{p,q}_h: C^{p, q} \to C^{p + 1, q}
		\quad
		\text{ and }
		\quad
		d^{p,q}_v: C^{p,q} \to C^{p, q + 1}
	\]
	for every $p,q \in \ZZ$ (with $d^{*, *}_h$ as usual referring
	to the unique linear map that restricts to $d^{p,q}$ on $C^{p,q}$ and
	similiar for $d^{*, *}_v$),  with $\dvb, \dhb$ each being
	nilpotent and commutative with each other, i.e:
	\[
		\dvb \circ \dvb = \dhb \circ \dhb = 0
	\]
	And the following commutes for every $p,q$:
	% https://q.uiver.app/#q=WzAsNCxbMCwwLCJDXntwLCBxKzF9Il0sWzAsMSwiQ157cCxxfSJdLFsxLDEsIkNee3AgKyAxLCBxfSJdLFsxLDAsIkNee3ArMSxxKzF9Il0sWzEsMCwiZF92XntwLHF9Il0sWzEsMiwiZF57cCxxfV9oIiwyXSxbMCwzLCJkXntwLHErMX1faCJdLFsyLDMsImRee3ArMSxxfV92IiwyXV0=
	\[
		\begin{tikzcd}
			{C^{p, q+1}} & {C^{p+1,q+1}} \\
			{C^{p,q}} & {C^{p + 1, q}}
			\arrow["{d^{p,q+1}_h}", from=1-1, to=1-2]
			\arrow["{d_v^{p,q}}", from=2-1, to=1-1]
			\arrow["{d^{p,q}_h}"', from=2-1, to=2-2]
			\arrow["{d^{p+1,q}_v}"', from=2-2, to=1-2]
		\end{tikzcd}\]
\end{definition}
\begin{remark}
	Some authors (e.g. \cite{weibel_introduction_1994}) prefer these maps to
	be anticommutative instead of commutative. While this saves some signs
	in some places (e.g. in the definition of homology of a chain complex),
	it also adds them in other places and since the later dominates in this
	thesis, we are going to follow \cite{tu_differential_1982} and assume
	commutativity.
\end{remark}

Double complexes are a powerfull tool and not only can they be seen as a generalization of
chain (co)complexes\footnote{
	The high-brow approach to this would be to define chain complexes over general abelian categories,
	then a double complex is "just" a chain complex of chain complexes, but let's not do that.
}, they also can be reduced back to a chain cocomplex by summing up their diagonals:
\begin{definition}[Associated Chain Cocomplex]
	Given a double complex $C^{*, *}$ with maps $\dvb, \dhb$ the
	\textbf{
		associated chain cocomplex of $C^{*, *}$
		}, denoted as
	$C^*(C^{*, *})$ is the chain complex defined as:
	\[
		C^n(C^{*, *}) \coloneq \bigoplus_{n = p + q} C^{p,q}
	\]
	with the differential $d^n: C^n \to C^{n + 1}$ being defined as:
	\[
		d^n \coloneq \sum_{p + q = n} d^{p,q}_v + (-1)^p d^{p,q}_h
	\]
\end{definition}
This actually being a chain cocomplex remains to be proven:
\begin{lemma}
	For a given double complex $C^{*, *}$ with maps $\dvb$ and $\dhb$
	the associated chain complex is a chain complex.
\end{lemma}
\begin{proof}
	We have to proof that $d^*$ is nilpotent. Let $c \in C^{p,q}$ for some $p,q \in \ZZ$.
	Then:
	\begin{align*}
		(d^{p + q + 1} \circ d^{p + q})(c)
		&=
		d^{p + q + 1}(d^{p,q}_v(c) + (-1)^p d^{p,q}_h(c)) \\
		&=
		\overbrace{
			(d^{p + 1, q}_v \circ d^{p,q}_v)(c)
		}^{\text{nilpotent}}
		+
		\overbrace{
			(d^{p, q + 1}_v \circ d^{p,q}_v)(c)
		}^{\text{nilpotent}} \\
		&\quad+ (-1)^p (\underbrace{
				(d^{p + 1, q}_v \circ d^{p,q}_h)(c)
				- (d^{p, q + 1}_h \circ d^{p,q}_v)(c)
			)
		}_{\text{commutative}}) \\
		&= 0 \qedhere
	\end{align*}
\end{proof}
\begin{remark}
	By abuse of notation we will sometimes call $C^{*, *}$
	a chain complex, this always refers to $C^*(C^{*,*})$.
\end{remark}
But the name of the game is homological algebra, not chain complex algebra, so the logical
next step is taking the cohomology of this:
\begin{definition}[Cohomology of a Double Complex]
	For a given double complex $C^{*,*}$ we define its
	\underline{cohomology} as the cohomology of the
	associated chain cocomplex, in notation:
	\[
		H^*(C^{*,*}) := H^*(C^*(C^{*,*}))
	\]
\end{definition}

Double complexes form a fascinating part of commutative algebra, being used to defined such marvelous (and
mysterious) tools like spectral sequences and one could study them on their own, but we are
only interested in using them to proof our statement and are therefore going to add some
additional assumptions about them. Particularly we are going to assume that all future double
complexes are non-trivial only in the upper-right quadrant i.e. $C^{p,k} = 0$ for $p < 0$ or
$k < 0$.

The Čech-de Rham double complex does not really fit this assumption, having an additional row
and column. We are going to treat these not as part of the double complex, but as an
augmentation of it and then proof de Rhams theorem by proving that the cohomology of the
double complex is isomorphic to the cohomologies of these augmentations. This is not true for
all double complexes and augmentations, but with some convenient assumptions it is true:
\begin{theorem}
\label{theorem:augmentation_iso_homology}
Let $C^{*,*}$ be a double complex, $(A^*, \delta)$ a chain
complex.

If for every $n \geq 0$ there is a map
\[
	\alpha^n: A^n \to C^{n,0}
\]
such that the following diagram commutes:
\begin{center}
% https://tikzcd.yichuanshen.de/#N4Igdg9gJgpgziAXAbVABwnAlgFyxMJZAJgBoBmAXVJADcBDAGwFcYkQBhAPWAAZTeAXxCDS6TLnyEU5CtTpNW7bsACMA4aPHY8BIgBY5NBizaJOPMkJFiQGHVKIBWIwtPsAOh6gQcCLXYSutIkpMTyJkrmKvyqmrb2knooZKoRimYWfGHx2kkhsmnGGco86sS5gQ7JyLLhxe7RZaRxNnnBBmHpjVmplYkdKIZFblG9OW1V+URW3WNetD5+kwOOMgJzmQtL-glBa8iGvJueHou+u+0HLvWjW94XK-s1LiOR9ztP1SHqbyXmAEEuMQvtMUOUToCuKpQYNkOoqA0xkDeLCDupjkj7udlgFVjVYpCQKi8c8Qvxbu92CS9t8iPxEXdqSJ5DAoABzeBEUAAMwAThAALZIfggHAQJDqJnmKA8fhCAD6AAtJvyhZKaOKkGRpSBZdk4srVQLhYgdVrEOQAmrTaKLYZdfr5YIFbRjerEA6LS5HXKWi63daTUgfRaAGxY9hOnJGoMer0SxAAdkjMr9hpVcdNKbFiYjvrU-td7tN+YtAE5U3rLEXM7YbUgcxaABxV-W-F113nBxCt3NISsFvrFrMa-vJtvNRWB+s982JvtUtPAKwBkshzWJ1RCWcestbuK702Di2qCpHpB909Wi+IE9b-SjxCqJtb-NLkBeWCMHD0Lg07sPVUfdJR9D8vxgH8-xhJ9VFDLdm1g+9QNg+CNzFegsEYdglQgCAAGsQCrLwmDQJU-xBJ9iDtN9YJAz1NUw7DzFwgiiN1EjGDI6D1zNKVTyTWDXxFRisJwvDCOIjxSPI-8WUEIA
\begin{tikzcd}
            & \vdots                                                & \vdots                                                      & \vdots                                                      & \vdots                                       &       \\
0 \arrow[r] & A^2 \arrow[u] \arrow[r, "\alpha^2", hook]             & {C^{0,2}} \arrow[r, "{d^{0,2}_h}"] \arrow[u]                & {C^{1,2}} \arrow[r, "{d^{2,1}_h}"] \arrow[u]                & {C^{2,2}} \arrow[u] \arrow[r]                & \dots \\
0 \arrow[r] & A^1 \arrow[u, "\delta^1"] \arrow[r, "\alpha^1", hook] & {C^{0,1}} \arrow[u, "{d^{0,1}_v}"] \arrow[r, "{d^{0,1}_h}"] & {C^{1,1}} \arrow[u, "{d^{1,1}_v}"] \arrow[r, "{d^{1,1}_h}"] & {C^{2,1}} \arrow[u, "{d^{2,1}_v}"] \arrow[r] & \dots \\
0 \arrow[r] & A^0 \arrow[u, "\delta^0"] \arrow[r, "\alpha^0", hook] & {C^{0,0}} \arrow[r, "{d^{0,0}_h}"] \arrow[u, "{d^{0,0}_v}"] & {C^{1,0}} \arrow[r, "{d^{0,1}_h}"] \arrow[u, "{d^{1,0}_v}"] & {C^{2,0}} \arrow[r] \arrow[u, "{d^{2,0}_v}"] & \dots
\end{tikzcd}
\end{center}
and every row is exact, then
\[
	H^*(A^*) \cong H^*(C^{*,*})
\]
\end{theorem}
Before proving this we are going to need a convenient lemma:
\begin{lemma}
\label{lemma:helpfull}
Given a double complex $C^{*, *}$ for which the sequence
\begin{center}
% https://tikzcd.yichuanshen.de/#N4Igdg9gJgpgziAXAbVABwnAlgFyxMJZABgBpiBdUkANwEMAbAVxiRAGEA9YMtAXxB9S6TLnyEUARnJVajFmy7Bp-QcJAZseAkQBMM6vWatEHbvtVCRW8UQDMBuccXcHl9ZrE6UAFkdGFUwAdIKgIHAQ+WRgoAHN4IlAAMwAnCABbJDIQHAgkSSsQVIz86lykXULizMR9HLzEOyq0mod6pB8oviA
\begin{tikzcd}
{C^{0,p}} \arrow[r] & {C^{1,p}} \arrow[r] & {C^{2,p}} \arrow[r] & {C^{3,p}} \arrow[r] & \dots
\end{tikzcd}
\end{center}
is exact for every $p$, then every homology class $[c] \in H^n(C^{*, *})$ is homologous to one
represented by a $\tilde{c} \in C^{0,n}$ i.e. $[c] = [\tilde{c}]$.
\end{lemma}
\begin{proof}
Let $[c] \in H^n(C^{*, *})$. One can write:
\[
	c = c^{0, n} + c^{1, n - 1} + \dots + c^{k, \ell}
\]
with $c^{i,j} \in C^{i,j}$.

We just have to prove that $c^{k, \ell}$ is homologous to an element in $C^{k - 1, \ell + 1}$,
the rest follows by induction.

It follows from $c \in \ker(d^n)$ that $\rmd_h^{k, \ell} c^{k, \ell} = 0$ and hence by exactness
of the rows there is a $c^{k - 1, \ell}$ such that:
\[
	\rmd_h^{k - 1, \ell} c^{k - 1, \ell} = c^{k, \ell}
\]
Which implies:
\begin{align*}
	&    &\rmd^{n - 1} c^{k - 1, \ell} = (-1)^{k - 1} \rmd_v^{k - 1, \ell} c^{k - 1, \ell} + c^{k, \ell}\\
	&\iff&  (-1)^{k - 1} \rmd_v^{k - 1, \ell} c^{k - 1, \ell} = c^{k, \ell} - \rmd^{n - 1} c^{k - 1, \ell} \\
	&\implies& [\underbrace{(-1)^{k - 1} \rmd_v^{k - 1, \ell} c^{k - 1, \ell}}_{\in C^{k - 1, \ell + 1}}] = [c^{k, \ell}]
\end{align*}
Thus by induction $[c] = [\tilde{c}]$ for some $\tilde{c} \in C^{0, n}$.
\end{proof}
This makes the proof of the original statement way easier:
\begin{proof}[Proof of \ref{theorem:augmentation_iso_homology}]
We start by proving that $\alpha^*$ is a cochain map between $A^*$ and $C^{*, *}$. Let $a \in
A^\ell$. Then we have:
\begin{align*}
	(\rmd^{\ell + 1} \circ \alpha^\ell)(a)
		&= \rmd_v \alpha^\ell a
			+ \underbrace{d_h \alpha^\ell a}_{\mathclap{=0 \text{ by exactness}}}
		= \rmd_v \alpha^\ell a \\
	(\alpha^{\ell + 1} \circ \delta^\ell)(a) &= \rmd_v \alpha^\ell a
\end{align*}
Hence $\alpha^*$ is a cochain map and thus descends to cohomology, it remains to show that
this descension $\tilde{\alpha}^*$ is an isomorphism.

We start by showing that $\tilde{\alpha}^n$ is injective. Let $[a] \in H^n(A^*)$ such that
$\tilde{\alpha}^n([a]) = [0]$. By definition there then exists a $c \in C^{n - 1, 0}$
such that $\rmd c = \alpha^n(a)$ and therefore in particular $\rmd_h c = 0$. By exactness
there is then a $\check{a} \in A^{n - 1}$ such that $\alpha^{n - 1}(\check{a}) = c$, or
putting all of this into a commutative diagram:
% https://q.uiver.app/#q=WzAsNSxbMCwwLCJhIl0sWzEsMCwiXFxhbHBoYV5uKGEpIl0sWzEsMSwiYyJdLFsyLDEsIjAiXSxbMCwxLCJcXGNoZWNre2F9Il0sWzAsMSwiXFxhbHBoYV5uIl0sWzIsMSwiXFxtYXRocm17ZH1fdiIsMl0sWzIsMywiXFxtYXRocm17ZH1faCJdLFs0LDJdXQ==
\[\begin{tikzcd}
	a & {\alpha^n(a)} \\
	{\check{a}} & c & 0
	\arrow["{\alpha^n}", from=1-1, to=1-2]
	\arrow[from=2-1, to=2-2]
	\arrow["{\mathrm{d}_v}"', from=2-2, to=1-2]
	\arrow["{\mathrm{d}_h}", from=2-2, to=2-3]
\end{tikzcd}\]
And since everything commutes $\alpha^n(\delta^{n - 1}(\check{a})) = \alpha^n(a)$, which by
injectivity of $\alpha$ implies $\delta^{n - 1}(\check{a}) = a$ and in particular $[a] = [0]$.
Therefore $\tilde{\alpha}^*$ is injective.

To prove surjectivity of $\alpha^*$, let $[c] \in H^n(C^{*,*})$. With the help of lemma \ref{lemma:helpfull}. 
we can assume w.l.o.g. that $c \in C^{0,n}$. By definition of homology $c \in \ker(d^n)$ and therefore
$d_v(c) = d_h(c) = 0$. By exactness there then exists $a \in A^n$ such that $\alpha^n(a) = c$. Putting this
in a commutative diagram:
% https://q.uiver.app/#q=WzAsNSxbMSwwLCIwIl0sWzEsMSwiYyJdLFsyLDEsIjAiXSxbMCwxLCJhIl0sWzAsMCwiXFxkZWx0YShhKSJdLFsxLDAsIlxcbWF0aHJte2R9X3YiLDJdLFsxLDIsIlxcbWF0aHJte2R9X2giLDJdLFszLDEsIlxcYWxwaGFebiIsMl0sWzMsNF0sWzQsMCwiXFxhbHBoYV57biArIDF9Il1d
\[\begin{tikzcd}
	{\delta(a)} & 0 \\
	a & c & 0
	\arrow["{\alpha^{n + 1}}", from=1-1, to=1-2]
	\arrow[from=2-1, to=1-1]
	\arrow["{\alpha^n}"', from=2-1, to=2-2]
	\arrow["{\mathrm{d}_v}"', from=2-2, to=1-2]
	\arrow["{\mathrm{d}_h}"', from=2-2, to=2-3]
\end{tikzcd}\]
By injectivity of $\alpha^*$ then also $\delta(a) = 0$ and in particular $[a] \in H^n(A^*)$,
which fulfills:
\[
	\alpha^*([a]) = [\alpha^*(a)] = [c]
\]
Since $[c]$ was chosen arbitarily $\tilde{\alpha}^*$ is therefore also surjective.
\end{proof}
By symmetry, everything we just did can also be done for double complexes
with exact columns, that are augmented at the bottom. Because of the 
definition of the cohomology of double complexes this requires some
additional sign trickery, but this does not change anything substantial
about the proof itself. In particular, if both rows and columns of a double complex
are exact, then the cohomologies of augmentations at the bottom and on the left
are both isomorphic to the cohomology of the double complex itself, resulting in
our final statement:
\begin{corolarry}
Given a double complex $C^{*,*}$ with exact rows and columns and two cochain complexes $A^*,
B^*$ together with maps $\alpha^*, \beta^*$ such that
\begin{center}
% https://tikzcd.yichuanshen.de/#N4Igdg9gJgpgziAXAbVABwnAlgFyxMJZAJgBoBGAXVJADcBDAGwFcYkQBhAPWAAZTiAXxCDS6TLnyEUAZgrU6TVu27BypNMNHjseAkQAs8mgxZtEnHmSEixIDLqlFrC08our+5LXYeT9KGQyrkrmlnykvD46-tLIcsQhZio86t62MXpxcsEmoSlqkdH2ElmGAknu4WTp2iWOAchGuYrJHlZFGfWxRACsxq1VADpDUBA4CHV+ZSj9iXltICNjE13TTrOkLW5hy+OTvqUbTZGVu0O0KweZx3K8Z+wjl-trR41k9wvDF1evDXFpB4WACCXGIfx6KHU80GYVB5AhM2Q6k+sMePxeUzeAK2QJAoN4iOOZAMeIAQlxCVj-kQ5KSvmEKQjqZCTvS0RYKeCWUj+uydujfjzjl48VTDjSUPwYQKLOKbo1+Nt8nKie9SL0xWrshqtcLGkZNQz2FSFDAoABzeBEUAAMwAThAALZIfggHAQJDeOwO51emgepBCH2Ol2IOTuz2IKIhv2IIyRpAyOq+sMRwOIXop0NIBMZgBs2bj+YDUazsbD-UTiG9dpziAA7KWg0WwwAOZuN1tIEvVtvdzOdhsDsjVgCcA6b1fIMbrcY709rIFTQc75GDc7D6mnyYrrrXBgH5CrGbdsqWQyYaAAFvQwSAA-QsIx2NeIBAANZdFc13sZ49HieUbkMOe41gu-6FmB5AQVGCbniMV63pSD7uk+L4WG+n7fvW5B-lGEYIZejA3neCKPs+r7vl+R5jp2xCzsu9YMfRS4-sQ24ZsQG5MXG5B0dW8EqheABGMA4HehIURhIBYTRYEsdWvZEWJElcORaGUZh1E4XGHGdguKniXe4LSVR2EjumwHlpuQZ5sBUG2YgxBAV6-YKfhXoTgpU5cYx7GwUG3iUIIQA
\begin{tikzcd}
            & \vdots                                    & \vdots                                   & \vdots                                   & \vdots                                   &       \\
0 \arrow[r] & A^2 \arrow[r, "\alpha^2", hook] \arrow[u] & {C^{0,2}} \arrow[r] \arrow[u]            & {C^{1,p}} \arrow[r] \arrow[u]            & {C^{2,2}} \arrow[r] \arrow[u]            & \dots \\
0 \arrow[r] & A^1 \arrow[u] \arrow[r, "\alpha^1", hook] & {C^{0,1}} \arrow[u] \arrow[r]            & {C^{1,1}} \arrow[u] \arrow[r]            & {C^{2,1}} \arrow[u] \arrow[r]            & \dots \\
0 \arrow[r] & A^0 \arrow[u] \arrow[r, "\alpha^0", hook] & {C^{0,0}} \arrow[u] \arrow[r]            & {C^{1,0}} \arrow[u] \arrow[r]            & {C^{2,0}} \arrow[u] \arrow[r]            & \dots \\
            &                                           & B^0 \arrow[r] \arrow[u, "\beta^0", hook] & B^1 \arrow[r] \arrow[u, "\beta^1", hook] & B^2 \arrow[r] \arrow[u, "\beta^2", hook] & \dots \\
            &                                           & 0 \arrow[u]                              & 0 \arrow[u]                              & 0 \arrow[u]                              &      
\end{tikzcd}
\end{center}
commutes, then:
\[
	H^*(A^*)
		\cong H^*(C^{*, *})	
		\cong H^*(B^*)
\]

\end{corolarry}
De Rhams theorem follows as an immediate consequence of this and we can even use the prior proofs
to describe this isomorphism quite explicitely

\section{Explicit Isomorphism \& Beyond Smoothness}
Let us take a good cover $\mathcal{U}_\alpha, \mathcal{U}_\beta, \mathcal{U}_\gamma$ of $S^1$. The nerve of
this cover is shown in TODO, together with a choice of orientation.
\tikzset{
  % style to apply some styles to each segment of a path
  on each segment/.style={
    decorate,
    decoration={
      show path construction,
      moveto code={},
      lineto code={
        \path [#1]
        (\tikzinputsegmentfirst) -- (\tikzinputsegmentlast);
      },
      curveto code={
        \path [#1] (\tikzinputsegmentfirst)
        .. controls
        (\tikzinputsegmentsupporta) and (\tikzinputsegmentsupportb)
        ..
        (\tikzinputsegmentlast);
      },
      closepath code={
        \path [#1]
        (\tikzinputsegmentfirst) -- (\tikzinputsegmentlast);
      },
    },
  },
  % style to add an arrow in the middle of a path
  mid arrow/.style={postaction={decorate,decoration={
        markings,
        mark=at position .5 with {\pgftransformscale{2}\arrow[#1]{stealth}}
      }}},
}
\begin{figure}
\begin{tikzpicture}
    \tikzstyle{point}=[thick,draw=black,inner sep=0pt]
    \node[label={[below left]$\alpha$}] (a) at (0,0) {};
    \node[label={[below right]$\beta$}] (b) at (5,0) {};
    \node[label={[above]$\gamma$}] (c) at (2.5,4.33) {};
	% yshift=-10pt
	% xshift=7pt, yshift=7pt
	% xshift=-7pt, yshift=7pt
    \draw[postaction={on each segment={mid arrow}}]
		(a.center) -- node[midway, label={[label distance=10pt]270:$c$}] {}
		(b.center) -- node[midway, label={[label distance=10pt]0:$a$}] {}
		(c.center) -- node[midway, label={[label distance=10pt]180:$b$}] {}
		cycle;
	\foreach \n in {a,b,c}
	{
		\fill (\n.center) circle (1.5pt);
	}
\end{tikzpicture}
\end{figure}
The Čech cochains of this complex are:
\begin{align*}
	C^0(K) = \langle \alpha, \beta, \gamma \rangle, \,
	C^1(K) = \langle a, b, c \rangle
\end{align*}
With the coboundary operator being defined as:
\begin{align*}
	\delta \alpha = b - c,\quad 
	\delta \beta = c - a,\quad
	\delta \gamma = a - b
\end{align*}
We already calculated the cohomology of this, its $\hat{H}^0(K) = \hat{H}^1(K) = \RR$,
with the 1-cohomology being generated by $[\alpha] = [\beta] = [\gamma]$. We can 
