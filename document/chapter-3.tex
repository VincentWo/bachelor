\chapter{The Čech-de Rham Double Complex}
\epigraph{
	Algebra is the offer made by the devil to the mathematician. The devil says: I will give you this powerful machine, it will answer any question you like. All you need to do is give me your soul: give up geometry and you will have this marvellous machine.
}{Sir Michael Atiyah}
So far Čech and de Rham Cohomology are similiar in name only: They are defined on different spaces using different 
machinery and seem to measure different things: Čech Cohomology the "holes" of a space and de Rham Cohomology the
solution spaces of differential equations modulo trivial solutions.

But at the same time there is a good reason to hope for a connection betweem them both: We already sketched in the
introduction that the zeroth and first de Rham Cohomology vanish if and only if the respective Čech Cohomologies
vanish and in all of the examples we have seen so far both cohomologies gave the same results. These examples can be
generalized into \textit{de Rhams theorem}, which proof is going to occupy the rest of this thesis:
\begin{theorem}[De Rhams theorem]
Let $M$ be a smooth manifold. There is a natural??
% TODO: Natural
isomorphism between the de Rham cohomology of $M$ and the Čech cohomology of $M$
\end{theorem}

This is not only a remarkable fact, but is also going to solve some non-trivial remaining issues
we had with both cohomologies. Our main problem with Čech-chomology was, that while the Cohomology of a given cover
is easily calculatable, its limit is not. But a corollary of our proof is going to be that the Čech cohomology of a
so called "good" cover of $M$ is naturally isomorphic to the Čech cohomology of $M$, allowing the calculation of 
de Rham and Čech cohomology based on combinatorial data.

Another question that this theorem answers is whether de Rham cohomology is a topological invariant i.e. whether
different non-diffeomorphic choices of a smooth structure on a manifold may lead to different cohomologies. But since
Čech cohomology by definition does not depend on any smooth structure, we may conclude:
\begin{corolarry}
Let $M, N$ be two homeomorphic smooth manifolds. Then the de Rham cohomology of $M$ is naturally
isomorphic to the de Rham cohomology of $N$.
\end{corolarry}

\section{Mayer Vietoris to the infinite}
To prove de Rhams theorem we are going to start with expressing the Čech Cohomology of a given cover in terms of
the de Rham cohomologies of the sets. We start by fixing some manifold $M$ and an open cover
$\mathfrak{U} = \{\U_\alpha\}_{\alpha \in I}$ (with $I$ being some totally ordered index set).

We recall that the Čech Cocomplex of $\mathfrak{U}$ was defined as the dual of the Čech Complex, which algebraicly
is just a direct sum over copies of $\RR$, with each copy corresponding to a non-empty intersection of $k$ different
sets in $\mathfrak{U}$. There is a canonical isomorphism from this to the product of the vector spaces of constant
functions on each non-empty intersection of $k$ different sets. That is, we define:
\begin{align*}
	C_c(\U) \coloneqq \{\, f: \U \to \RR\ \text{constant} \,\}
\end{align*}
Taking the usual vector space structure, there is a canonical isomorphism:
\begin{align*}
	\prod_{\alpha_0 <  \dots < \alpha_k}
	C_c(\U_{\alpha_0 \dots \alpha_k}) \cong \hat{C}^k(M, \mathfrak{U})
\end{align*}
Now if we would assume that all sets and intersections are connected, then the constant functions are exactly the locally
constant one. And the product of the vector spaces of locally constant functions on all $\U_{\alpha_0\dots\alpha_k}$ is
naturally isomorphic to the vector space of locally constant functions on the disjoint union of the spaces\footnote{
	This is easy to see, since locally constant functions are constant on connected components and the connected components
	of the disjoint union of connected spaces are the spaces itself
}.

But we already know a alternative characterisation of locally constant functions: These are exactly the 0-forms with
vanishing exterior derivative! Thus (given our connectedness assumption) there is a natural representation of the Čech
Cocomplex using the kernel of the exterior derivative of the zero forms on the disjoint union of the cover.

This information can be used to extend the Čech Cocomplex "upwards":
% https://q.uiver.app/#q=WzAsMTQsWzAsMV0sWzEsMSwiXFxPbWVnYV4wKFxccHJvZF97XFxhbHBoYSBcXGluIEl9IFxcbWF0aGNhbHtVfV9cXGFscGhhKSJdLFsyLDEsIiBcXE9tZWdhXjAoXFxwcm9kX3tcXGFscGhhLFxcYmV0YSBcXGluIEl9XFxtYXRoY2Fse1V9X3tcXGFscGhhXFxiZXRhfSkiXSxbMSwyLCJcXGhhdHtDfV4wKE0sIFxcbWF0aGZyYWt7VX0pIl0sWzIsMiwiXFxoYXR7Q31eMShNLCBcXG1hdGhmcmFre1V9KSJdLFszLDIsIlxcaGF0e0N9XjIoTSwgXFxtYXRoZnJha3tVfSkiXSxbMywxLCJcXE9tZWdhXjAoXFxwcm9kX3tcXGFscGhhLFxcYmV0YSxcXGdhbW1hIFxcaW4gSX0gXFxtYXRoY2Fse1V9X3tcXGFscGhhXFxiZXRhXFxnYW1tYX0pIl0sWzQsMiwiXFxjZG90cyJdLFsxLDAsIlxcT21lZ2FeMShcXHByb2Rfe1xcYWxwaGEgXFxpbiBJfSBcXG1hdGhjYWx7VX1fXFxhbHBoYSkiXSxbMSwzLCIwIl0sWzIsMywiMCJdLFszLDMsIjAiXSxbMiwwLCIgXFxPbWVnYV4xKFxccHJvZF97XFxhbHBoYSxcXGJldGEgXFxpbiBJfVxcbWF0aGNhbHtVfV97XFxhbHBoYVxcYmV0YX0pIl0sWzMsMCwiXFxPbWVnYV4wKFxccHJvZF97XFxhbHBoYSxcXGJldGEsXFxnYW1tYSBcXGluIEl9IFxcbWF0aGNhbHtVfV97XFxhbHBoYVxcYmV0YVxcZ2FtbWF9KSJdLFszLDEsImleMCIsMix7InN0eWxlIjp7InRhaWwiOnsibmFtZSI6Imhvb2siLCJzaWRlIjoiYm90dG9tIn19fV0sWzQsMiwiaV4xIiwyLHsic3R5bGUiOnsidGFpbCI6eyJuYW1lIjoiaG9vayIsInNpZGUiOiJib3R0b20ifX19XSxbMyw0XSxbNCw1XSxbNSw2LCJpXjIiLDIseyJzdHlsZSI6eyJ0YWlsIjp7Im5hbWUiOiJob29rIiwic2lkZSI6ImJvdHRvbSJ9fX1dLFs1LDddLFsxLDgsImReezAsMH0iLDJdLFs5LDNdLFsxMCw0XSxbMTEsNV0sWzIsMTIsImReezAsMX0iLDJdLFs2LDEzLCJkXnswLDJ9IiwyXV0=
\[\begin{tikzcd}[cramped]
	& {\Omega^1(\prod_{\alpha \in I} \mathcal{U}_\alpha)} & { \Omega^1(\prod_{\alpha,\beta \in I}\mathcal{U}_{\alpha\beta})} & {\Omega^0(\prod_{\alpha,\beta,\gamma \in I} \mathcal{U}_{\alpha\beta\gamma})} \\
	{} & {\Omega^0(\prod_{\alpha \in I} \mathcal{U}_\alpha)} & { \Omega^0(\prod_{\alpha,\beta \in I}\mathcal{U}_{\alpha\beta})} & {\Omega^0(\prod_{\alpha,\beta,\gamma \in I} \mathcal{U}_{\alpha\beta\gamma})} \\
	& {\hat{C}^0(M, \mathfrak{U})} & {\hat{C}^1(M, \mathfrak{U})} & {\hat{C}^2(M, \mathfrak{U})} & \cdots \\
	& 0 & 0 & 0
	\arrow["{d^{0,0}}"', from=2-2, to=1-2]
	\arrow["{d^{0,1}}"', from=2-3, to=1-3]
	\arrow["{d^{0,2}}"', from=2-4, to=1-4]
	\arrow["{i^0}"', hook', from=3-2, to=2-2]
	\arrow[from=3-2, to=3-3]
	\arrow["{i^1}"', hook', from=3-3, to=2-3]
	\arrow[from=3-3, to=3-4]
	\arrow["{i^2}"', hook', from=3-4, to=2-4]
	\arrow[from=3-4, to=3-5]
	\arrow[from=4-2, to=3-2]
	\arrow[from=4-3, to=3-3]
	\arrow[from=4-4, to=3-4]
\end{tikzcd}\]
And by the characterisation of the Čech Cocomplex as the kernel of the exterior derivative all the columns are exact. Tightening our assumption even further by
also requiring $H^\ell_\text{DR}(\prod_{\alpha_0 < \dots < \alpha_k})$ to be trivial for every $\ell > 0$, allows us to extend this
further upwards, producing an even larger diagram. But before we are going to show this, we are going to define some notation:
\begin{align*}
	C^{k,\ell} \coloneqq \Omega^k(\prod_{\alpha_0 < \dots < \alpha_k} \U_{\alpha_0 \dots \alpha_k})
\end{align*}
% https://q.uiver.app/#q=WzAsMjAsWzAsM10sWzEsMywiQ157MCwwfSJdLFsyLDMsIkNeezEsMH0iXSxbMSw0LCJcXGhhdHtDfV4wKE0sIFxcbWF0aGZyYWt7VX0pIl0sWzIsNCwiXFxoYXR7Q31eMShNLCBcXG1hdGhmcmFre1V9KSJdLFszLDQsIlxcaGF0e0N9XjIoTSwgXFxtYXRoZnJha3tVfSkiXSxbMywzLCJDXnsyLDB9Il0sWzQsNCwiXFxjZG90cyJdLFsxLDIsIkNeezAsMX0iXSxbMSw1LCIwIl0sWzIsNSwiMCJdLFszLDUsIjAiXSxbMiwyLCJDXnsxLDF9Il0sWzMsMiwiQ157MiwxfSJdLFsxLDEsIkNeezAsMn0iXSxbMiwxLCJDXnsxLDJ9Il0sWzEsMCwiXFx2ZG90cyJdLFsyLDAsIlxcdmRvdHMiXSxbMywxLCJDXnsyLDJ9Il0sWzMsMCwiXFx2ZG90cyJdLFszLDEsImleMCIsMCx7InN0eWxlIjp7InRhaWwiOnsibmFtZSI6Imhvb2siLCJzaWRlIjoiYm90dG9tIn19fV0sWzQsMiwiaV4xIiwyLHsic3R5bGUiOnsidGFpbCI6eyJuYW1lIjoiaG9vayIsInNpZGUiOiJib3R0b20ifX19XSxbMyw0XSxbNCw1XSxbNSw2LCJpXjIiLDIseyJzdHlsZSI6eyJ0YWlsIjp7Im5hbWUiOiJob29rIiwic2lkZSI6ImJvdHRvbSJ9fX1dLFs1LDddLFsxLDgsImReezAsMH0iXSxbOSwzXSxbMTAsNF0sWzExLDVdLFsyLDEyLCJkXnswLDF9IiwyXSxbNiwxMywiZF57MCwyfSIsMl0sWzE0LDE2XSxbOCwxNCwiZF57MSwwfSJdLFsxMiwxNSwiZF57MSwxfSIsMl0sWzE1LDE3XSxbMTMsMTgsImReezEsMn0iLDJdLFsxOCwxOV1d
\[\begin{tikzcd}[cramped]
	& \vdots & \vdots & \vdots \\
	& {C^{0,2}} & {C^{1,2}} & {C^{2,2}} \\
	& {C^{0,1}} & {C^{1,1}} & {C^{2,1}} \\
	{} & {C^{0,0}} & {C^{1,0}} & {C^{2,0}} \\
	& {\hat{C}^0(M, \mathfrak{U})} & {\hat{C}^1(M, \mathfrak{U})} & {\hat{C}^2(M, \mathfrak{U})} & \cdots \\
	& 0 & 0 & 0
	\arrow[from=2-2, to=1-2]
	\arrow[from=2-3, to=1-3]
	\arrow[from=2-4, to=1-4]
	\arrow["{d^{1,0}}", from=3-2, to=2-2]
	\arrow["{d^{1,1}}"', from=3-3, to=2-3]
	\arrow["{d^{1,2}}"', from=3-4, to=2-4]
	\arrow["{d^{0,0}}", from=4-2, to=3-2]
	\arrow["{d^{0,1}}"', from=4-3, to=3-3]
	\arrow["{d^{0,2}}"', from=4-4, to=3-4]
	\arrow["{i^0}", hook', from=5-2, to=4-2]
	\arrow[from=5-2, to=5-3]
	\arrow["{i^1}"', hook', from=5-3, to=4-3]
	\arrow[from=5-3, to=5-4]
	\arrow["{i^2}"', hook', from=5-4, to=4-4]
	\arrow[from=5-4, to=5-5]
	\arrow[from=6-2, to=5-2]
	\arrow[from=6-3, to=5-3]
	\arrow[from=6-4, to=5-4]
\end{tikzcd}\]
While it might seem a bit pointless at this point there is a natural question that arises everytime 
one sees a diagram like this: Is there an obvious way to complete this such that all arrows commute?

The most obvious way of doing this would be to take $\prod_{\alpha_0 < \dots < \alpha_k} \U_{\alpha_0\dots\alpha_k}$
as a subspace of $\prod_{\alpha_0 < \dots < \alpha_{k - 1}} \U_{\alpha_0\dots\alpha_{k - 1}}$ and then connect
the columns using restriction of differential forms. But this naïve way would not commute with the differential of the Čech Cocomplex.
To see that we are going to define some notational conventions. For $\alpha_0 < \dots < \alpha_k$ we define:
\begin{align*}
	\mathds{1}_{\alpha_0\dots\alpha_k}(x) \coloneqq
		\begin{cases}
			1 &\quad x \in \U_\alpha \\
			0 &\quad \text{otherwise}
		\end{cases}
\end{align*}
To define this for every possible combinations of indices, let $j_0, \dots, j_k$ be some indices. If they are
pairwise different then there is a permutation $\sigma$ such that $j_{\sigma(0)} < \dots < j_{\sigma(k)}$ and we define:
\begin{align*}
	\indicator_{\alpha_0\dots\alpha_k} \coloneqq \sign(\sigma) \indicator_{j_{\sigma(0)}\dots j_{\sigma(k)}}
\end{align*}
If any indices are duplicated we define $\indicator_{\alpha_0\dots\alpha_k} \coloneqq 0$

With some abuse of notation we identify these indicator functions with the corresponding Čech cochains. Then the Čech 
differential acts in the following way:
\begin{align*}
	\delta \indicator_\alpha(c) = \indicator_\alpha(\partial c)
\end{align*}
By denoting the chain corresponding to the set $\U_{\alpha_0\dots\alpha_k}$ as $e_{\alpha_0\dots\alpha_k}$ for
$\alpha_0 < \dots < \alpha_K$ (extended to arbitrary indizes as above) we can describe this as follows:
Let $\beta \in I$, then:
\begin{align*} 
	\delta \indicator_\alpha(e_{\beta \alpha})
	=
	\indicator_\alpha(\partial e_{\beta \alpha})
	=
	\indicator_\alpha(e_\alpha - e_\beta)
	=
	1
\end{align*}
And by antisymmetry:
\begin{align*}
	\delta \indicator_\alpha(e_{\alpha \beta})
	= -1
\end{align*}
And for $\beta, \gamma \in I$ with $\beta \neq \alpha, \gamma \neq \alpha$:
\begin{align*}
	\delta \indicator_\alpha(e_{\beta \gamma}) = 0
\end{align*}
Thus $\delta \indicator_\alpha$ acts exactly as all the $\indicator_{\beta \alpha}$, or in other words:
\begin{align*}
	\delta \indicator_\alpha = \sum_{\beta \in I} \indicator_{\beta \alpha}
\end{align*}
This can be easily extended to any group of indices $\alpha_0, \dots, \alpha_k$ as:
\begin{align*}
	\delta \indicator_{\alpha_0 \dots \alpha_k} = \sum_{\beta \in I} \indicator_{\beta \alpha_0 \dots \alpha_k}
\end{align*}
This is not that different from restricting the indicator functions, it just adds some extra signs. Thus the correct
way to connect our lattice is going to go as follows: Let $\omega \in \Omega^\ell(\prod \U_{\alpha_0\dots\alpha_k})$.
Then we can write:
\begin{align*}
	\omega = \sum_{\alpha_0 < \dots < \alpha_k} \omega \indicator_{\alpha_0 \dots \alpha_k}
\end{align*}
and define a homomorphism $\delta^{k, \ell}$ by just defining it on $\omega \indicator_{\alpha_0 \dots \alpha_k}$
for any $\alpha_0 < \dots < \alpha_k$ by:
\begin{align*}
	\delta^{k, \ell} (\omega \indicator_{\alpha_0 \dots \alpha_k}) \coloneqq \sum_{\beta \in I} \omega \indicator_{\beta \alpha_0 \dots \alpha_k}
\end{align*}
It follows directly from the definition that these homomorphisms make our diagram commute. Even better: They make
the upper rows exact:
\begin{lemma}
The sequence
% https://q.uiver.app/#q=WzAsNCxbMCwwLCJDXnswLFxcZWxsfSJdLFsxLDAsIkNeezEsXFxlbGx9Il0sWzIsMCwiQ157MixcXGVsbH0iXSxbMywwLCJcXGNkb3RzIl0sWzAsMSwiXFxkZWx0YV57MCxcXGVsbH0iXSxbMSwyLCJcXGRlbHRhXnsxLFxcZWxsfSJdLFsyLDMsIlxcZGVsdGFeezIsXFxlbGx9Il1d
\[\begin{tikzcd}[cramped]
	{C^{0,\ell}} & {C^{1,\ell}} & {C^{2,\ell}} & \cdots
	\arrow["{\delta^{0,\ell}}", from=1-1, to=1-2]
	\arrow["{\delta^{1,\ell}}", from=1-2, to=1-3]
	\arrow["{\delta^{2,\ell}}", from=1-3, to=1-4]
\end{tikzcd}\]
is exact.
\end{lemma}
\begin{proof}
We start by showing that $\delta^{\bullet, \ell}$ is nilpotent. Let $\omega \in C^{k, \ell}$. W.l.o.g. we can assume that $\omega$
is only defined on one $\U_{\alpha_0 \dots \alpha_k}$. Then we have:
\begin{align*}
	(\delta^{k + 1, \ell} \circ \delta^{k,\ell}) \omega
		&= \sum_{\gamma \in I} \sum_{\beta \in I} \omega \indicator_{\beta \alpha_0 \dots \alpha_k} \indicator_{\gamma \beta \alpha_0 \dots \alpha_k} \\
		&= \sum_{\gamma \in I} \sum_{\beta \in I} \omega \indicator_{\gamma \beta \alpha_0 \dots \alpha_k}
\end{align*}
If $\beta = \gamma$ this is zero anyway, but if they are not equal, then there is also another summand with them switched and all other indices the same, hence with opposite sign and therefore this sum cancels to zero,
$\delta^{\bullet, \ell}$ is nilpotent.

It is left to show that $\ker(\delta^{k + 1, \ell}) \subset \Img(\delta^{k, \ell})$ holds. Let $\omega \in C^{k + 1, \ell}$ such that
$\delta^{k + 1, \ell}(\omega) = 0$. This is equivalent to stating that the support of $\omega$ is disjoint from all $\U_{\alpha_0 \dots \alpha_{k + 2}}$
TODO
\end{proof}
We can make this diagram even better by connecting it directly to the de Rham complex of the manifold itself. We can augment all rows at the left side by
the de Rham complex of the manifold itself, with the morphism being just the restriction of differential forms
\begin{lemma}
	The sequence
	% https://q.uiver.app/#q=WzAsNixbMiwwLCJDXnswLFxcZWxsfSJdLFszLDAsIkNeezEsXFxlbGx9Il0sWzQsMCwiQ157MixcXGVsbH0iXSxbNSwwLCJcXGNkb3RzIl0sWzEsMCwiXFxPbWVnYV5cXGVsbChNKSJdLFswLDAsIjAiXSxbMCwxLCJcXGRlbHRhXnswLFxcZWxsfSJdLFsxLDIsIlxcZGVsdGFeezEsXFxlbGx9Il0sWzIsMywiXFxkZWx0YV57MixcXGVsbH0iXSxbNCwwLCJqXlxcZWxsIl0sWzUsNF1d
\[\begin{tikzcd}[cramped]
	0 & {\Omega^\ell(M)} & {C^{0,\ell}} & {C^{1,\ell}} & {C^{2,\ell}} & \cdots
	\arrow[from=1-1, to=1-2]
	\arrow["{j^\ell}", from=1-2, to=1-3]
	\arrow["{\delta^{0,\ell}}", from=1-3, to=1-4]
	\arrow["{\delta^{1,\ell}}", from=1-4, to=1-5]
	\arrow["{\delta^{2,\ell}}", from=1-5, to=1-6]
\end{tikzcd}\]
	is exact.
\end{lemma}
\begin{proof}
	$j^\ell$ is clearly injective. Let $\omega \in \Omega^\ell(M)$, then we can abuse notation to write:
	\begin{align*}
		j^\ell(\omega) = \sum_{\alpha \in I} \omega \indicator_\alpha
	\end{align*}
	Thus we have:
	\begin{align*}
		(\delta^{0, \ell} \circ j^\ell)(\omega) = \sum_{\beta \in I} \sum_{\alpha \in I} \omega \indicator_{\alpha \beta}
	\end{align*}
	And as in a similiar argument above these cancel each other out, proving $\delta^{0, \ell} \circ j^\ell = 0$. 

	Now let $\sum_{\alpha \in I} \omega_\alpha \indicator_\alpha$ be in the kernel of $\delta^{0, \ell}$. Let $\rho_\alpha$ be a partition of unity
	subordinate to $I$. Because it is in the kernel the support of $\omega_\alpha$ is disjoint from all $\U_\beta$ for $\beta \neq \alpha$ and hence we have:
	\begin{align*}
		\sum_{\alpha \in I} \omega_\alpha \indicator_\alpha = \sum_{\alpha \in I} \omega_\alpha \indicator_\alpha \rho_\alpha
	\end{align*}
	Hence:
	\begin{align*}
		j^\ell(\sum_{\alpha \in I} \omega_\alpha \rho_\alpha) = \sum_{\alpha \in I} \omega_\alpha \indicator_\alpha
	\end{align*}
	and the sequence is exact.
\end{proof}
\begin{remark}
Note that we did slightly more than necesarry: We also provided a left inverse of $j^\ell$. This will become important later when we are going to
try to construct an explicit isomorphism between Čech and de Rham cohomology.
\end{remark}
This big diagram we produced is called the \underline{Čech-de Rham double complex}. It has a fancy name and looks nice, but how do we get any concrete relationship between our cohomologies out of it?
This can actually be done with purely algebraic methods and hence we are going to have to take a detour into the land of commutative algebra and look at so called \textit{double complexes} as their own
objects.
\section{Double Complexes}
The diagram we constructed in the last paragraph is an instance of a more abstract phenomenom:
\begin{definition}[Double Complex]
	A \underline{double complex} is a $\ZZ \times \ZZ$-graded vector space
	$C^{\bullet, \bullet} := \{C^{p,q}\}_{p,q \in \ZZ}$ together with linear
	maps
	\[
		d^{p,q}_h: C^{p, q} \to C^{p + 1, q}
		\quad
		\text{ and }
		\quad
		d^{p,q}_v: C^{p,q} \to C^{p, q + 1}
	\]
	for every $p,q \in \ZZ$ (with $d^{\bullet, \bullet}_h$ as usual referring
	to the unique linear map that restricts to $d^{p,q}$ on $C^{p,q}$ and
	similiar for $d^{\bullet, \bullet}_v$),  with $\dvb, \dhb$ each being
	nilpotent and commutative with each other, i.e:
	\[
		\dvb \circ \dvb = \dhb \circ \dhb = 0
	\]
	And the following commutes for every $p,q$:
	% https://q.uiver.app/#q=WzAsNCxbMCwwLCJDXntwLCBxKzF9Il0sWzAsMSwiQ157cCxxfSJdLFsxLDEsIkNee3AgKyAxLCBxfSJdLFsxLDAsIkNee3ArMSxxKzF9Il0sWzEsMCwiZF92XntwLHF9Il0sWzEsMiwiZF57cCxxfV9oIiwyXSxbMCwzLCJkXntwLHErMX1faCJdLFsyLDMsImRee3ArMSxxfV92IiwyXV0=
	\[
		\begin{tikzcd}
			{C^{p, q+1}} & {C^{p+1,q+1}} \\
			{C^{p,q}} & {C^{p + 1, q}}
			\arrow["{d^{p,q+1}_h}", from=1-1, to=1-2]
			\arrow["{d_v^{p,q}}", from=2-1, to=1-1]
			\arrow["{d^{p,q}_h}"', from=2-1, to=2-2]
			\arrow["{d^{p+1,q}_v}"', from=2-2, to=1-2]
		\end{tikzcd}\]
\end{definition}
\begin{remark}
	Some authors (e.g. \cite{weibel_introduction_1994}) prefer these maps to
	be anticommutative instead of commutative. While this saves some signs
	in some places (e.g. in the definition of homology of a chain complex),
	it also adds them in other places and since the later dominates in this
	thesis, we are going to follow \cite{tu_differential_1982} and assume
	commutativity.
\end{remark}

Double complexes can be seen as a generalization of chain complexes\footnote{
	The high-brow approach to this would be to define chain complexes over general abelian categories,
	then a double complex is "just" a chain complex of chain complexes.
}, hence we would also like to take its homology. This is done by first reducing the double complex back
to a chain complex, by summing over the diagonals:
\begin{definition}[Associated Chain Complex]
	Given a double complex $C^{\bullet, \bullet}$ with maps $\dvb, \dhb$ the
	\underline{
		associated chain complex of $C^{\bullet, \bullet}$
		}, denoted as
	$C^\bullet(C^{\bullet, \bullet})$ is the chain complex defined as:
	\[
		C^n(C^{\bullet, \bullet}) \coloneq \bigoplus_{n = p + q} C^{p,q}
	\]
	with the differential $d^n: C^n \to C^{n + 1}$ being defined as:
	\[
		d^n \coloneq \sum_{p + q = n} d^{p,q}_v + (-1)^p d^{p,q}_h
	\]
\end{definition}
The usefullness of this definition heavily depends on this actually being a
chain complex, hence the first thing one wants to proof:
\begin{lemma}
	For a given double complex $C^{\bullet, \bullet}$ with maps $\dvb$ and $\dhb$
	the associated chain complex is a chain complex.
\end{lemma}
\begin{proof}
	Being defined as a $\ZZ$-graded vector space we only have to proof that $d^\bullet$ is nilpotent.
	Let $c \in C^{p,q}$ for some $p,q \in \ZZ$. Then:
	\begin{align*}
		(d^{p + q + 1} \circ d^{p + q})(c)
		&=
		d^{p + q + 1}(d^{p,q}_v(c) + (-1)^p d^{p,q}_h(c)) \\
		&=
		\overbrace{
			(d^{p + 1, q}_v \circ d^{p,q}_v)(c)
		}^{\text{nilpotent}}
		+
		\overbrace{
			(d^{p, q + 1}_v \circ d^{p,q}_v)(c)
		}^{\text{nilpotent}} \\
		&\quad+ (-1)^p (\underbrace{
				(d^{p + 1, q}_v \circ d^{p,q}_h)(c)
				- (d^{p, q + 1}_h \circ d^{p,q}_v)(c)
			)
		}_{\text{commutative}}) \\
		&= 0 \qedhere
	\end{align*}
\end{proof}
\begin{remark}
	By abuse of notation we will sometimes call $C^{\bullet, \bullet}$
	a chain complex, this always refers to $C^\bullet(C^{\bullet,\bullet})$.
\end{remark}
With this definition it becomes obvious how to define the homology of a double complex:
\begin{definition}[Homology of a Double Complex]
	For a given double complex $C^{\bullet,\bullet}$ we define the
	\underline{homology of the double complex} as the homology of the
	associated chain complex, in notation:
	\[
		H^\bullet(C^{\bullet,\bullet}) := H^\bullet(C^\bullet(C^{\bullet,\bullet}))
	\]
\end{definition}

Double complexes form a fascinating part of commutative algebra, being used to defined such marvelous (and
mysterious) tools like spectral sequences and can be definitely their own subject of study.

We however are only interested in them as a tool in our proof and are therefore going to add some
additional assumptions about them, particularly we are going to assume that all future double complexes
only have positive parts i.e. $C^{p,k} = 0$ for $p < 0$ or $k < 0$.

But the Čech-de Rham double complex was not really just a double complex, but one that has been augmented at the left
and below. We would like to show that the cohomologies of these augmentations are isomorphic by showing that they
both are isomorphic to the cohomology of the double complex itself. This is not true in general, but with some
convenient assumptions it is true:
\begin{theorem}
\label{theorem:augmentation_iso_homology}
Let $C^{\bullet,\bullet}$ be a double complex, $(A^\bullet, \delta)$ a chain
complex.

If for every $n >= 0$ there is a map
\[
	\alpha^n: A^n \to C^{n,0}
\]
such that the following diagram commutes:
\begin{center}
% https://tikzcd.yichuanshen.de/#N4Igdg9gJgpgziAXAbVABwnAlgFyxMJZAJgBoBmAXVJADcBDAGwFcYkQBhAPWAAZTeAXxCDS6TLnyEU5CtTpNW7bsACMA4aPHY8BIgBY5NBizaJOPMkJFiQGHVKIBWIwtPsAOh6gQcCLXYSutIkpMTyJkrmKvyqmrb2knooZKoRimYWfGHx2kkhsmnGGco86sS5gQ7JyLLhxe7RZaRxNnnBBmHpjVmplYkdKIZFblG9OW1V+URW3WNetD5+kwOOMgJzmQtL-glBa8iGvJueHou+u+0HLvWjW94XK-s1LiOR9ztP1SHqbyXmAEEuMQvtMUOUToCuKpQYNkOoqA0xkDeLCDupjkj7udlgFVjVYpCQKi8c8Qvxbu92CS9t8iPxEXdqSJ5DAoABzeBEUAAMwAThAALZIfggHAQJDqJnmKA8fhCAD6AAtJvyhZKaOKkGRpSBZdk4srVQLhYgdVrEOQAmrTaKLYZdfr5YIFbRjerEA6LS5HXKWi63daTUgfRaAGxY9hOnJGoMer0SxAAdkjMr9hpVcdNKbFiYjvrU-td7tN+YtAE5U3rLEXM7YbUgcxaABxV-W-F113nBxCt3NISsFvrFrMa-vJtvNRWB+s982JvtUtPAKwBkshzWJ1RCWcestbuK702Di2qCpHpB909Wi+IE9b-SjxCqJtb-NLkBeWCMHD0Lg07sPVUfdJR9D8vxgH8-xhJ9VFDLdm1g+9QNg+CNzFegsEYdglQgCAAGsQCrLwmDQJU-xBJ9iDtN9YJAz1NUw7DzFwgiiN1EjGDI6D1zNKVTyTWDXxFRisJwvDCOIjxSPI-8WUEIA
\begin{tikzcd}
            & \vdots                                                & \vdots                                                      & \vdots                                                      & \vdots                                       &       \\
0 \arrow[r] & A^2 \arrow[u] \arrow[r, "\alpha^2", hook]             & {C^{0,2}} \arrow[r, "{d^{0,2}_h}"] \arrow[u]                & {C^{1,2}} \arrow[r, "{d^{2,1}_h}"] \arrow[u]                & {C^{2,2}} \arrow[u] \arrow[r]                & \dots \\
0 \arrow[r] & A^1 \arrow[u, "\delta^1"] \arrow[r, "\alpha^1", hook] & {C^{0,1}} \arrow[u, "{d^{0,1}_v}"] \arrow[r, "{d^{0,1}_h}"] & {C^{1,1}} \arrow[u, "{d^{1,1}_v}"] \arrow[r, "{d^{1,1}_h}"] & {C^{2,1}} \arrow[u, "{d^{2,1}_v}"] \arrow[r] & \dots \\
0 \arrow[r] & A^0 \arrow[u, "\delta^0"] \arrow[r, "\alpha^0", hook] & {C^{0,0}} \arrow[r, "{d^{0,0}_h}"] \arrow[u, "{d^{0,0}_v}"] & {C^{1,0}} \arrow[r, "{d^{0,1}_h}"] \arrow[u, "{d^{1,0}_v}"] & {C^{2,0}} \arrow[r] \arrow[u, "{d^{2,0}_v}"] & \dots
\end{tikzcd}
\end{center}
and every row is exact, then
\[
	H^\bullet(A^\bullet) \cong H^\bullet(C^{\bullet,\bullet})
\]
\end{theorem}
Before proving this we are going to need a convenient lemma:
\begin{lemma}
\label{lemma:helpfull}
Given a double complex $C^{\bullet, \bullet}$ for which the sequence
\begin{center}
% https://tikzcd.yichuanshen.de/#N4Igdg9gJgpgziAXAbVABwnAlgFyxMJZABgBpiBdUkANwEMAbAVxiRAGEA9YMtAXxB9S6TLnyEUARnJVajFmy7Bp-QcJAZseAkQBMM6vWatEHbvtVCRW8UQDMBuccXcHl9ZrE6UAFkdGFUwAdIKgIHAQ+WRgoAHN4IlAAMwAnCABbJDIQHAgkSSsQVIz86lykXULizMR9HLzEOyq0mod6pB8oviA
\begin{tikzcd}
{C^{0,p}} \arrow[r] & {C^{1,p}} \arrow[r] & {C^{2,p}} \arrow[r] & {C^{3,p}} \arrow[r] & \dots
\end{tikzcd}
\end{center}
is exact for every $p$, then every homology class
$[c] \in H^n(C^{\bullet, \bullet})$ is homologous to one represented by a
$\tilde{c} \in C^{0,n})$, i.e. $[c] = [\tilde{c}]$.
\end{lemma}
\begin{proof}
Let $[c] \in H^n(C^{\bullet, \bullet})$. One can write:
\[
	c = c^{0, n} + c^{1, n - 1} + \dots + c^{k, \ell}
\]
with $c^{i,j} \in C^{i,j}$.

We just have to prove that $c^{k, \ell}$ is homologous to an element in $C^{k - 1, \ell + 1}$,
the rest follows from linearity.

It follows from $c \in \ker(d^n)$ that $\rmd_h^{k, \ell} c^{k, \ell} = 0$ and hence by exactness
of the rows there is a $c^{k - 1, \ell}$ such that:
\[
	\rmd_h^{k - 1, \ell} c^{k - 1, \ell} = c^{k, \ell}
\]
Which implies:
\begin{align*}
	&    &\rmd^{n - 1} c^{k - 1, \ell} = (-1)^{k - 1} \rmd_v^{k - 1, \ell} c^{k - 1, \ell} + c^{k, \ell}\\
	&\iff&  (-1)^{k - 1} \rmd_v^{k - 1, \ell} c^{k - 1, \ell} = c^{k, \ell} - \rmd^{n - 1} c^{k - 1, \ell} \\
	&\implies& [\underbrace{(-1)^{k - 1} \rmd_v^{k - 1, \ell} c^{k - 1, \ell}}_{\in C^{k - 1, \ell + 1}}] = [c^{k, \ell}]
\end{align*}
Thus by induction $[c] = [\tilde{c}]$ for some $\tilde{c} \in C^{0, n}$.
\end{proof}
This makes the proof of the original statement way easier:
\begin{proof}[Proof of \ref{theorem:augmentation_iso_homology}]
We start by proving that $\alpha^\bullet$ is actually a chain map between $A^\bullet$ and $C^{\bullet, \bullet}$.
Let $a \in A^\ell$. Then we have:
\begin{align*}
	(\rmd^\ell \circ \alpha^\ell)(a) &= \rmd_v \alpha^\ell a + \underbrace{d_n \alpha^\ell}_{=0 \text{ by exactness}} a = \rmd_v \alpha^\ell a \\
	(\alpha^{\ell + 1} \circ \delta^\ell)(a) &= \rmd_v \alpha^\ell a
\end{align*}
Hence $\alpha^\bullet$ is a chain map and descends to cohomology.

It is left to proof that $\alpha^\bullet$ descends to an isomorphism.

We start by showing that the map induced by $\alpha^n$ on cohomology is injective.
Let $a \in A^n$ such that $[\alpha^n(a)] = [0]$. By definition
there then exists a $c \in C^{0, n - 1}$ such that:
\begin{align*}
	\rmd c = \alpha^n(a)
\end{align*}
This especially implies $\rmd_h c = 0$ and hence by exactness there is a
$\tilde{a} \in A^{n - 1}$ such that $\alpha^{n - 1}(\tilde{a}) = c$. By commutativity we thus have:
\begin{align*}
	\alpha^n(a) = \rmd c = (\rmd \circ \alpha^{n - 1})(\tilde{a}) = (\alpha^n \circ \delta)(\tilde{a})
\end{align*}
And by injectivity of $\alpha^n$\footnote{
	Of the original map, not of the induced map on cohomology. Yes, it is confusing.
} $\delta(\tilde{a}) = a$, hence $[a] = [0]$ and $\alpha^n$ induces an injective map on cohomology.

Surjectivity of $\alpha^\bullet$ can now be proven by using Lemma
\ref{lemma:helpfull}. Let $[c] \in H^n(C^{\bullet, \bullet})$, by Lemma \ref{lemma:helpfull}
we may assume w.l.o.g. that $c \in C^{0,n}$. By definition of homology
$c \in \ker(d^n)$ and therefore $d_v(c) = d_h(c) = 0$, by exactness
there then is a $a \in A^n$ such that $\alpha^n(a) = c$ and by commutativity:
\[
	0 = (d^{0,n}_v \circ \alpha^n)(a) = (\alpha^{n + 1} \circ \delta^n)(a)
\]
and by injectivity of $\alpha^{n + 1}$ also $\delta^n(a)$, implying
$[a] \in H^n(A^\bullet)$ and especially:
\[
	\alpha^\bullet([a]) = [\alpha^\bullet(a)] = [c]
\]
Since $[c]$ was chosen arbitarily $\alpha^\bullet$ is therefore also
surjective.
\end{proof}

Everything we did here can be also done for a double complex with exact
coloumns instead of rows and augmented at the bottom instead of the left.
This requires some sign trickery, but leads to the final version of our
statement
\begin{corolarry}
Given two chain complexes $A^\bullet, B^\bullet$ a double complex
$C^{\bullet,\bullet}$ and maps $\alpha^\bullet, \beta^\bullet$ such that
\begin{center}
% https://tikzcd.yichuanshen.de/#N4Igdg9gJgpgziAXAbVABwnAlgFyxMJZAJgBoBGAXVJADcBDAGwFcYkQBhAPWAAZTiAXxCDS6TLnyEUAZgrU6TVu27BypNMNHjseAkQAs8mgxZtEnHmSEixIDLqlFrC08our+5LXYeT9KGQyrkrmlnykvD46-tLIcsQhZio86t62MXpxcsEmoSlqkdH2ElmGAknu4WTp2iWOAchGuYrJHlZFGfWxRACsxq1VADpDUBA4CHV+ZSj9iXltICNjE13TTrOkLW5hy+OTvqUbTZGVu0O0KweZx3K8Z+wjl-trR41k9wvDF1evDXFpB4WACCXGIfx6KHU80GYVB5AhM2Q6k+sMePxeUzeAK2QJAoN4iOOZAMeIAQlxCVj-kQ5KSvmEKQjqZCTvS0RYKeCWUj+uydujfjzjl48VTDjSUPwYQKLOKbo1+Nt8nKie9SL0xWrshqtcLGkZNQz2FSFDAoABzeBEUAAMwAThAALZIfggHAQJDeOwO51emgepBCH2Ol2IOTuz2IKIhv2IIyRpAyOq+sMRwOIXop0NIBMZgBs2bj+YDUazsbD-UTiG9dpziAA7KWg0WwwAOZuN1tIEvVtvdzOdhsDsjVgCcA6b1fIMbrcY709rIFTQc75GDc7D6mnyYrrrXBgH5CrGbdsqWQyYaAAFvQwSAA-QsIx2NeIBAANZdFc13sZ49HieUbkMOe41gu-6FmB5AQVGCbniMV63pSD7uk+L4WG+n7fvW5B-lGEYIZejA3neCKPs+r7vl+R5jp2xCzsu9YMfRS4-sQ24ZsQG5MXG5B0dW8EqheABGMA4HehIURhIBYTRYEsdWvZEWJElcORaGUZh1E4XGHGdguKniXe4LSVR2EjumwHlpuQZ5sBUG2YgxBAV6-YKfhXoTgpU5cYx7GwUG3iUIIQA
\begin{tikzcd}
            & \vdots                                    & \vdots                                   & \vdots                                   & \vdots                                   &       \\
0 \arrow[r] & A^2 \arrow[r, "\alpha^2", hook] \arrow[u] & {C^{0,2}} \arrow[r] \arrow[u]            & {C^{1,p}} \arrow[r] \arrow[u]            & {C^{2,2}} \arrow[r] \arrow[u]            & \dots \\
0 \arrow[r] & A^1 \arrow[u] \arrow[r, "\alpha^1", hook] & {C^{0,1}} \arrow[u] \arrow[r]            & {C^{1,1}} \arrow[u] \arrow[r]            & {C^{2,1}} \arrow[u] \arrow[r]            & \dots \\
0 \arrow[r] & A^0 \arrow[u] \arrow[r, "\alpha^0", hook] & {C^{0,0}} \arrow[u] \arrow[r]            & {C^{1,0}} \arrow[u] \arrow[r]            & {C^{2,0}} \arrow[u] \arrow[r]            & \dots \\
            &                                           & B^0 \arrow[r] \arrow[u, "\beta^0", hook] & B^1 \arrow[r] \arrow[u, "\beta^1", hook] & B^2 \arrow[r] \arrow[u, "\beta^2", hook] & \dots \\
            &                                           & 0 \arrow[u]                              & 0 \arrow[u]                              & 0 \arrow[u]                              &      
\end{tikzcd}
\end{center}
commutes with every row and coloum except for the left- and bottommost
being exact, then:
\[
	H^\bullet(A^\bullet)
		\cong H^\bullet(C^{\bullet, \bullet})	
		\cong H^\bullet(B^\bullet)
\]

\end{corolarry}
\begin{proof}
As said this follows from \ref{theorem:augmentation_iso_homology}
using symmetry and sign trickery.
\end{proof}


\section{Mayer-Vietoris to the Infinite}
It remained to show that the rows of the double complex are exact.
\begin{theorem}[The Generalized Mayer-Vietoris Sequence]
The sequence
% https://q.uiver.app/#q=WzAsNSxbMSwwLCJcXE9tZWdhXiooTSkiXSxbMCwwLCIwIl0sWzIsMCwiXFxPbWVnYV4qKFxcY29wcm9kX3tcXGFscGhhXzAgPCBcXGFscGhhXzF9IFxcbWF0aGNhbHtVfV97XFxhbHBoYV8wLCBcXGFscGhhXzF9KSJdLFszLDAsIlxcT21lZ2FeKihcXGNvcHJvZF97XFxhbHBoYV8wIDwgXFxhbHBoYV8xIDwgXFxhbHBoYV8yfSBcXG1hdGhjYWx7VX1fe1xcYWxwaGFfMCwgXFxhbHBoYV8xLCBcXGFscGhhXzJ9KSJdLFs0LDAsIlxcZG90cyJdLFsxLDBdLFswLDJdLFsyLDNdLFszLDRdXQ==
\[\begin{tikzcd}[column sep=small]
	0 & {\Omega^*(M)} & {\Omega^*(\coprod_{\alpha_0 < \alpha_1} \mathcal{U}_{\alpha_0, \alpha_1})} & {\Omega^*(\coprod_{\alpha_0 < \alpha_1 < \alpha_2} \mathcal{U}_{\alpha_0, \alpha_1, \alpha_2})} & \dots
	\arrow[from=1-1, to=1-2]
	\arrow[from=1-2, to=1-3]
	\arrow[from=1-3, to=1-4]
	\arrow[from=1-4, to=1-5]
\end{tikzcd}\]
is exact.
\end{theorem}
\begin{proof}
We already proved that $\delta^2 = 0$, hence it is enough to find a map $r$ that is a
left inverse to $\delta$ on the kernel.

Let $\omega \in \Omega^*(\coprod_{\alpha_0 < \dots < \alpha_k} \mathcal{U}_{\alpha_0, \dots, \alpha_k})$.
Let $\{\rho_\alpha\}$ be a partition of unity subordinate to the cover $\{\mathcal{U}_\alpha\}$.

Let $\omega$ be a closed $k$-form on $\coprod \mathcal{U}_{\alpha_0, \dots, \alpha_p}$. Then we define
a $k$-form $\tau$ on $\coprod \mathcal{U}_{\alpha_0, \dots, \alpha_{p - 1}}$ by:
\[
	\tau_{\alpha_0, \dots, \alpha_{p - 1}} = \sum_{\beta \in I} \rho_\beta \omega_{\beta, \alpha_0, \dots, \alpha_{p - 1}}
\]


\end{proof}

\section{Explicit Isomorphism \& Applications}

Let us take a good cover $\mathcal{U}_\alpha, \mathcal{U}_\beta, \mathcal{U}_\gamma$ of $S^1$. The nerve of
this cover is shown in TODO, together with a choice of orientation.
\tikzset{
  % style to apply some styles to each segment of a path
  on each segment/.style={
    decorate,
    decoration={
      show path construction,
      moveto code={},
      lineto code={
        \path [#1]
        (\tikzinputsegmentfirst) -- (\tikzinputsegmentlast);
      },
      curveto code={
        \path [#1] (\tikzinputsegmentfirst)
        .. controls
        (\tikzinputsegmentsupporta) and (\tikzinputsegmentsupportb)
        ..
        (\tikzinputsegmentlast);
      },
      closepath code={
        \path [#1]
        (\tikzinputsegmentfirst) -- (\tikzinputsegmentlast);
      },
    },
  },
  % style to add an arrow in the middle of a path
  mid arrow/.style={postaction={decorate,decoration={
        markings,
        mark=at position .5 with {\pgftransformscale{2}\arrow[#1]{stealth}}
      }}},
}
\begin{figure}
\begin{tikzpicture}
    \tikzstyle{point}=[thick,draw=black,inner sep=0pt]
    \node[label={[below left]$\alpha$}] (a) at (0,0) {};
    \node[label={[below right]$\beta$}] (b) at (5,0) {};
    \node[label={[above]$\gamma$}] (c) at (2.5,4.33) {};
	% yshift=-10pt
	% xshift=7pt, yshift=7pt
	% xshift=-7pt, yshift=7pt
    \draw[postaction={on each segment={mid arrow}}]
		(a.center) -- node[midway, label={[label distance=10pt]270:$c$}] {}
		(b.center) -- node[midway, label={[label distance=10pt]0:$a$}] {}
		(c.center) -- node[midway, label={[label distance=10pt]180:$b$}] {}
		cycle;
	\foreach \n in {a,b,c}
	{
		\fill (\n.center) circle (1.5pt);
	}
\end{tikzpicture}
\end{figure}
The Čech cochains of this complex are:
\begin{align*}
	C^0(K) = \langle \alpha, \beta, \gamma \rangle, \,
	C^1(K) = \langle a, b, c \rangle
\end{align*}
With the coboundary operator being defined as:
\begin{align*}
	\delta \alpha = b - c,\quad 
	\delta \beta = c - a,\quad
	\delta \gamma = a - b
\end{align*}
We already calculated the cohomology of this, its $\hat{H}^0(K) = \hat{H}^1(K) = \RR$,
with the 1-cohomology being generated by $[\alpha] = [\beta] = [\gamma]$. We can 
