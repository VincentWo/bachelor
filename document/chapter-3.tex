\chapter{The Čech-de Rham Double Complex}
When we introduced the Čech and the de Rham Cohomology we did so for a similiar
reason: To find an algebraic invariant that corresponds to the "holes" of a space.
This conceptual simmilarity is a strong reason for us to believe that both cohomologies
should give the same result for smooth manifolds, in spite of their otherwise completely
different construction. And in fact: for the limited amount of spaces at which we looked
at so far, they booth indeed produced the same results.

In this chapter we are going to make this hope a reality by proving the "de Rham theorem":
\begin{theorem}[De Rhams theorem]
Let $M$ be a smooth manifold. There is a natural?? isomorphism between the de Rham cohomology of $M$
and the Čech cohomology of any good cover of $M$ (for $M$ seen as a topological manifold)
\end{theorem}

This is not only a remarkable fact, but it also solves some non-trivial remaining issues
we had with both cohomologies. Our main problem with Čech-chomology was the potential dependence
on a specific choice of cover, but since the de Rham cohomology does not depend on any cover,
this isomorphism lets us conclude:
\begin{corolarry}
Let $M$ be a smooth manifold. The Čech cohomology of a good cover of $M$ is independent
of the choice of good cover.
\end{corolarry}
While not being a practical issue for us so far, there are indeed smooth manifolds that
are homeomorphic, but not diffeomorphic. There is hence no a priori reason for de Rham
cohomology to be a topological invariant, but one would hope that a measurement of "holes"
is a topological invariant and since Čech cohomology is defined only in terms of topological
properties, it also follows:
\begin{corolarry}
Let $M, N$ be two homeomorphic smooth manifolds. Then the de Rham cohomology of $M$ is naturally
isomorphic to the de Rham cohomology of $N$.
\end{corolarry}

% techniques one usually uses & their drawbacks, so far (research more):
% - Integration over singular simplices (Drawbacks no clear)
% - Using sheaf cohomology (Very algebraic/category theoric, research: concrete isomorphism)
% - Using spectral sequences (Not clear whether our approach is basically using a spectral sequence)

We are going to proof this using the approach outlined in TODO
To achieve this we are going to find a connection between the Čech-Cohomology
of an open cover and the de Rham-cohomology of it's intersection in the form
of a exact sequence, by finding an alternate representation of Čech-Cohomology
using locally const functions.
\begin{definition}
Let $M$ be a smooth manifold with open covering $\mathfrak{U}$. Define
the \underline{chain complex of locally const functions} for every $n \geq 0$
as:
\[
	C^n_l(M, \mathfrak{U}) 
		\coloneq \prod_{\alpha_0, \dots, \alpha_n}
		\{\, f: \bigcap_{i = 0}^n U_{\alpha_i} \to \RR \  \text{locally constant} \,\}
\]
and the coboundary operator:
\begin{align}
	\delta^n: C^n_l &\to C^{n + 1}_l, \\
	f &\mapsto \prod_{i = 0}^\infty (-1)^c
\end{align}
\end{definition}

% Theorem that this identical to the Cech cohomology for good enough covers

\section{The Double Complex}
Extending the Mayer-Vietoris Sequence to countable covers gives us all the
differential geometric tools we need, what remains is mostly Algebra in the form
of diagram chasing.

\begin{definition}[Double Complex]
	A \underline{double complex} is a $\ZZ \times \ZZ$-graded abelian group
	$C^{\bullet, \bullet} := \{C^{p,q}\}_{p,q \in \ZZ}$ together with group
	homomorphisms
	\[
		d^{p,q}_h: C^{p, q} \to C^{p + 1, q}
		\quad
		\text{ and }
		\quad
		d^{p,q}_v: C^{p,q} \to C^{p, q + 1}
	\]
	for every $p,q \in \ZZ$ (with $d^{\bullet, \bullet}_h$ as usual referring
	to the unique homomorphism that restricts to $d^{p,q}$ on $C^{p,q}$ and
	similiar for $d^{\bullet, \bullet}_v$),  with $\dvb, \dhb$ each being
	nilpotent and commutative with each other, i.e:
	\[
		\dvb \circ \dvb = \dhb \circ \dhb = \dhb \circ \dvb - \dvb \circ \dhb = 0
	\]
\end{definition}
\begin{remark}
	Some authors (e.g. \cite{weibel_introduction_1994}) prefer these maps to
	be anticommutative instead of commutative. While this saves some signs
	in some places (e.g. in the definition of homology of a chain complex),
	it also adds them in other places and since the later dominates in this
	thesis, we are going to follow \cite{tu_differential_1982} and assume
	commutativity.
\end{remark}

Differently phrased: A double complex is a commutative lattice of groups:
\begin{center}
	% https://tikzcd.yichuanshen.de/#N4Igdg9gJgpgziAXAbVABwnAlgFyxMJZARgBoAGAXVJADcBDAGwFcYkQAdD2qCHBALNl0mXPkIoATBWp0mrdlx59BwkBmx4CRAMwyaDFm0SduvfiCEjN4ouVLFZhhSa7nV1sdpRlHB+cYgAMIAesD2kgKWahpeEsjSfnJG7KHAZJHRnlrxeknOgWnSmVbqojlEACwOTgHsAHSNWWU23sgRtSmuHO7NsRU+pJKdLsFh9sRRpf22UkMjhWG+UzHls8h6w-5dY8CJK9nr1VvJo431fWtt9joLij0ql63xZLfbo2n25ActcUTSb1Oi3SFB+MzaekBBVSYWk3yefxQ1ShdRM5wRAxIpEqd26ygs0yu8WkOPegSUvUJz102NxpnxglkMCgAHN4ERQAAzABOEAAtkg9CAcBAkNUgUgwMxGIwaIx6AAjGCMAAKRPYjBgnJwzR5-LFNBFSHsEsQUplcsVyrV1JMmu1ut5AsQ4qNiAArGTJdLZSB5UrVeq7VqdaU9c7PcLRYgyKbzb7-dag36Q479R7DdGAGxes0+y0Bm2IlMOsNOpA5qNIaRx-N+q2B20l0NqcMVzNIADsufjBaTTftLa55cQAA4O4gAJw9uuJxvFwdp53TqsumcW+uF5OLsvpldu4gm6F5jdzosDZtLpCHifEWPH3ub-sL1O75031eRh+zhvn2aXt9r1jA8a2-U9f23V9WxHO8J0rMCEwggcoOHdNiCFA9xQQvt5wvHdoLQ11o2IL9UUfM9INLAj3yIpAV2wp9cP-fDUPfSMD3gsify3ZCqNY692OIo8uPAniXz4kA2xjSsD27WtROfPCUMkmCZOI+8RMQsSlIkqTiHHVdiCwzScL-bwAOo6992I0iunIpDxKHFT00kE0OPXLTFOY19KAEIA
\begin{tikzcd}
                & \vdots                        & \vdots                        & \vdots                        &     \\
\dots \arrow[r] & {C^{0,2}} \arrow[u] \arrow[r] & {C^{1,2}} \arrow[u] \arrow[r] & {C^{2,2}} \arrow[u] \arrow[r] & ... \\
\dots \arrow[r] & {C^{0,1}} \arrow[u] \arrow[r] & {C^{1,1}} \arrow[r] \arrow[u] & {C^{2,1}} \arrow[r] \arrow[u] & ... \\
\dots \arrow[r] & {C^{0,0}} \arrow[r] \arrow[u] & {C^{1,0}} \arrow[r] \arrow[u] & {C^{2,0}} \arrow[r] \arrow[u] & ... \\
                & \vdots \arrow[u]              & \vdots \arrow[u]              & \vdots \arrow[u]              &    
\end{tikzcd}
\end{center}
with each row and column being a chain complex.

As the double complex can be seen as a generalization of the chain complex one
naturally wonders whether one can define the homology of it.
To accomplish this we first need to reduce the double complex to a chain 
complex, thus we define:
\begin{definition}[Associated Chain Complex]
	Given a double complex $C^{\bullet, \bullet}$ with maps $\dvb, \dhb$ the
	\underline{associated chain complex of $C^{\bullet, \bullet}$}, denoted as
	$C^\bullet(C^{\bullet, \bullet})$ is defined as:
	\[
		C^n(C^{\bullet, \bullet}) \coloneq \bigoplus_{n = p + q} C^{p,q}
	\]
	with the associated map $d^n: C^n \to C^{n + 1}$
	defined as:
	\[
		d^n \coloneq \sum_{p + q = n} d^{p,q}_v + (-1)^p d^{p,q}_h
	\]
\end{definition}
The usefullness of this definition heavily depends on this actually being a
chain complex, hence the first thing one wants to proof:
\begin{lemma}
	For a given double complex $C^{\bullet, \bullet}$ with maps $\dvb$ and $\dhb$
	the associated chain complex is a chain complex.
\end{lemma}
\begin{proof}
	Being a $\ZZ$-graded abelian group immediately follows from the definition,
	so it only remains to proof that $d^\bullet$ is nilpotent.

	Let $c \in C^{p,q}$ for some $p,q \in \ZZ$. Then:
	\begin{align*}
		(d^{p + q + 1} \circ d^{p + q})(c)
		&=
		d^{p + q + 1}(d^{p,q}_v(c) + (-1)^p d^{p,q}_h(c)) \\
		&=
		\overbrace{
			(d^{p + 1, q}_v \circ d^{p,q}_v)(c)
		}^{\text{nilpotent}}
		+
		\overbrace{
			(d^{p, q + 1}_v \circ d^{p,q}_v)(c)
		}^{\text{nilpotent}} \\
		&+ (-1)^p (\underbrace{
				(d^{p + 1, q}_v \circ d^{p,q}_h)(c)
				- (d^{p, q + 1}_h \circ d^{p,q}_v)(c)
			)
		}_{\text{commutative}}) \\
		&= 0 \qedhere
	\end{align*}
\end{proof}
Note that by abuse of notation we will sometimes call $C^{\bullet, \bullet}$
a chain complex, this always refers to $C^\bullet(C^{\bullet,\bullet})$.

Having obtained an ordinary cell complex from a double complex we can
now define the homology of it:
\begin{definition}[Homology of a Double Complex]
	For a given double complex $C^{\bullet,\bullet}$ we define the
	\underline{homology of the double complex} as the homology of the
	associated chain complex, in notation:
	\[
		H^\bullet(C^{\bullet,\bullet}) := H^\bullet(C^\bullet(C^{\bullet,\bullet}))
	\]
\end{definition}
While double complexes are interesting objects in their own right, we
will simply use them as a tool and therefore will impose some additional
restrictions. Henceforth we will always assume that double complexes have
only positive parts, that is $C^{p,k} = 0$ for $p < 0$ or $k <0$.

We are doing all these because we want to find a double complex
$C^{\bullet, \bullet}$ that can be augmented by chain complexes
$\Omega^\bullet, \hat{C}^\bullet$ such that the following diagram commutes:
\begin{center}
% https://tikzcd.yichuanshen.de/#N4Igdg9gJgpgziAXAbVABwnAlgFyxMJZARgBoAGAXVJADcBDAGwFcYkQAdD2qCHBAL6l0mXPkIoATBWp0mrdlx59BwkBmx4CRAMwyaDFm0SduvfiCEjN4ouVLFZhhSa4B5ALYwA5vQB6kpZqGmLaKGSOBvLGIADCfsD2kgJB1qESyNKRckbs8cBkyanqoloZetnOMfnSRVYlNmHIACwOTtHsAHTdxSFldqSS7bmuHJ4+-sS9pbbhg8MucQn2xCn1fbOZ81EjSwUOa8EzTXpDO4s1B9ONGa1nOYvdndfpAzoLMe5evn7kL-1zd7narLCiHNIArZAh4g-bkcENV4oPTQqp5BLSeH-TatVEdExPbFNMjND6KDgAC3oOGAsQEvyJGWkpOB5KpNLpfim62O5VILJhbOptPpgR5NwGVFZo2UFnFSJa-LJo3MglkMCg3ngRFAADMAE4QDxIPQgHAQJCtQWIMDMRiMGiMegAIxgjAACrz2IwYLqcMUDUbLTRzUh7NbbfbHS63Z6JSYfX6A4bjYgraHEABWaWRh0gJ2uj1ehO+-31QOp7Nmi2IMgRu15gux4v50vJoNZkM1gBsOYb0cLcaRraT5ZTSF71aQ0nrUfzMaL8ZHZbUFYnXaQAHY+3Om4vh4mV3rx4gABwbxAATh3jYXQ4By-bqevU7TN4HzaXh6fSBfGeI4Zoja-bzoOLbfmOHYARexB1kBuYfvuD4QauJ7Qa+VbwSBe73rMj6QamsEwTOWG7ne4FtgRSBEa+k6kbeYFfpRqFQaa-5WvRiG4WE+EsYR6Y1sQmH4ghoGfgezHHlBAm-u+YlIXhKFSYRVb-nRInYeRTGjnx1GqYJgEaWRjESTpynUZO-7brODHichkkgGutaWYJcFGbZCk8Upjloeer7EBx7lcRRZk+VBf6CcJIyiThIVHmFqaSOGalybF2nxU5xBsTWp5UWeF46Hl2XTqsum1n5-6XnlxARdO8JlUlF6SMklACEAA
\begin{tikzcd}
\vdots                       & \vdots                        & \vdots                        & \vdots                        &       \\
\Omega^2 \arrow[r] \arrow[u] & {C^{0,2}} \arrow[u] \arrow[r] & {C^{1,2}} \arrow[u] \arrow[r] & {C^{2,2}} \arrow[u] \arrow[r] & ...   \\
\Omega^1 \arrow[r] \arrow[u] & {C^{0,1}} \arrow[u] \arrow[r] & {C^{1,1}} \arrow[r] \arrow[u] & {C^{2,1}} \arrow[r] \arrow[u] & ...   \\
\Omega^0 \arrow[r] \arrow[u] & {C^{0,0}} \arrow[r] \arrow[u] & {C^{1,0}} \arrow[r] \arrow[u] & {C^{2,0}} \arrow[r] \arrow[u] & ...   \\
                             & \hat{C}^0 \arrow[u] \arrow[r] & \hat{C}^1 \arrow[u] \arrow[r] & \hat{C}^2 \arrow[u] \arrow[r] & \dots
\end{tikzcd}
\end{center}
And our hope would be to conclude from this that the homology of
$\Omega^\bullet$ and $\hat{C}^\bullet$ are isomorphic to the homology of
the double complex (and especially to each other). But to actually do this
we are going to require something more:
\begin{theorem}
\label{theorem:augmentation_iso_homology}
Let $C^{\bullet,\bullet}$ be a double complex, $(A^\bullet, \delta)$ a chain
complex.

If for every $n >= 0$ there is a map
\[
	\alpha^n: A^n \to C^{n,0}
\]
such that the following diagram commutes:
\begin{center}
% https://tikzcd.yichuanshen.de/#N4Igdg9gJgpgziAXAbVABwnAlgFyxMJZAJgBoBmAXVJADcBDAGwFcYkQBhAPWAAZTeAXxCDS6TLnyEU5CtTpNW7bsACMA4aPHY8BIgBY5NBizaJOPMkJFiQGHVKIBWIwtPsAOh6gQcCLXYSutIkpMTyJkrmKvyqmrb2knooZKoRimYWfGHx2kkhsmnGGco86sS5gQ7JyLLhxe7RZaRxNnnBBmHpjVmplYkdKIZFblG9OW1V+URW3WNetD5+kwOOMgJzmQtL-glBa8iGvJueHou+u+0HLvWjW94XK-s1LiOR9ztP1SHqbyXmAEEuMQvtMUOUToCuKpQYNkOoqA0xkDeLCDupjkj7udlgFVjVYpCQKi8c8Qvxbu92CS9t8iPxEXdqSJ5DAoABzeBEUAAMwAThAALZIfggHAQJDqJnmKA8fhCAD6AAtJvyhZKaOKkGRpSBZdk4srVQLhYgdVrEOQAmrTaKLYZdfr5YIFbRjerEA6LS5HXKWi63daTUgfRaAGxY9hOnJGoMer0SxAAdkjMr9hpVcdNKbFiYjvrU-td7tN+YtAE5U3rLEXM7YbUgcxaABxV-W-F113nBxCt3NISsFvrFrMa-vJtvNRWB+s982JvtUtPAKwBkshzWJ1RCWcestbuK702Di2qCpHpB909Wi+IE9b-SjxCqJtb-NLkBeWCMHD0Lg07sPVUfdJR9D8vxgH8-xhJ9VFDLdm1g+9QNg+CNzFegsEYdglQgCAAGsQCrLwmDQJU-xBJ9iDtN9YJAz1NUw7DzFwgiiN1EjGDI6D1zNKVTyTWDXxFRisJwvDCOIjxSPI-8WUEIA
\begin{tikzcd}
            & \vdots                                                & \vdots                                                      & \vdots                                                      & \vdots                                       &       \\
0 \arrow[r] & A^2 \arrow[u] \arrow[r, "\alpha^2", hook]             & {C^{0,2}} \arrow[r, "{d^{0,2}_h}"] \arrow[u]                & {C^{1,2}} \arrow[r, "{d^{2,1}_h}"] \arrow[u]                & {C^{2,2}} \arrow[u] \arrow[r]                & \dots \\
0 \arrow[r] & A^1 \arrow[u, "\delta^1"] \arrow[r, "\alpha^1", hook] & {C^{0,1}} \arrow[u, "{d^{0,1}_v}"] \arrow[r, "{d^{0,1}_h}"] & {C^{1,1}} \arrow[u, "{d^{1,1}_v}"] \arrow[r, "{d^{1,1}_h}"] & {C^{2,1}} \arrow[u, "{d^{2,1}_v}"] \arrow[r] & \dots \\
0 \arrow[r] & A^0 \arrow[u, "\delta^0"] \arrow[r, "\alpha^0", hook] & {C^{0,0}} \arrow[r, "{d^{0,0}_h}"] \arrow[u, "{d^{0,0}_v}"] & {C^{1,0}} \arrow[r, "{d^{0,1}_h}"] \arrow[u, "{d^{1,0}_v}"] & {C^{2,0}} \arrow[r] \arrow[u, "{d^{2,0}_v}"] & \dots
\end{tikzcd}
\end{center}
and every row is exact, then
\[
	H^\bullet(A^\bullet) \cong H^\bullet(C^{\bullet,\bullet})
\]
\end{theorem}
Before proving this we are going to need a convenient lemma:
\begin{lemma}
\label{lemma:helpfull}
Given a chain complex $C^{\bullet, \bullet}$ for which the sequence
\begin{center}
% https://tikzcd.yichuanshen.de/#N4Igdg9gJgpgziAXAbVABwnAlgFyxMJZABgBpiBdUkANwEMAbAVxiRAGEA9YMtAXxB9S6TLnyEUARnJVajFmy7Bp-QcJAZseAkQBMM6vWatEHbvtVCRW8UQDMBuccXcHl9ZrE6UAFkdGFUwAdIKgIHAQ+WRgoAHN4IlAAMwAnCABbJDIQHAgkSSsQVIz86lykXULizMR9HLzEOyq0mod6pB8oviA
\begin{tikzcd}
{C^{0,p}} \arrow[r] & {C^{1,p}} \arrow[r] & {C^{2,p}} \arrow[r] & {C^{3,p}} \arrow[r] & \dots
\end{tikzcd}
\end{center}
is exact for every $p$, then every homology class
$[c] \in H^n(C^{\bullet, \bullet})$ is homological to one represented by a
$\tilde{c} \in C^{0,n})$, i.e. $[c] = [\tilde{c}]$.
\end{lemma}
\begin{proof}
Let $[c] \in H^n(C^{\bullet, \bullet})$. We assume w.l.o.g. $c \in C^{j, n - j}$
for some $j > 0$ (for $j = 0$ we are finished anyway). By definition of homology
we know
\[
	c \in \ker(d^n) => c \in \ker(d^{j, n - j}_v + (-1)^j d^{j, n - j}_h)
\]
And since these lay in different parts of a direct sum, it can be concluded:
\[
	d^{j, n - j}_v(c) = d^{j, n - j}_h(c) = 0
\]
And by exactness of the rows, there is a $\tilde{c}$ such that
$d^{j - 1, n - j}_h(\tilde{c}) = c$. Then:
\[
	d^{n - 1}(\tilde{c}) = d^{j - 1, n - j}_v(\tilde{c}) + (-1)^{j - 1} c
\]
And therefore in homology:
\[
	[c] = [\underbrace{
		(-1)^{j} d^{j - 1, n - j}_v(\tilde{c})
	}_{\in C^{j - 1, n - j + 1}}]
\]
This can be iterated until finding a $c' \in C^{0, n}$ such that $[c] = [c']$.
\end{proof}
This gives us a clear path to proof it:
\begin{proof}[Proof of \ref{theorem:augmentation_iso_homology}]
$\alpha^\bullet$ is actually a chain map, since commutativity gives:
\[
	(\alpha^{n + 1} \circ \delta^n)
		= (\delta^n \circ \alpha^n)
\]
And by exactness for every $a \in A^n$:
\[
	(d^n \circ \alpha^n)(a) = d^{0,n}_v(\alpha(a)) + d^{0,n}_h(\alpha(a)) = d^{0,n}_v(a)
\]
Hence $\alpha^\bullet$ is a chain map and descends to a map on homology.

Proving that this is an isomorphism can be done quite directly, we start by
showing that $\alpha^n$ is injective: Let $[a] \in H^n(A^\bullet)$ such that
\[
	\alpha^n([a])
		= [\alpha^n(a)]
		= 0
\]
Then by definition of homology $\alpha^n(a) \in \Img(d^{n - 1})$,
therefore there is a $c \in C^{0, n - 1}$ such that:
\[
	a = d^{n - 1}(c) = d^{0, n - 1}_v(c) + d^{0, n - 1}_h(c)
\]
therefore also $c \in \ker(d^{0, n - 1}_h)$, which by exactness is
$\Img(\alpha^{n - 1})$, therefore there is a $\tilde{a}$ such that
$\alpha^{n - 1}(a) = c$. Putting this in a "commutative" diagram with
concrete elements instead of groups gives us:
\begin{center}
% https://tikzcd.yichuanshen.de/#N4Igdg9gJgpgziAXAbVABwnAlgFyxMJZABgBoBGAXVJADcBDAGwFcYkR6QBfU9TXfIRRli1Ok1bti3XiAzY8BIuQpiGLNohAAdbUzQALegAp6AShl8Fg5aQBMaiZpABjS3P6KhyFaJrrJLWkeKwElFDtVfycpd3kw70iHaI1YkI9rcJJ7R1StXTxGWGB6Lm4xGCgAc3giUAAzACcIAFskMhAcCCRydKbW9poupEjO+ixGdgMICABrd362xABmIe7EOz7mpdHhxAAWLYGNtaQAViOl1c71gDZLpAB2U5Wh8cmtabnyriA
\begin{tikzcd}
0                           & 0                             &   \\
a \arrow[u] \arrow[r, hook] & \alpha(a) \arrow[u] \arrow[r] & 0 \\
\tilde{a} \arrow[r, hook]   & c \arrow[u] \arrow[r]         & 0
\end{tikzcd}
\end{center}
Which shows that by commutativity
$(\alpha^n \circ \delta^{n - 1})(\tilde{a}) = \alpha^n(a)$, hence by
injectivity of $\alpha^n$ also $\delta^{n - 1}(\tilde{a}) = a$, 
implying $[a] = 0$ and especially that $\alpha^\bullet$ is injective.

Surjectivity of $\alpha^\bullet$ can now be proven by using Lemma
\ref{lemma:helpfull}. Let $[c] \in H^n(C^{\bullet, \bullet})$,
assume w.l.o.g. that $c \in C^{0,n}$. By definition of homology
$c \in \ker(d^n)$ and therefore $d^{0,n}_v(c) = d^{0,n}_h(c) = 0$, by exactness
there then is a $a \in A^n$ such that $\alpha^n(a) = c$ and by commutativity:
\[
	0 = (d^{0,n}_v \circ \alpha^n)(a) = (\alpha^{n + 1} \circ \delta^n)(a)
\]
and by injectivity of $\alpha^{n + 1}$ also $\delta^n(a)$, implying
$[a] \in H^n(A^\bullet)$ and especially:
\[
	\alpha^\bullet([a]) = [\alpha^\bullet(a)] = [c]
\]
Since $[c]$ was chosen arbitarily $\alpha^\bullet$ is therefore also
surjective.
\end{proof}

Everything we did here can be also done for a double complex with exact
coloumns instead of rows and augmented at the bottom instead of the left.
This requires some sign trickery, but leads to the final version of our
statement
\begin{corolarry}
Given two chain complexes $A^\bullet, B^\bullet$ a double complex
$C^{\bullet,\bullet}$ and maps $\alpha^\bullet, \beta^\bullet$ such that
\begin{center}
% https://tikzcd.yichuanshen.de/#N4Igdg9gJgpgziAXAbVABwnAlgFyxMJZAJgBoBGAXVJADcBDAGwFcYkQBhAPWAAZTiAXxCDS6TLnyEUAZgrU6TVu27BypNMNHjseAkQAs8mgxZtEnHmSEixIDLqlFrC08our+5LXYeT9KGQyrkrmlnykvD46-tLIcsQhZio86t62MXpxcsEmoSlqkdH2ElmGAknu4WTp2iWOAchGuYrJHlZFGfWxRACsxq1VADpDUBA4CHV+ZSj9iXltICNjE13TTrOkLW5hy+OTvqUbTZGVu0O0KweZx3K8Z+wjl-trR41k9wvDF1evDXFpB4WACCXGIfx6KHU80GYVB5AhM2Q6k+sMePxeUzeAK2QJAoN4iOOZAMeIAQlxCVj-kQ5KSvmEKQjqZCTvS0RYKeCWUj+uydujfjzjl48VTDjSUPwYQKLOKbo1+Nt8nKie9SL0xWrshqtcLGkZNQz2FSFDAoABzeBEUAAMwAThAALZIfggHAQJDeOwO51emgepBCH2Ol2IOTuz2IKIhv2IIyRpAyOq+sMRwOIXop0NIBMZgBs2bj+YDUazsbD-UTiG9dpziAA7KWg0WwwAOZuN1tIEvVtvdzOdhsDsjVgCcA6b1fIMbrcY709rIFTQc75GDc7D6mnyYrrrXBgH5CrGbdsqWQyYaAAFvQwSAA-QsIx2NeIBAANZdFc13sZ49HieUbkMOe41gu-6FmB5AQVGCbniMV63pSD7uk+L4WG+n7fvW5B-lGEYIZejA3neCKPs+r7vl+R5jp2xCzsu9YMfRS4-sQ24ZsQG5MXG5B0dW8EqheABGMA4HehIURhIBYTRYEsdWvZEWJElcORaGUZh1E4XGHGdguKniXe4LSVR2EjumwHlpuQZ5sBUG2YgxBAV6-YKfhXoTgpU5cYx7GwUG3iUIIQA
\begin{tikzcd}
            & \vdots                                    & \vdots                                   & \vdots                                   & \vdots                                   &       \\
0 \arrow[r] & A^2 \arrow[r, "\alpha^2", hook] \arrow[u] & {C^{0,2}} \arrow[r] \arrow[u]            & {C^{1,p}} \arrow[r] \arrow[u]            & {C^{2,2}} \arrow[r] \arrow[u]            & \dots \\
0 \arrow[r] & A^1 \arrow[u] \arrow[r, "\alpha^1", hook] & {C^{0,1}} \arrow[u] \arrow[r]            & {C^{1,1}} \arrow[u] \arrow[r]            & {C^{2,1}} \arrow[u] \arrow[r]            & \dots \\
0 \arrow[r] & A^0 \arrow[u] \arrow[r, "\alpha^0", hook] & {C^{0,0}} \arrow[u] \arrow[r]            & {C^{1,0}} \arrow[u] \arrow[r]            & {C^{2,0}} \arrow[u] \arrow[r]            & \dots \\
            &                                           & B^0 \arrow[r] \arrow[u, "\beta^0", hook] & B^1 \arrow[r] \arrow[u, "\beta^1", hook] & B^2 \arrow[r] \arrow[u, "\beta^2", hook] & \dots \\
            &                                           & 0 \arrow[u]                              & 0 \arrow[u]                              & 0 \arrow[u]                              &      
\end{tikzcd}
\end{center}
commutes with every row and coloum except for the left- and bottommost
being exact, then:
\[
	H^\bullet(A^\bullet)
		\cong H^\bullet(C^{\bullet, \bullet})	
		\cong H^\bullet(B^\bullet)
\]

\end{corolarry}
\begin{proof}
As said this follows from \ref{theorem:augmentation_iso_homology}
using symmetry and sign trickery.
\end{proof}


\section{Mayer-Vietoris to the Infinite}
It remained to show that the rows of the double complex are exact.
\begin{theorem}[The Generalized Mayer-Vietoris Sequence]
The sequence
% https://q.uiver.app/#q=WzAsNSxbMSwwLCJcXE9tZWdhXiooTSkiXSxbMCwwLCIwIl0sWzIsMCwiXFxPbWVnYV4qKFxcY29wcm9kX3tcXGFscGhhXzAgPCBcXGFscGhhXzF9IFxcbWF0aGNhbHtVfV97XFxhbHBoYV8wLCBcXGFscGhhXzF9KSJdLFszLDAsIlxcT21lZ2FeKihcXGNvcHJvZF97XFxhbHBoYV8wIDwgXFxhbHBoYV8xIDwgXFxhbHBoYV8yfSBcXG1hdGhjYWx7VX1fe1xcYWxwaGFfMCwgXFxhbHBoYV8xLCBcXGFscGhhXzJ9KSJdLFs0LDAsIlxcZG90cyJdLFsxLDBdLFswLDJdLFsyLDNdLFszLDRdXQ==
\[\begin{tikzcd}[column sep=small]
	0 & {\Omega^*(M)} & {\Omega^*(\coprod_{\alpha_0 < \alpha_1} \mathcal{U}_{\alpha_0, \alpha_1})} & {\Omega^*(\coprod_{\alpha_0 < \alpha_1 < \alpha_2} \mathcal{U}_{\alpha_0, \alpha_1, \alpha_2})} & \dots
	\arrow[from=1-1, to=1-2]
	\arrow[from=1-2, to=1-3]
	\arrow[from=1-3, to=1-4]
	\arrow[from=1-4, to=1-5]
\end{tikzcd}\]
is exact.
\end{theorem}
\begin{proof}
We already proved that $\delta^2 = 0$, hence it is enough to find a map $r$ that is a
left inverse to $\delta$ on the kernel.

Let $\omega \in \Omega^*(\coprod_{\alpha_0 < \dots < \alpha_k} \mathcal{U}_{\alpha_0, \dots, \alpha_k})$.
Let $\{\rho_\alpha\}$ be a partition of unity subordinate to the cover $\{\mathcal{U}_\alpha\}$.

Let $\omega$ be a closed $k$-form on $\coprod \mathcal{U}_{\alpha_0, \dots, \alpha_p}$. Then we define
a $k$-form $\tau$ on $\coprod \mathcal{U}_{\alpha_0, \dots, \alpha_{p - 1}}$ by:
\[
	\tau_{\alpha_0, \dots, \alpha_{p - 1}} = \sum_{\beta \in I} \rho_\beta \omega_{\beta, \alpha_0, \dots, \alpha_{p - 1}}
\]


\end{proof}

\section{Explicit Isomorphism \& Applications}

Let us take a good cover $\mathcal{U}_\alpha, \mathcal{U}_\beta, \mathcal{U}_\gamma$ of $S^1$. The nerve of
this cover is shown in TODO, together with a choice of orientation.
\tikzset{
  % style to apply some styles to each segment of a path
  on each segment/.style={
    decorate,
    decoration={
      show path construction,
      moveto code={},
      lineto code={
        \path [#1]
        (\tikzinputsegmentfirst) -- (\tikzinputsegmentlast);
      },
      curveto code={
        \path [#1] (\tikzinputsegmentfirst)
        .. controls
        (\tikzinputsegmentsupporta) and (\tikzinputsegmentsupportb)
        ..
        (\tikzinputsegmentlast);
      },
      closepath code={
        \path [#1]
        (\tikzinputsegmentfirst) -- (\tikzinputsegmentlast);
      },
    },
  },
  % style to add an arrow in the middle of a path
  mid arrow/.style={postaction={decorate,decoration={
        markings,
        mark=at position .5 with {\pgftransformscale{2}\arrow[#1]{stealth}}
      }}},
}
\begin{figure}
\begin{tikzpicture}
    \tikzstyle{point}=[thick,draw=black,inner sep=0pt]
    \node[label={[below left]$\alpha$}] (a) at (0,0) {};
    \node[label={[below right]$\beta$}] (b) at (5,0) {};
    \node[label={[above]$\gamma$}] (c) at (2.5,4.33) {};
	% yshift=-10pt
	% xshift=7pt, yshift=7pt
	% xshift=-7pt, yshift=7pt
    \draw[postaction={on each segment={mid arrow}}]
		(a.center) -- node[midway, label={[label distance=10pt]270:$c$}] {}
		(b.center) -- node[midway, label={[label distance=10pt]0:$a$}] {}
		(c.center) -- node[midway, label={[label distance=10pt]180:$b$}] {}
		cycle;
	\foreach \n in {a,b,c}
	{
		\fill (\n.center) circle (1.5pt);
	}
\end{tikzpicture}
\end{figure}
The Čech cochains of this complex are:
\begin{align*}
	C^0(K) = \langle \alpha, \beta, \gamma \rangle, \,
	C^1(K) = \langle a, b, c \rangle
\end{align*}
With the coboundary operator being defined as:
\begin{align*}
	\delta \alpha = b - c,\quad 
	\delta \beta = c - a,\quad
	\delta \gamma = a - b
\end{align*}
We already calculated the cohomology of this, its $\hat{H}^0(K) = \hat{H}^1(K) = \RR$,
with the 1-cohomology being generated by $[\alpha] = [\beta] = [\gamma]$. We can 
