\chapter{Čech Cohomology}
One of the most noticable invariants of a topological space is the number of its connected components.
While being intuitivle understandable, it is also limited: There are a lot of topological spaces that are
widely different, but have the same number of connected components. This limitation can be in a certain sense
traced back to connectedness being a "low-dimensional" invariant, best seen by trying to prove a seemingly obvious statement
\begin{theorem}
	Let $n \neq m \in \NN$. $\RR^n$ is not homeomorphic to $\RR^m$
\end{theorem}
We can prove this in the case of $n = 1, m > 1$ as follows: Take any $p \in \RR$, then $\RR \setminus \{p\}$ has
two connected components, but $\RR^m \setminus \{q\}$ is connected for any $q \in \RR^m$, hence the spaces can't
be homeomorphic. For distinguishing the case of $n = 2, m > 2$ this argument fails – in that sense connectedness
is a lower dimensional phenomonen.

Various higher dimensional generalizations of connected components have been developed
that overcome this limitation by measuring higher dimensional phenomena. Of these generalizations
Cohomology is arguably the easiest to compute and understand. While there are in general multiple different
ways to define cohomology, on "nice" spaces (which includes all smooth manifolds) all of them agree, allowing
us to restrict our attention to just one: Čech Cohomology. It tries to reduce the geometric structure of our
usually uncountable spaces back to finite combinatorial objects, whose properties are easily calculated.

\begin{remark}
One would normally state most of the definitions in this chapter in
the context of modules of a ring $R$, getting vector spaces as a special
case (since vector spaces are exactly the modules of their field),
but since de Rham cohomology only makes sense as a vector space this
would only add needlees abstraction \footnote{
	One could also do everything in the context of abelian categories,
	the most general setting for homological algebra
	% citation needed 
	(and some argue the most natural), but this would be even more of an
	overkill.
}
\end{remark}

\section{Simplicial cohomology \& nerves}
We want to approximate topological spaces with simplicial complexes, a combinatorial
object:
\begin{definition}
We call $K = (V, S)$ for sets $V,S$ a \underline{simplicial complex} with
\underline{vertices} $V$ and \underline{simplices} $S$ if the elements of $S$
are non-empty subsets of $V$, calling $\sigma \in S$ a $n$-simplex of $K$ if
it has $n + 1$ elements and if they fullfill the following conditions:
\begin{enumerate}
	\item For every vertex $v \in V$ there is a 0-simplex $\{v\} \in S$
	\item For every $\sigma \in S$ all subsets $\sigma' \subseteq \sigma$
		are also in $S$.
\end{enumerate}
\end{definition}
Even though this definition is quite technical, it has an intuitive counterpart:
One can imagine the vertices/0-simplices as points, the 1-simplices as lines between
their elements seen as 0-simplices, the 2-simplices as filled triangles, with
every 2 element subset of them seen as their boundary lines etc.

% TODO: Define topology

The classical way to assign a simplicial complex to a topological space was by triangulation,
that is one would search a simplicial complex, whose topology would be homeomorphic to
the original space. While this method does have inutive appeal it also has weaknesses,
a major one being that not all topological spaces have one (not even all topological
manifolds).

This lead to the development the approximation of a space as a simplicial complex using
open covers:
\begin{definition}
	The \underline{nerve} $\mathfrak{N}(\mathfrak{U}, X)$ of an open cover
	$\mathfrak{U}$ of a space $X$ is the simplicial complex $K$ with
	vertices $V = \mathfrak{U}$ and simplices:
	\[
		S = \bigcup_{n = 0}^\infty
			\{\,
				\{ U_0, \dots, U_n \}
				\mid
				U_0, \dots, U_n \in \mathfrak{U}:
				U_0 \cap \dots \cap U_n \neq \emptyset
			\,\}
	\]
	We will denote the $n$-simplices of this nerve by $\mathfrak{N}^n(\mathfrak{U}, X)$.
\end{definition}
That is: We take the open sets as vertices, with two intersecting sets being lines, three intersecting
set being 2-simplices and in general $n + 1$ intersecting sets being $n$-simplices.

To make this more algebraicly interesting we turn it into a graded vector space structure:
\begin{align*}
	C^n(\mathfrak{N}(\mathfrak{U}, X)) \coloneqq \bigoplus_{\sigma \in \mathfrak{N}^n(\mathfrak{U}, X)} \RR
\end{align*}
% TODO: To this better, especially since this is the first of many uses of such a convention
We will also establish a convention to refer its elements. For $\alpha_0, \dots, \alpha_n \in I$ we 
denote the element representing the intersection of $\U_{\alpha_0}, \dots, \U_{\alpha_n}$ by
$e_{\alpha_0, \dots, \alpha_n}$ (in the case of an empty intersection we will treat it as being zero).

We will also use this to introduce the concept of orientation to simplicial complexes by defining:
\begin{align*}
	e_{\alpha_{\sigma(0)}, \dots, \alpha_{\sigma(n)}} = \sign(\sigma) e_{\alpha_0, \dots, \alpha_n}
\end{align*}
which also defines elements with any index repetition as zero.

This allows us to enrichen the structure on this even further by introducing a boundary operator,
encapsulating the idea of mapping an oriented $n$-simplex to its oriented boundary of $(n - 1)$-simplices.
Given any $n$-simplex $e_{\alpha_0, \dots, \alpha_n}$ we define its boundary:
\begin{align*}
	\partial e_{\alpha_0, \dots, \alpha_n} \coloneqq \sum_{i = 0}^n (-1)^n e_{\alpha_0, \dots, \hat{\alpha}_i, \dots, \alpha_n}
\end{align*}
where $\hat{\alpha}_i$ denotes the removal of the $i$th index.

\begin{example}
The boundary of $e_{a, b}$ is $b - a$, this can be interpreted as $e_{a, b}$ being the line from $a$ to $b$
and $e_{b, a}$ being the line from $b$ to $a$.

The triangle $e_{a,b,c}$ has the boundary $e_{b, c} - e_{a, c} + e_{a, b} = e_{a,b} + e_{b, c} + e_{c, a}$.
This can be interpreted as its choice of orientation being whether to go clockwise or counter-clockwise around it.
\end{example}

This map is obviously a linear map on $C_*(\mathfrak{N}(\mathfrak{U}, X))$ and it is nil-potent, that is
$\partial \circ \partial = 0$. Intuitively that is because the boundary of a boundary is empty, but one
can also do the algebraic calculation of $\partial^2$ to see that the signs all cancel each other.

This is now an example of a more general algebraic concept:

\begin{definition}
	A \underline{Chain complex} $(C_\bullet, \partial_\bullet)$ is a $\ZZ$-graded
	vector space $C_\bullet$ together with a map
	$\partial_\bullet: C_\bullet \to C_\bullet$ (called \underline{boundary map})
	whose restriction to $C_n$ (which we will denote by $\partial_n$) maps to
	$C_{n - 1}$ and which is nilpotent, i.e. $\partial_\bullet^2 = 0$.
\end{definition}
In abuse of notation we will denote this by writing
$\partial: C_\bullet \to C_{\bullet - 1}$.

To "detect holes" we are now going to take inspiration from the definition of
the fundamental group.
% TODO: Expand on this
There one is interested in non-contractible loops, that
is: maps from $S^1$ into the space that cannot be extended to $D^2$. Putting
this into a more general context: One is interested in loops that aren't the boundary
of anything. Taking a loop to mean everything that does not have boundary gives us
the algebraic definition:
\begin{align*}
	\ker \partial_n / \Img \partial_{n + 1}
\end{align*}
Following this reasoning we are going to call the elements of $\Img(\partial_\bullet)$
\underline{boundaries} and the elements of $\ker(\partial_\bullet)$ \underline{cycles}.

Doing this procedure for a chain complex leads us to the namesake of homological algebra:
\begin{definition}
The \underline{Homology} of a chain complex $(C_\bullet, \partial)$ is the
$\ZZ$-graded vector space:
\[	
	H_n(C_\bullet, \partial_\bullet) = \ker(\partial_n)/\Img(\partial_{n - 1})
\]
\end{definition}

All of this reminds us of the way we defined de Rham cohomology, but not quite:
The exterior derivative goes in the "opposite direction", i.e. it maps $n$-forms
to $n+1$ forms and we put "co" in front of everything.

Because we want to proof a relationship to de Rham cohomology \footnote{
	There are also other reasons to prefer cohomology. Similiar to de Rham
	cohomology it is possible to define a product structure on it, turning
	it into an algebra (and
	there are spaces which cohomologies are isomorphic as vector spaces,
	but not as algebras). Also Čech cohomology has certain "good" properties
	that Čech homology has not because of category theoretic reasons, see
	Lecture 46 of \cite{wendl_topology_2024} for more detail
 }
we have to "reverse the direction" of Čech-Homology in a process called
dualisation:
\begin{definition}
The \underline{Cochain Complex} $(C^\bullet, \delta^\bullet)$ of a 
chain complex $(C_\bullet, \partial_\bullet)$ is its dualization:
\begin{align}
	C^n \coloneq \{\, f: C_n \to \RR \text{ linear}  \,\}\\
	\delta^n: C^n \to C^{n + 1} \\
	f \mapsto (f \circ \partial_{n + 1})
\end{align}
% \label{def:cochain_complex}
\end{definition}
\begin{remark}
Note that this is the so called algebraic dual since we do not require
boundedness/continouity of the linear maps.
\end{remark}
% Finally arriving at:
\begin{definition}
The \underline{Cohomology} of a cochain complex
$(C^\bullet, \delta^\bullet)$ is the $\ZZ$-graded vector space:
\[
	H^n(C^\bullet, \delta^\bullet) := \ker(\delta^n)/\Img(\delta^{n - 1})
\]
\end{definition}
Dual to our previous remark we are going to denote the elements of $\Img(\delta)$ as coboundaries
and the elements of $\ker(\delta)$ as cocycles.

Combining all of this leads us to finally define the Čech-cohomology of an open cover:
\begin{definition}
Let $X$ be a topological space with open cover $\mathfrak{U}$. We define \textbf{the Čech-cohomology of
$\mathfrak{U}$} as:
\begin{align*}
	\hat{C}^\bullet(\mathfrak{U}, X) \coloneqq H^\bullet(C^\bullet(\mathfrak{U}, X))
\end{align*}
\end{definition}

% After getting this out of the way, we can proceed to define the precursor to
% Čech-Cohomology, simplicial cohomology. This cohomology won't be defined on
% all spaces, but only on so called simplicial complexes, a foremost combinatorial
% object:
% %-TODO: Find reference for this definition!

% This abstract combinatorial definition has an intuitive interpretation by
% taking  elements of $V$ as discrete points with every 2-element set
% $\{a,b\} \subseteq S$ being a line connecting $a$ and $b$, every 3-element
% set $\{a,b,c\} \subseteq S$ being a filled triangle connecting the lines
% $\{a,b\}, \{a,c\}, \{c,b\}$, etc...

\begin{figure}[h]
\centering
\includegraphics{graphics/simplicial_complex2.pdf}
\caption{A triangle as a simplicial complex}
\label{fig:triangle_simplicial_complex}
\end{figure}

% One could make this more precise and assign a topological space
% to every simplicial complex, but we are not going to do that
% since we only need this for motivation and because this topology
% agrees with the obvious one for finite simplicial complexes and 
% is nontrivial to define for infinite ones.
% TODO: Add reference to rigorous definition

% The motivating reason behind the definition of chain complexes was to
% link the $n$-dimensional elements of a space to a vector space. Here a
% identification with the $n$-simplices seems like the obvious choice and the
% easiest vector space one could imagine would just be the direct sum of a copy
% of $\RR$ for every one, i.e.:
% \[
% 	\tilde{C}_n^\Delta
% 		= \bigoplus_{
% 		\substack{
% 			\sigma \subseteq S \\
% 			|\sigma| = n + 1
% 		}} \RR
% \]
% , but defining the boundary operator then becomes
% difficult, especially since the naive definition of just mapping a simplex
% to the direct sum of its boundary faces does not work: This operator would
% not be nilpotent and the boundary of something like $\{a,b\} + \{b,c\}$ would be
% $a + 2b + c$ since the inner points would not cancel each other out.

% To solve this we are going to get inspired by de Rham cohomology and give
% the simplices an orientation by referring to ordered $n + 1$-tuple
% $(v_0, \dots, v_n)$ instead of unordered $n + 1$-element sets and requiring:
% \[
% 	(v_0, \dots, v_i, v_{i + 1}, \dots, v_n)
% 	\sim
% 	-(v_0, \dots, v_{i + 1}, v_i, \dots, v_n)
% \]

% TODO: shorter and more precise

% \begin{definition}
% Given a simplicial complex $K = (V, S)$, we define its \underline{simplicial
% chain complex} as the $\ZZ$-graded abelian group $C^\Delta_\bullet(K)$ together
% with the boundary operator $\partial_\bullet: C^\Delta_\bullet(K) \to C^\Delta_{\bullet - 1}(K)$.
% \end{definition}
% \begin{lemma}
% The simplicial chain complex is a chain complex. \qed
% \end{lemma}
% From an algebraic perspective this leads us to define:
% \begin{definition}
% Given a simplicial complex $K = (V, S)$, we define its \underline{simplicial
% homology} as the homology of its simplicial chain complex, denoted by:
% \[
% 	H_n^\Delta(K) \coloneq H_n(C_\bullet(K))
% \]
% \end{definition}
% TODO: Introduction of cochain complex/cohomology and this inconsistent.
% \begin{definition}
% Given a simplicial complex $K = (V, S)$ we define its \underline{simplicial
% cohomology} as the cohomology of its simplicial chain complex, denoted by:
% \[
% 	H^n_\Delta(K) \coloneqq H^n(C^\Delta_\bullet(K))	
% \]
% \end{definition}
% As we already briefly mentioned, one can associate a topological space
% to every simplicial complex. The reverse of this, associating a simplicial
% complex to a topological space, is called triangulation. This is neither unique
% nor always possible, but one can show that the homology of two triangulations
% $K, K'$ of two spaces $X, X'$ are naturally
% TODO: Should we say something about this word
% TODO: What about homotopy equivalent
% isomorphic if $X$ and $X'$ are homeomorphic, therefore the simplicial homology
% of two different triangulations of a space $X$ does not depend on the choice
% of triangulation (up to a natural isomorphism) allowing one to speak of
% {the} simplicial homology of a triangulable topological space, denoted by 
% $H^\Delta_\bullet(X)$. We are not going to prove any of this, since there is
% an alternative way to define a Cohomology for any (not just trianguable)
% topological space called Čech-Cohomology, by associating simplicial complexes
% to open covers
% That is, the vertices of a nerve are the open sets of the cover and the
% $n$-simplices are the nonempty intersections of $n$ sets. Since open covers are
% taken to only contain nonempty sets
% TOO: Define this somehow
% and if $U_1 \cap U_2 \cap U_3 \neq \emptyset$ then
% $U_1 \cap U_2 \neq \emptyset$, therefore the nerve is always actually a
% simplicial complex. This provides the straightforward definition:
% \begin{definition}
% The \underline{Čech-Cohomology} of an open cover $\mathfrak{U}$ of a space $X$ is the 
% simplicial cohomology of the nerve of the cover:
% \[
% 	\check{C}^\bullet(X, \mathfrak{U})
% 		\coloneq C^\bullet_\Delta(\mathfrak{N}(X, \mathfrak{U}))
% \]
% \end{definition}

\begin{figure}
	\centering
	\tcbsetforeverylayer{enhanced, colback=white, frame hidden}
	\begin{tcolorbox}[segmentation style={solid}]
	    \begin{tcbraster}[raster columns=2]
		    \tcbincludegraphics[height=6cm]{graphics/circle_coarse_cover.pdf}
			\tcbincludegraphics[height=6cm]{graphics/circle_coarse_nerve.pdf}
		\end{tcbraster}
		\tcblower
		\begin{tcbraster}[raster columns=2]
		    \tcbincludegraphics[height=6cm]{graphics/circle_fine_cover.pdf}
		    \tcbincludegraphics[height=6cm]{graphics/circle_fine_nerve.pdf}
	    \end{tcbraster}
	\end{tcolorbox}
	\label{figure:circle_nerves}
	\caption{A coarse and a fine open cover of $S^1$ and their nerves}
\end{figure}

\section{Čech-Cohomology as a limit}
% TODO: Another advantage would be good
We are almost finished with the definition of Čech-cohomology, but there remains one issue that we
swept a bit under the rug: So far we did not define the Čech-cohomology of a space, only of an open
cover of a space.

And for good reason: Not every open cover of a space produces a reasonable approximation. In fact: every
space can be covered by just itself, with the resulting nerve being one point, resulting in a cohomology
% TODO: Something is weird with the numbering here
that is trivial in every dimension but 0. For a less contrived example see Figure \ref{figure:circle_nerves}:
$S^1$ can be covered by two overlapping open sets, which produces a nerve that consists only of a 1-simplex –
not a very circle-like shape.

And this makes sense conceptually: We talked about taking covers as approximations of spaces
and these are bad approximations. The more interesting question is when we can talk about one approximation being better
than another and whether there is always going to be a "good" approximation. 

While it is not always possible to talk about one cover being better than another, it is sometimes possible:
\begin{definition}
An open cover $\mathfrak{U}$ is a \underline{refinement} of another open cover
$\mathfrak{V}$ if every open set $U \in \mathfrak{U}$ is contained in some
$U \subseteq V \in \mathfrak{V}$
\end{definition}
For this concept to be usefull we have to find a relation between the Čech-cohomologies of an open cover
and its refinement. We will do this by defining a map on the chain complexes: Given a
refinement $\mathfrak{V}$ with index set $J$ of a cover $\mathfrak{U}$ with index set $I$ we
can define a (non-canonical) map
$\lambda: J \to I$ that fullfills the relation:
\begin{align*}
	\V_\alpha \subseteq \U_{\lambda(\alpha)}
\end{align*}
We can use this to define a map between the cochain complexes:
\begin{align*}
	i: C^n(\mathfrak{U}, X) &\to C^n(\mathfrak{V}, X) \\
		c &\mapsto i(c)
\end{align*}
as:
\begin{align*}
	(i(c))(e_{\alpha_0, \dots, \alpha_n}) = c(e_{\lambda(\alpha_0), \dots, \lambda(\alpha_n)})
\end{align*}
Before using this to define a map on cohomology we are going to establish a general
condition for a map on cochains to descend to a map on cohomology. 
\begin{definition}
A \underline{cochain map} $f: C^\bullet \to D^\bullet$ is a map between two
cochain complexes
$(C^\bullet, \delta_C^\bullet), (D^\bullet, \delta_D^\bullet)$ that maps
$C^n$ to $D^n$ and commutes with the boundary operators, i.e.
\[
	\delta_D^\bullet \circ \phi^\bullet
	=
	\phi^\bullet \circ \delta_C^\bullet
\]
\end{definition}
This condition can be equally stated as requiring the diagram
% https://q.uiver.app/#q=WzAsNCxbMCwwLCJDXlxcYnVsbGV0Il0sWzAsMSwiRF5cXGJ1bGxldCJdLFsxLDAsIkNee3tcXGJ1bGxldH0gKyAxfSJdLFsxLDEsIkRee3tcXGJ1bGxldH0gKyAxfSJdLFswLDEsImYiLDJdLFswLDIsIlxcbWF0aHJte2R9X0FeXFxidWxsZXQiXSxbMiwzLCJmIl0sWzEsMywiXFxtYXRocm17ZH1fQl5cXGJ1bGxldCIsMl1d
\[\begin{tikzcd}[cramped]
	{C^\bullet} & {C^{{\bullet} + 1}} \\
	{D^\bullet} & {D^{{\bullet} + 1}}
	\arrow["{\mathrm{d}_A^\bullet}", from=1-1, to=1-2]
	\arrow["f"', from=1-1, to=2-1]
	\arrow["f", from=1-2, to=2-2]
	\arrow["{\mathrm{d}_B^\bullet}"', from=2-1, to=2-2]
\end{tikzcd}\]
to commute. Chain maps do descend unto cohomology, because:
\begin{lemma}
Every cochain map $\phi: C^\bullet \to D^\bullet$ descends to cohomology,
which we will denote by
\[	
	\phi^\bullet: H^\bullet(C^\bullet) \to H^\bullet(D^\bullet)
\]
\end{lemma}
\begin{proof}
Proving this is equal to proving $\phi(\Img(\delta_C^\bullet)) \subseteq \Img(\delta_D^\bullet)$.
Let $c \in C^\bullet$, then by the chain map condition:
\begin{align*}
	\phi(\delta_C^\bullet(c)) = \delta_D^\bullet(\phi(c)) & \qedhere
\end{align*}
\end{proof}

Our map $i$ fullfills this condition since:
\begin{align*}
	\delta i(c)(e_{\alpha_0, \dots, \alpha_n})
		= \delta c(e_{\lambda(\alpha_0), \dots, \lambda(\alpha_n)})
		&= c(\partial e_{\lambda(\alpha_0), \dots, \lambda(\alpha_n)}) \\
		&= \sum_{j = 0}^n (-1)^j c(e_{\lambda(\alpha_0), \dots, \hat{\lambda(\alpha_j)}, \dots, \lambda(\alpha_n)}) \\
		&= i\left(\sum_{j = 0}^n (-1)^j c(e_{\alpha_0, \dots, \hat{\alpha}_j, \dots, \alpha_n}) \right) \\
		&= i(\delta c) (e_{\alpha_0, \dots, \alpha_n})
\end{align*}
% TODO: Hat looks weird here
Hence it descends to a map $i^*$ on cohomology.

\begin{remark}
The attentive reader might have already noticed it, but the definition of $i$ that we give here
is far frome unique. Depending on the refinement there might be a lot of choice involved
in selecting $\lambda(\alpha)$ for alla $\alpha \in I$. But we do not have to worry about
this since the map induced on cohomology is independent of this chain level choices.
But we are not going to prove this, since this is rather annoying to do without providing much insight.
If interested, one can look at a proof in TODO
\end{remark}

For a fixed topological space $X$ we can now define $\mathcal{O}(X)$ as the set of all open
covers of $X$ and equip it with a relation:
\begin{align*}
	\mathfrak{U} \leqslant \mathfrak{V}
		\iff
	\mathfrak{V} \text{ is a refinement of } \mathfrak{U}
\end{align*}
While two open covers don't have to be refinements of each other, there is always going
to be another open cover that is a refinement of both. Thus $(\mathcal{O}(X), \leqslant)$ 
is a directed set. Indexing the set of cohomologies of nerves of all open covers of $X$
by this directed set (together with the linear maps we established above) turns it into a
richer structure:
\begin{definition}
A \textbf{directed system} is a family of vector spaces $\{A\}_{\alpha \in I}$ indexed by a
directed set $\langle I, \leqslant \rangle$ together with a morphism
\begin{align*}
	f_{\alpha \beta}: A_\alpha \to A_\beta
\end{align*}
for every $\alpha \leqslant \beta$ that fullfill:
\begin{itemize}
	\item $f_{\alpha \alpha}$ is the identity on $A_\alpha$
	\item For every $\alpha \leqslant \beta \leqslant \gamma$:
		\begin{align*}
			f_{\alpha \gamma} = f_{\beta \gamma} \circ f_{\alpha \beta}
		\end{align*}
\end{itemize}
\end{definition}
One can imagine a directed system as a tree-like structure with infinite leaves:
% https://q.uiver.app/#q=WzAsMTEsWzAsMCwiQV97XFxhbHBoYX0iXSxbMSwxLCJBX1xcZ2FtbWEiXSxbMCwxLCJBX1xcYmV0YSJdLFswLDIsIkFfXFxkZWx0YSJdLFsxLDIsIkFfXFx6ZXRhIl0sWzAsMywiQV9cXGVwc2lsb24iXSxbMiwyLCJBX1xcZXRhIl0sWzMsMiwiXFxkb3RzIl0sWzEsMCwiXFx2ZG90cyJdLFsxLDMsIlxcaWRkb3RzIl0sWzIsMSwiXFx2ZG90cyJdLFswLDFdLFsyLDFdLFszLDRdLFs1LDRdLFsxLDZdLFs0LDZdLFs2LDddLFs4LDFdLFs5LDZdLFsxMCw2XV0=
\[\begin{tikzcd}[column sep=small]
	{A_{\alpha}} & \vdots \\
	{A_\beta} & {A_\gamma} & \vdots \\
	{A_\delta} & {A_\zeta} & {A_\eta} & \dots \\
	{A_\epsilon} & \iddots
	\arrow[from=1-1, to=2-2]
	\arrow[from=1-2, to=2-2]
	\arrow[from=2-1, to=2-2]
	\arrow[from=2-2, to=3-3]
	\arrow[from=2-3, to=3-3]
	\arrow[from=3-1, to=3-2]
	\arrow[from=3-2, to=3-3]
	\arrow[from=3-3, to=3-4]
	\arrow[from=4-1, to=3-2]
	\arrow[from=4-2, to=3-3]
\end{tikzcd}\]
Now a natural question would be whether this tree has any "root". While there in general
isn't any such root in the tree itself there may be candidates that we can add to the
system. We do this by first defining possible candidates:
\begin{definition}[Target]
	Given a directed system $\{A_\alpha\}_{\alpha \in I}$ of vector spaces we say
	that a vector space $B$ together with morphisms $\{\phi_\alpha\}_{\alpha \in I}$:
	\[
		\phi_\alpha: A_\alpha \to B	
	\]
	is called \underline{Target} of the directed system if for every $\alpha \in I$ 
	the diagram

	TODO

	commutes.
\end{definition}
Note that by abuse of notation we will sometimes refer to a target as just a vector space
$B$ while eliding the morphisms if they are clear from context.

While this definition is already usefull it is too broad. Every vector space together with
the trivial morphism for every vector space in a system is the target of every direct system of vector spaces.
Wanting to include the information of all possible targets into it, we define:
\begin{definition}[Direct Limit]
Given a direct system $\{A_\alpha\}_{\alpha \in I}$ of vector spaces we call a target
$B$ together with morphisms $\{\phi_\alpha\}_{\alpha \in I}$ a direct limit of the system
if for every other target $C$ together with morphisms $\{\psi_\alpha\}_{\psi \in I}$ there
is a unique morphism TODO such that the following diagram commutes:

TODO
\end{definition}
The term "limit" implies some kind of uniqueness and indeed
\begin{lemma}
Given two limits of a direct system there exists a natural isomorphisms between them.
\end{lemma}
\begin{proof}
TODO
\end{proof}
Because of this we are going to talk about \textbf{the} limit of a directed system
and denote it by
\[
	\varinjlim \{A_\alpha\}_{\alpha \in I}
\]

Using this we can define the Čech cohomology of a space:
\begin{definition}[Čech Cohomology]
Let $X$ be a topological space. We define its \textbf{Čech cohomology} as the direct limit
of the Čech cohomology of its open cover, denoted by
\[
	\hat{H}^\bullet(X) \coloneqq \varinjlim \{\hat{H}(X, \mathfrak{U})\}_{\mathfrak{U} \in \mathcal{O}(X)}
\]
\end{definition}

\begin{remark}
If one looks too close at this definition, one notices that it does not really define the
Čech cohomology of a space as a vector space, but rather as an equivalence class
of naturally isomorphic vector spaces. This is more a philosophical than a practical issue though
since the natural isomorphisms allow us to just take a specific representative as "the" Čech cohomology.
\end{remark}

While this definition is quite elegant, it does suffer from the problem of being impractical.
Given a topological space we would have to calculate the Čech cohomology of all its open covers and
then apply the limiting operation – in most cases this is all but impossible.

But there is a solution to this problem. It is actually enough to calculate the Čech-cohomology of
a specific subset of open covers, if it fullfills some conditions. To formulate these we first define:
\begin{definition}
A subset $I_0 \subseteq I$ of a directed set $I$ is called \textbf{cofinal} if for every $\alpha \in I$
there is a $\beta \in I_0$ such that $\beta \geqslant \alpha$
\end{definition}

Then we can put this into a lemma:
\begin{lemma}
	Let $\{X_\alpha\}_{\alpha \in I}$ be a direct system with morphism $\phi_{\alpha \beta}$.

	If $I_0 \subseteq I$ is a cofinal set such that $\phi_{\alpha, \beta}$ is an isomorphism for every
	$\alpha, \beta \in I_0, \alpha \leqslant \beta$ than $\injlim \{A_\alpha\}_{\alpha \in I}$ is isomorphic
	to every $X_\gamma$ for $\gamma \in I_0$.
\end{lemma}

\begin{proof}
Pick $\alpha_0 \in I_0$. For any $\alpha \in I$ there is a $\beta \in I$ such that $\alpha_0 \leqslant \beta$ and $\alpha \leqslant \beta$
and since $I_0$ there is a $\gamma \in I_0$ such that $\beta \leqslant \gamma$. Putting the respective spaces and morphisms in a diagram:
% https://q.uiver.app/#q=WzAsNCxbMCwwLCJBX1xcYWxwaGEiXSxbMSwwLCJBX1xcYmV0YSJdLFswLDEsIkFfe1xcYWxwaGFfMH0iXSxbMiwwLCJBX1xcZ2FtbWEiXSxbMCwxLCJcXHZhcnBoaV97XFxhbHBoYSBcXGJldGF9Il0sWzIsMSwiXFx2YXJwaGlfe1xcYWxwaGFfMCBcXGJldGF9Il0sWzEsMywiXFx2YXJwaGlfe1xcYmV0YSBcXGdhbW1hfSJdLFsyLDMsIlxcdmFycGhpX3tcXGFscGhhXzAgXFxnYW1tYX0iLDJdXQ==
\[\begin{tikzcd}
	{A_\alpha} & {A_\beta} & {A_\gamma} \\
	{A_{\alpha_0}}
	\arrow["{\varphi_{\alpha \beta}}", from=1-1, to=1-2]
	\arrow["{\varphi_{\beta \gamma}}", from=1-2, to=1-3]
	\arrow["{\varphi_{\alpha_0 \beta}}", from=2-1, to=1-2]
	\arrow["{\varphi_{\alpha_0 \gamma}}"', from=2-1, to=1-3]
\end{tikzcd}\]
In this diagram $\varphi_{\alpha_0 \gamma}$ is an isomorphism, hence we have:
\begin{align*}
	\varphi_\alpha:
		A_\alpha &\to A_{\alpha_0} \\
		x &\mapsto (\varphi_{\alpha_0 \gamma})^{-1} \circ \varphi_{\beta \gamma} \circ \varphi_{\alpha \beta}(x)
\end{align*}

Therefore $A_{\alpha_0}$ is a target of $\{A_\alpha\}_{\alpha \in I}$.

Now let $B$ be any other target, with morphisms $\{\psi_\alpha\}_{\alpha \in I}$. Since $A_{\alpha_0}$
is in the system there is a morphism $\psi_{\alpha_0}: A_{\alpha_0} \to B$. This morphism is also unique
by construction and hence $A_{\alpha_0}$ is the limit of the direct system.
\end{proof}

Now it remains to show that there is such a subset of open covers. This is going to be the so called
"good" covers
\begin{definition}
Given a topological space $X$, a \textbf{good cover} is a cover $\{\U_\alpha\}_{\alpha \in I}$ such that
for any subset $J \subseteq I$ the intersection
\[
	\bigcap_{\alpha \in J} \U_\alpha
\]
is TODO
\end{definition}

Good covers are a good candidate for an eventually constant sequence:
\begin{lemma}
	TODO
\end{lemma}
It of course remains to show that the Čech cohomology of all good covers is isomorphic. While this is a
quite nontrivial fact by itself, for us it will follow as a direct corollary of the de Rham theorem – the
main subject of this thesis.

\section{TODO}
TODO:
	- Figure out whether concrete calculations of Čech cohomology even need their own chapter since we
	  already have quite some pages.
