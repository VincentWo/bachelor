\chapter{De Rham Cohomology}
Properly understanding the phenomena we have seen so far requires rephrasing them in the
language of differential geometry, which we are going to develop in this chapter –
starting with the generalization of differentiation to more general domains then just open
subsets of $\RR^n$.

While we are going to aim for a self-contained introduction to all the needed differential geometric
background, it is impossible to do so in our limited format without skipping quite some detail. If a full
fledged introduction is desired (and the topic is definitely more than interesting enough to deserve one),
then reading one of the plethora of proper ones is recommended e.g. \cite{lee_introduction_2013},
\cite{spivak_comprehensive_1979} or \cite{wendl_differential_2022} to name some of the authors favorites
in no particular order.

\section{Smooth Manifolds}
The usual definition of differentiability only works for open subsets of $\RR^n$, but
we want to generalize the notion to a more general class of spaces known as 
\textit{manifolds}. Manifolds are a foremost topological concept generalizing the concept
of embedded curves/surfaces:
\begin{definition}[Topological Manifold]
A \textbf{Topological Manifold} of dimension $n$ is a metrizable and separable\footnote{
	These are technical requirements to ensure that manifolds really behave like we would except of an
	"euclidean" space, because being locally homeomorphic to $\RR^n$ still leaves room for weird things to
	happen. Intuitively metrizability ensures uniqueness of limits while separability ensures that our spaces
	don't get "too big".
}
topological space $M$ in which every point $p \in M$ has a neighborhood $\U$ that is homeomorphic to
an open set $\V \subseteq \RR^n$.

Given such a neighborhood $\U$ and homeomorphism $\varphi: \U \to \V$ we call $(\U, \varphi)$ a \textbf{chart}
or \textbf{local coordinate system} of $M$.
\end{definition}

Though this abstract definition of a manifold might be new to the reader, there are some topological manifolds that
they probably already know since grade school\footnote{not to be confused with the ones encountered
in grad school}:
\begin{example}
The archetypical (nontrivial) example of this concepts are the $n$-spheres, defined as
\[
	S^n \coloneqq \{\, x \in \RR^{n + 1} \mid |x| = 1 \,\}
\]

with $S^1$ more commonly known as the circle. As one sees in \ref{figure:spheres}
for $S^1$ and $S^2$, these all are quite different from $\RR^n$, while locally being homeomorphic to it.
Something that becomes evident every time one uses a navigation app, since these usually display a projection
of some subset of $S^2$ to a subset of $\RR^2$.
\end{example}
\begin{figure}[thb]
	\centering
	\tcbsetforeverylayer{enhanced, colback=white, frame hidden}
	\begin{tcolorbox}[segmentation style={solid}, sidebyside]
		    \tcbincludegraphics[height=6cm]{graphics/2_sphere.pdf}
		\tcblower
		    \tcbincludegraphics[height=6cm]{graphics/3_sphere.pdf}
	\end{tcolorbox}
	\label{figure:spheres}
	\caption{$S^1$ and $S^2$ respectively}
\end{figure}


Topological manifolds are an interesting subject by themselves and one might wonder whether it is already
possible to define a notion of "differentiation" for maps between them. Since differentiability itself is a
local condition, coming up with a definition of differentiability that seems to work does not take too long:
Given a function $f: M \to N$ between two manifolds we could call it differentiable at a point $p \in M$
if for all local choices of coordinates $(\U, x)$ around $p$ and $(\V, y)$ around $f(p)$ the function
\[
	y \circ f \circ x^{-1}: \RR^m \to \RR^n
\]
is differentiable. While looking sensible, this approach does not really work.
It has the major issue of excluding too many functions, because differentiability of a function with respect
to two charts does not imply anything about its differentiability with respect to other charts. Even something
like the identity on $\RR$, which is obviously smooth with respect to the standard choices of coordinates
(i.e. the identity itself) would not be differentiable by this definition: Defining $y: \RR \to \RR$ to be
identity and $x: \RR \to \RR$ as $y(p) = p^3$ results in
\[
	y \circ \mathrm{Id} \circ x^{-1}(z) = z^{1/3}
\]
which is not differentiable! But this flaw already suggests a fix: One would need to ensure that
differentiability of a function with respect to some charts also implies differentiability with
respect to all other charts. The easiest way to achieve this is by using only charts that are
\textbf{smoothly\footnote{
	We are going to assume smoothness (i.e. $C^\infty$) instead of differentiability from here on since it
	makes a lot of definitions and theorems a lot easier to digest and we are only interested in the smooth
	case anyway.
} compatible with each other}:
\begin{definition}
	Let $M$ be a topological manifold. We call two charts $(\U, x), (\V, y)$ \textbf{smoothly compatible}
	if $x(\U \cap \V)$ and $y(\U \cap \V)$ are open and $x \circ y^{-1}$ and $y \circ x^{-1}$ are smooth.
\end{definition}
Using the chain rule, a map $f$ is smooth with respect to two charts if and only if it is smooth with respect
to all smoothly compatible charts. Adding a choice of smoothly compatible charts on a manifold gives us a new
object:
\begin{definition}
A smooth manifold is a topological manifold with a choice of pairwise smoothly compatible charts
$\{(\U_\alpha, x_\alpha)\}_{\alpha \in I}$ such that $\bigcup_{\alpha \in I} \U_\alpha = M$.
\end{definition}
Given smooth manifolds, our suggested definition of smoothness makes sense:
\begin{definition}
Given two smooth manifolds $M, N$ we call a function $M \to N$ smooth if for all choices of coordinates
$(\U, x)$ of $M$ and $(\V, y)$ of $N$ the function
\[
	y \circ f \circ x^{-1}: x(f^{-1}(\V)) \to y(\U)
\]
is smooth.
\end{definition}
Unsurprisingly smooth functions are the natural maps between smooth manifolds and hereinafter every map between
smooth manifolds is assumed to be smooth unless stated otherwise.

Smoothness is also used to define an equivalence relation on manifolds that is more useful than
strict equality:
\begin{definition}
	Let $M,N$ be smooth manifolds. We call a smooth map $f: M \to N$ \textit{diffeomorphism} if it is
	bijective and its inverse is also smooth.

	Two manifolds that admit a diffeomorphism between them are called \textit{diffeomorphic}.
\end{definition}
\begin{remark}
Given a topological manifold $M$ one would hope that all the smooth manifolds arising from $M$ 
would be diffeomorphic to each other, but in general this is not the case. 
\end{remark}

Though we defined smoothness itself, it remains to define the actual derivative of a smooth function. To
accomplish this let's take a closer look at the familiar derivative. Intuitively the derivative of
a function at a point $p$ takes a direction and outputs a linear approximation of the change of $f$ given
a move from $p$ into that direction. The space of "directions" at $p$ is usually just taken to be $\RR^n$,
but we are going to get a bit technical about it and denote it as $T_p \RR^n$, which we call the
\textbf{tangent space at $p$} (Morally the difference between $T_p \RR^n$ and $\RR^n$ is that the former
describes vectors based at $0$ and the later vectors based at $p$). This tangent space can also be used to
describe the linear approximation of the change of $f$, enabling us to denote the derivative of $f$ at $p$
as a linear map
\[
	D_p f: T_p \RR^n \to T_{f(p)} \RR^m
\]
While this might seem like meaningless notational tricks, it hints at finding a useful definition of $T_p M$
being a substantial step in the translation of the derivative to arbitrary smooth manifolds.
Notice that for any $p \in M$ a choice of chart $(\U, x)$ gives rise to a chart-specific tangent space
$T_{x(p)} \RR^n$, which we are going to denote as $T_p^x M$ (Note that this space has a canonical basis
arising from the canonical basis of $T_p \RR^n$, which we are going to denote by
$\frac{\partial}{\partial x_1}, \dots, \frac{\partial}{\partial x_n}$). While this seems promising, one would
not want to have to choose local coordinates to make sense of $T_p M$. And one does not have to, by realizing
that while any other chart $(y, \V)$ around $p$ induces a different tangent space $T_p^y M$, there is
a canonical isomorphism to $T_p^x M$ given by the derivative of the change of coordinates:
\[
	D_{x(p)} (x \circ y^{-1}): T^x_p M \to T^y M
\]
(This is an isomorphism since $x \circ y^{-1}$ is a diffeomorphism). Then one can avoid any choice of
coordinates by defining $T_pM$ as the isomorphism class of $T^x_p M$ for all charts $(x, \U)$ around $p$. This is
still a vector space, though without a canonical basis (but choosing local coordinates $(x, \U)$ induces one,
which with a little abuse of notation we also denote
$\frac{\partial}{\partial x_1}, \dots, \frac{\partial}{\partial x_n}$). 

It remains to define the derivative itself. For any given charts $(x, \U)$ around $p$ and $(y, \V)$ around
$f(p)$ there is a derivative of $f$ with respect to these coordinates given by
\[
	D_{x(p)} (y^{-1} \circ f \circ x): T_{x(p)} \RR^n = T^x_p M \to T^y_{f(p)} N = T_{y(f(p))} \RR^n
\]
This can then be used to induce a map from $T_p M$ to $T_{f(p)}N$:
\begin{definition}
Let $M, N$ be smooth manifolds and $f: M \to N$ a smooth function. We define its derivative or \textbf{tangent
map} at $p$ denoted by $T_p f: T_p M \to T_{f(p)}N$ as the unique linear map that for any choices of
coordinates $(x, \U)$ around $p$ and $(y, \V)$ around $f(p)$ fulfills 
\[
	T_p f = j^y_{f(p)} \circ \left(D_{x(p)} (y \circ f \circ x^{-1})\right) \circ  (i^x_p)^{-1}
\]
with $i^x_p: T_p^x M \to T_p M$ and $j^y_{f(p)}: T^y_{f(p)}N \to T_{f(p)} N$ being the canonical isomorphisms.
\end{definition}
To ensure that this map really exists we have to prove that any map fulfilling this equality for charts
$(\U, x)$ around $p$ and $(\V, y)$ around $f(p)$ also fulfills it for a different choice of charts
$(\tilde{\U}, \tilde{x})$ around $p$ and $(\tilde{V}, \tilde{y})$ around $f(p)$. The chain rule gives us a
connection between the derivative of $f$ with respect different choices of coordinates:
\begin{align*}
	D_{x(p)} (y \circ f \circ x^{-1})
		&= D_{x(p)} (y \circ \tilde{y}^{-1} \circ \tilde{y} \circ f \circ \tilde{x}^{-1} \circ \tilde{x} \circ x^{-1}) \\
		&= D_{\tilde{y}(f(p))} (y \circ \tilde{y}^{-1})
		   \circ D_{\tilde{x}(p)} (\tilde{y} \circ f \circ \tilde{x}^{-1})
		   \circ D_{x(p)} (\tilde{x} \circ x^{-1}) \\
\end{align*}
In this equation $D_{x(p)}(\tilde{x} \circ x^{-1})$ is the coordinate changing isomorphism between
$T^x_p M$ and $T^{\tilde{x}}_p M$ hence $D_{x(p)}(\tilde{x} \circ x^{-1}) \circ (i^x_p)^{-1}$ is just
$(i^{\tilde{x}}_p)^{-1}$. Vice versa with $D_{\tilde{y}(f(p))} (y \circ \tilde{y}^{-1})$ resulting in
\begin{align*}
	&j^y_{f(p)} \circ \left(D_{x(p)} (y \circ f \circ x^{-1})\right) \circ  (i^x_p)^{-1}\\
	&= j^y_{f(p)} \circ D_{\tilde{y}(f(p))} (y \circ \tilde{y}^{-1})
		    \circ D_{\tilde{x}(p)} (\tilde{y} \circ f \circ \tilde{x}^{-1})
		    \circ D_{x(p)} (\tilde{x} \circ x^{-1})
			\circ (i^x_p)^{-1} \\
	&= j^{\tilde{y}}_{f(p)} \circ D_{\tilde{x}(p)}(\tilde{y} \circ f \circ \tilde{x}^{-1}) \circ (i^{\tilde{x}}_p)^{-1}
\end{align*}
This makes $T_p f$ well-defined.

The theory surrounding tangent spaces is rich and interesting and we only have time to scratch its surface,
having to take some of the remaining things as black boxes with proofs and details left to the earlier
mentioned proper accounts.

One can define the \textbf{tangent bundle} $TM \coloneqq \bigcup_{p \in M} T_p M$ as the disjoint union of
all tangent spaces\footnote{Though the disjoint union is meant we used $\bigcup$ instead of $\coprod$ since
the later carries certain (wrong) connotations about the topology of $TM$}. $TM$ can then be equipped
with a smooth manifold structure of dimension $2n$. For any smooth function $f: M \to N$ one can then
unify all tangent maps to define $Tf: TM \to TN$, which is smooth with respect to the aforementioned
smooth structure of $TM$/$TN$. Vector fields on $\RR^n$ can then be generalized to smooth manifolds as
maps $X: M \to TM$ which fulfill $X(p) \in T_p M$. This definition includes the old one if one takes
$\RR^n$ as a smooth manifold with its canonical smooth structure.

\section{Differential Forms}
Having developed the theory of tangent spaces, we can start translating our introductory motivation from
$\RR^n$ to general manifolds. Our first example was about the relationship between the connected
components of open subsets of $\RR^n$ and the smooth functions with vanishing derivative defined on it.
The fact that locally constant functions are related to connected component directly translates to smooth
manifolds, all that is left is to find a nice description of these using derivatives. This is a bit more involved:
The tangent map of a smooth function $f: M \to \RR$ is not going to vanish if $f$ is locally constant. This is
because $Tf$ not only contains information about the change of $f(p)$, but also about $f(p)$ itself. In the
case of general manifolds this is necessary, but in the case of $\RR$ there is a canonical isomorphism between
$T \RR$ and $\RR \times \RR$ (with the first component representing points and the second tangent vectors).
This isomorphism can be used to project $T f$ to the second component (i.e. to the "change"). We call this
projection the \textbf{differential of $f$} and denote it by $\rmd f$. Then $f$ is locally constant if and
only if its differential is zero everywhere, letting us describe the locally constant functions as the kernel
of $\rmd$.

While this went quite nicely, the second example does not translate quite as well. Recall that we talked
about a connection between a space being simply connected and vector fields with vanishing curl being
conservative. While vector fields do exist on arbitrary manifolds there is neither an obvious way to define
their curl nor what it means for a vector field to be "conservative" since the derivative of smooth functions
is not a vector field. These issue arise because on general smooth manifolds vector fields aren't the right
tool to capture the phenomena we have seen so far. To do that we need to define \textbf{differential forms},
which we are going to motivate as the natural objects of integration on smooth manifolds, before using them
to define the de Rham cohomology, the true object of our study.
% form of derivative called the \textbf{exterior derivative}.
% Neither is $\rmd f$ a
% vector field nor is there 
% problem of $\mathrm{d}f$ not actually being a vector field i.e. a function $M \to TM$, but rather a
% function $TM \to \RR$. But one can make more sense of this by rewriting it a bit: Since $\rmd f$ maps
% $(p, v)$ to some value in $\RR$ with $p \in M, v \in T_p M$ and is linear in $v$ one can also view it as 
% mapping every $p$ to a linear map from $T_p M$ to $\RR$ i.e. to the dual space of $T_p M$. The dual space
% of $T_p M$ is going to be denoted as $T_p^* M$, and then similarly to the case of the tangent bundle it is
% possible to extend this definition to the so called \textbf{cotangent bundle} $T^* M$ consisting of the
% union of all $T^*_pM$. Then $\rmd f$ can be taken as a function $M \to T^*M$ or a \textbf{covector field}.

% The connection to our original motivation can then be found as follows: On $\RR^n$ there is a natural
% isomorphism between $\RR^n$ and its dual space given by the euclidean dot product and every vector field
% $\RR^n \to T\RR^n$ can be identified with a covector field $\RR^n \to T^*\RR^n$.\footnote{
% 	A similar isomorphism is also possible in the more general case of Riemannian manifolds. These
% 	are intuitively smooth manifolds with a dot product structure on the tangent space, which produces
% 	a natural isomorphism between tangent and cotangent vectors. But in general there is no such canonical
% 	isomorphism.
% }
%
% Differential forms are

Differential forms are needed for integration on manifolds since by itself a manifold has no canonical
notion of "distance"/area/volume (e.g which value should $\int_{S^1} 1\, \rmd x$ take?). While one could
define integrals by adding data to the manifold itself (e.g. a measure) differential forms offer an
alternative approach, including these informations in the integrands themselves. They do that by mapping each
point $p$ to a local notion of volume, though in a way that needs a bit of explanation.

This local notion of volume around $p$ can be thought of assigning volume to parallelepipeds based at $p$.
Of course parallelepipeds do not exist on arbitrary manifolds, but one can approximate them through the use
of tangent vectors i.e. every $n$-tuple of tangent vectors in $T_p M$ can be interpreted as an
$n$-parallelepiped. Thus we want to describe volume using maps from $(T_p M)^n$ to $\RR$, though we also
want them to fulfill some more requirements to make them more "volume-like".

Firstly we are going to require them to map linear dependent tangent vectors to $0$ since these represent
degenerate parallelepipeds. Secondly we want them to be multi-linear, like the normal volume of
parallelepipeds. This has the odd consequence of having to deal with negative "volume", but this can be seen
as a feature instead of a flaw by interpreting signed volume as a volume (the absolute value)
together with an \textit{orientation} (the sign).

Orientation can be intuitively understood as some additional data added to a space that gives a sense of
direction. This sense of direction is probably easiest to understand in one dimension, where it is really
just a choice of direction e.g. to go clockwise or counterclockwise around a circle. In higher dimensions it
is a bit more involved and can be best understood as choosing an in-/outside e.g. choosing on a plane which
side is up.

% We are going to formalize this concept without having to completely define what an orientation is, just by
% looking at how it behaves under diffeomorphisms. In one dimension every diffeomorphism from $\RR$ to $\RR$ is
% either increasing or decreasing and only the increasing functions preserve direction i.e. orientation. By
% characterizing these as the diffeomorphisms with positive derivative one can take this to
% higher dimensions as functions whose differential has positive determinant everywhere. \footnote{
% 	This can also be motivated by realizing that every invertible matrix can be decomposed into a
% 	rotation, a (positive) scaling and another rotation (singular value decomposition) and that the
% 	matrices with negative determinant are exactly those for which one of these rotations has negative
% 	determinant and thus includes a mirroring i.e. a orientation reversal.
% }
% \begin{definition}
% A diffeomorphism $f: \RR^n \supseteq \U \to \V \subseteq \RR^n$ is called \textbf{orientation preserving} if
% $\det(D f(p)) > 0$ for every $p \in \RR^n$.
% \end{definition}
% If one would want to talk about orientation on a manifold 
% Using this definition the generalization of orientation to smooth manifolds becomes quite doable, just by
% noticing what one has to prevent. even talk about orientation on a manifolds requires it to be coordinate independent i.e. all of our coordinate changes need to
% be orientation preserving, which turns out to be enough:
% \begin{definition}[Oriented Manifold]
% 	An \textbf{Oriented Manifold} is a smooth manifold for which all changes of coordinates are orientation preserving i.e. whose
% 	differential has positive determinant everywhere.
% \end{definition}
% Note that this definition falls apart in the 0-dimensional case, but for technical reason we define an
% oriented 0-manifold to be a 0-manifold $M$ together with a map $\varepsilon: M \to \{-1, +1\}$ and call
% the points in $\varepsilon^{-1}(\{+1\})$ positively oriented and vice versa. 

After this short interlude about orientations has (hopefully) made the notion of negative volume seem sensible
we can almost define differential forms, though some technicalities are still needed.

We wanted our differential forms to map linear dependent vectors to zero. While this requirement is a bit
unwieldy one might recall from linear algebra the following convenient lemma:
\begin{lemma}
A multi-linear map $f: V \times \dots \times V \to \RR$ is anti-symmetric if and only if it maps every collection
of linearly dependent vectors to zero.
\end{lemma}
\begin{proof}
	We start with proving that any anti-symmetric multi-linear function $f$ maps linearly dependent vectors to
	zero: Let $v_1, \dots, v_n$ be linearly dependent. Then one can replace one of them (w.l.o.g. assume it
	is $v_1$) with a linear combination of the others:
	\begin{align*}
		f(v_1, \dots, v_n)
			= f\left(\sum_{i = 2}^n \lambda_i v_i, v_2, v_2, \dots, v_n\right)
			= \sum_{i = 2}^n \lambda_i f(v_i, v_2, v_2, \dots, v_n)
	\end{align*}
	Since $f$ is antisymmetric it maps any combination of vectors that includes duplicates to zero hence
	every term in the sum is zero, therefore the whole sum is zero.

	If on the other hand $f$ maps linearly dependent vectors to zero than it holds for all vectors
	$v_1, \dots, v_n$:
	\begin{align*}
		0 &= f(v_1 - v_2, v_2 - v_1, v_3, \dots, v_n) \\
		  &= f(v_1, v_2 - v_1, v_3, \dots, v_n) - f(v_2, v_2 - v_1, v_3, \dots, v_n) \\
		  &= f(v_1, v_2, v_3, \dots, v_n) + f(v_2, v_1, v_3, \dots, v_n) \\
		  &\phantom{=} - \underbrace{f(v_1, v_1, v_3, \dots, v_n)}_{= 0} - \underbrace{f(v_2, v_2, v_3, \dots, v_n)}_{= 0} \\
		  &=f(v_1, v_2, v_3, \dots, v_n) + f(v_2, v_1, v_3, \dots, v_n) \\
	\end{align*}
	Hence
	\[
		f(v_1, v_2, v_3, \dots, v_n) = - f(v_2, v_1, v_3, \dots, v_n)
	\]
	This argument can be done for any combination of indices and therefore $f$ is antisymmetric. 
\end{proof}
Because of this lemma one usually sees differential forms defined as antisymmetric multilinear maps and we
are going to refer to this property in the future since it is more succinct, but it is important to remember
where it comes from.

As often in differential geometry, we aren't  interested in all differential forms, but just the smooth ones,
though their smoothness remains to be defined with the use of a bit more multi-linear algebra. We recall that
any given basis $b_1, \dots, b_n$ of a vector space $V$ gives rise to a dual basis $e^1, \dots, e^n$  defined
as the unique linear maps $V \to \RR$ satisfying:
\[
	e^i(b_j) = \begin{cases}
		1 &\text{if } i = j \\
		0 &\text{otherwise}
	\end{cases}
\]

Since any chart $(\U, x)$ around a point $p$ induces a basis on $T_p M$ it also induces a dual basis on
$T_p^* M$, which we are going to denote as $\rmd x^1, \dots, \rmd x^n$.
Then, given a differential 1-form $\omega: M \to T^*M$ for any chart $(\U, x)$ there are unique component functions $\omega_1, \dots, \omega_n$ such that
\[
	\omega(p) = \omega_1(p) \rmd x^1 + \dots + \omega_n(p) \rmd x^n
\]
for any $p \in \U$. This makes it reasonable to call a differential form smooth if and only if its component
functions are smooth for any chart. Taking this definition to higher dimensions requires a representation of
higher dimensional multi-linear maps in a similar form, which in the special case of anti-symmetric maps
can be achieved in a succinct way using the so called \textbf{wedge product}.

To define the wedge-product one needs to start with defining the tensor product:
\begin{definition}  
	Let $\alpha: V^k \to \RR, \beta: V^\ell \to \RR$ be multi-linear maps, we define their
	\textbf{tensor product} as the multi-linear map $\alpha \otimes \beta: V^{k + \ell} \to \RR$ defined
	as 
	\[
		(\alpha \otimes \beta)(v_1, \dots, v_k, w_{1}, \dots, w_{\ell})
			= \alpha(v_1, \dots, v_k)\beta(w_1, \dots, w_\ell)
	\]		
\end{definition}
The tensor product of two anti-symmetric maps is sadly not anti-symmetric itself, but there is a way to
anti-symmetrize any multi-linear map:
\begin{definition}
	Let $\alpha: V^k \to \RR$ be a multi-linear map. We define its anti-symmetrization as
	\[
		\mathrm{Alt}(\alpha)(v_1, \dots, v_k)
			= \frac{1}{k!} \sum_{\sigma \in S_k} (-1)^{|\sigma|} \alpha(v_{\sigma(1)}, \dots, v_{\sigma(k)})
	\]
\end{definition}
It is easy to see that $\mathrm{Alt}(\alpha)$ is always anti-symmetric and we can now use it (plus a
combinatorial factor) to define:
\begin{definition}
	Let $\alpha: V^k \to \RR, \beta: V^\ell \to \RR$ be anti-symmetric multilinear maps. We define their
	wedge product as
	\[
		\alpha \wedge \beta = \frac{(k + \ell)!}{k! \ell!} \mathrm{Alt}(\alpha \otimes \beta)
	\]
\end{definition}
The wedge product of dual basis vectors is especially interesting and can be described explicitly:
\begin{remark}
This whole section is not finished yet!
\end{remark}
\begin{theorem}
Given a basis $v_1, \dots, v_n$ of a vector space $V$ with corresponding dual basis
$\rmd x^1, \dots, \rmd x^n$, the wedge product of $k$ dual basis vectors $\rmd x^{i_1}, \dots, \rmd x^{i_k}$
fulfills
\[
	(\rmd x^{i_1} \wedge \dots \wedge \rmd x^{i_k})(e^{j_1}, \dots, e^{})
\]
\end{theorem}
Starting with two dual basis vectors $\rmd x^i, \rmd x^j$:
\[
	(\rmd x^i \wedge \rmd x^j)(v, w)
		= (\rmd x^i \otimes \rmd x^j)(v, w) - (\rmd x^i \otimes \rmd x^j)(w,v)
		= \rmd x^i(v) \rmd x^j(w) - \rmd x^i(w) \rmd x^j(v)
\]

Before doing anything with the wedge product, we are going to prove some of its elemental properties:
\begin{theorem}
	The wedge product fulfills the following properties:
	\begin{itemize}
		\item
			The wedge product is bilinear
		\item
			The wedge product is graded commutative, that is for a $k$-form $\alpha$ and an $\ell$-form
			$\beta$ 
			\[
				\alpha \wedge \beta = (-1)^{k \ell} \beta \wedge \alpha	
			\]
		\item
			The wedge product is associative
	\end{itemize}
\end{theorem}
\begin{proof}
Since the tensor product is bilinear and $\mathrm{Alt}$ is linear, the wedge product is bilinear.

To prove its associativity, let $\alpha, \beta, \gamma$ be anti-symmetric $k, \ell, m$-forms. Then
\begin{align*}
	(\alpha \wedge \beta) \wedge \gamma
		&= \frac{((k + \ell + m)!)}{(k + \ell)!m!}%\Alt((\alpha \wedge \beta) \otimes \gamma)
\end{align*}

\end{proof}
\begin{remark}
The unfinished bits end here
\end{remark}

By using the wedge product we can use a dual basis $\rmd x^1, \dots, \rmd x^n$ to form a basis for the
space of $k$-fold anti-symmetric maps in the form of wedge products of $k$ dual basis vectors so that any
differential $k$-form $\omega$ on an manifold can locally be written as
\[
	\omega(p) = \sum_{i_1 < \dots < i_k} \omega_{i_1, \dots, i_k}(p) \rmd x^{i_1} \wedge \dots \wedge \rmd x^{i_k}
\]
for some chart $(\U, x)$. Then we can call a differential form smooth if its components functions with respect
to all charts are smooth.

% We are also going to use this to introduce a useful abuse of notation: Though technically
% $\omega_{i_1, \dots, i_k}$ is only defined for increasing indices, define it for all indices by antisymmetry
% i.e. $\omega_{\sigma(i_1), \dots, \sigma(i_k)} = (-1)^{|\sigma|} \omega_{i_1, i_2, \dots, i_k}$ for any
% permutation $\sigma$.

Combining all these we can finally define:
\begin{definition}
	A \textbf{differential $k$-form} on an $n$-manifold $M$ is a map that assigns to any $p \in M$
	an antisymmetric $k$-fold multilinear function from $T_pM \times \dots \times T_pM$ to $\RR$ such that
	all its components functions with respect to all charts are smooth.

	The space of all differential $k$-forms on $M$ (also called the \textbf{de Rham cocomplex of $M$} is
	denoted as $\Omega^k(M)$ and
	\[
		\Omega^*(M) \coloneqq \bigoplus_{k = 0}^\infty \Omega^k(M)
	\]
\end{definition}
Note that this definition also defines differential 0-forms, which smoothly assign to every point
$p \in M$ a function from the Cartesian product of 0 vector spaces (which is by definition just a point) to
$\RR$ i.e. just a value from $\RR$. Thus differential 0-forms are just the smooth functions on a manifold.

We actually already encountered examples of 1-forms:
\begin{example}
	Given a smooth map $f: M \to N$ its differential $\rmd f: TM \to \RR$ restricted to any $T_p M$ is a linear
	map and therefore can be seen as assigning a linear map to each point of $M$ i.e. it is a smooth 1-form.
	Even better we can write this 1-form down explicitly for some local chart $(\U, x)$:
	\[
		df = \sum_{i = 1}^n \frac{\partial f}{\partial x^i} \rmd x^i
	\]
	with $\frac{\partial f}{\partial x^i}$ being a common shorthand for $\partial_i (f \circ x^{-1})$
\end{example}
% Note that the wedge product can be uniquely extended to all differential forms by requiring it to be
% bilinear and graded commutative i.e. for differential forms $\alpha$ and $\beta$ of degree $q$ and $p$ it
% holds
% \[
% 	\alpha \wedge \beta = (-1)^{p q} \beta \wedge \alpha
% \]
% This gives the differential forms on a manifold not only the structure of a vector space, but also the
% structure of a graded algebra (you do not have to care about this).
Having defined differential forms it remains to define their derivative, which we are going to do in a
purely algebraic way:
\begin{definition}
Let $M$ be a smooth manifold. The exterior derivative $\rmd: \Omega^*(M) \to \Omega^*(M)$ is the
unique linear operator mapping $k$-forms to $k + 1$-forms that fulfills:
\begin{itemize}
	\item On 0-forms i.e. smooth functions it matches the differential
	\item It satisfies the graded Leibniz rule
		\[
			\rmd (\alpha \wedge \beta) = \rmd \alpha \wedge \beta + (-1)^k \alpha \wedge \rmd \beta
		\]
		for any homogeneous forms $\alpha, \beta$ with $k$ being the degree of $\alpha$
	\item It is nilpotent i.e. $\rmd \circ \rmd = 0$
	\item It is local, in the sense of $(\rmd \omega)(p)$ only depends on the values of $\omega$ on
		small neighborhoods of $p$.
\end{itemize}
\end{definition}
We now just have to proof that a map with these properties is indeed unique and that it is exists. First
about uniqueness:
\begin{proof}
	Since $\rmd$ is defined to be local it is sufficient to prove uniqueness of $\rmd$ only locally.
	Any $k$-form $\omega$ is locally just a sum of terms of the form
	$\omega_{i_1 \dots i_k} \rmd x^{i_1} \wedge \dots \wedge \rmd x^{i_k}$ for some chart $(\U, x)$. This
	already determines $\rmd \omega$ since $\rmd$ being linear, fulfilling the graded Leibniz rule and being
	equivalent to the differential implies:
	\begin{align*}
		\rmd \omega
			&= \rmd \left( \sum_{i_1 < \dots < i_k} \omega_{i_1 \dots i_k} \rmd x^{i_1} \wedge \dots \wedge \rmd x^{i_k} \right) \\
			&= \sum_{i_1 < \dots < i_k} \rmd \left(\omega_{i_1 \dots i_k} \rmd x^{i_1} \wedge \dots \wedge \rmd x^{i_k} \right) \\
			&= \sum_{i_1 < \dots < i_k} \rmd \omega_{i_1 \dots i_k} \rmd x^{i_1} \wedge \dots \wedge \rmd x^{i_k} \\
			&= \sum_{i_1 < \dots < i_k} \sum_{j = 1}^n \frac{\partial }{\partial x^j}\omega_{i_1 \dots i_k} \rmd x^j \wedge \rmd x^{i_1} \wedge \dots \wedge \rmd x^{i_k} \\
	\end{align*}
	This not only proves that $\rmd$ is unique, it also gives us a possible formula for it. Thus proofing
	existence is reduced to proving that this formula is independent of choices of coordinates and that it
	actually fulfills all required properties.

	This formula is by definition local and it is easy to see that it is indeed a linear operator and 
	matches the differential on smooth functions. It remains to prove the Leibniz rule and nilpotency.
	Because of linearity it is enough to prove the Leibniz rule for homogeneous terms
	$f \rmd x^{i_1} \wedge \dots \wedge \rmd x^{i_k}$ and $g \rmd x^{j_1} \wedge \dots \wedge x^{j_\ell}$,
	which we are going to abbreviate as $f \rmd x^I$ and $g \rmd x^J$ for multi-indices $I,J$.
	A computation then leads to:
	\begin{align*}
		\rmd( (f \rmd x^I) \wedge (g \rmd x^J) )
			&= \rmd ( fg \rmd x^I \wedge \rmd x^J) \\
			&= \sum_{i = 1}^n \frac{\partial}{\partial x^i} (fg) \rmd x^i \wedge \rmd x^I \wedge \rmd x^J \\
			&= \sum_{i = 1}^n
				\left(\frac{\partial f}{\partial x^i} g + f \frac{\partial g}{\partial x^i}\right)
					\rmd x^i \wedge \rmd x^I \wedge \rmd x^J \\
			&= (g \rmd f + f \rmd g) \rmd x^I \wedge \rmd x^J \\
			&= g (\rmd f \wedge \rmd x^I \wedge \rmd x^J) + f (\rmd g \wedge \rmd x^I \wedge \rmd x^J) \\
			&= \rmd (f \rmd x^I) \wedge (g \rmd x^J) + (-1)^k (f \rmd x^I) \wedge \rmd (g \rmd x^J)
	\end{align*}		
	With the $(-1)^k$ appearing since $\rmd g$ needs to be "moved" right over the $k$ wedge-products in
	$\rmd x^I$. Therefore the Leibniz rule holds for $\rmd$, leaving only nilpotency. By linearity it
	again suffices to prove this for terms $f \rmd x^I$ (with $I$ being a multi-index as above):
	\begin{align*}
		\rmd \rmd (f \rmd x^I)
			&= \rmd(\sum_{i = 1}^n \frac{\partial f}{\partial x^i} \rmd x^i \wedge \rmd x^I) \\ 
			&= \sum_{i = 1}^n \sum_{j = 1}^n \frac{\partial^2 f}{\partial x^j \partial x^i} \rmd x^j \wedge \rmd x^i \wedge \rmd x^I \\
	\end{align*}
	Since partial derivatives commute and $\rmd x^i \wedge \rmd x^j = -\rmd x^j \wedge \rmd x^i$ this sum is
	zero, hence $\rmd \circ \rmd = 0$.

	It remains to prove that $\rmd$ defined in this way is coordinate independent, but since we have proven
	that $\rmd$ is unique with these properties every other choice of coordinates has to result in the same
	map.
\end{proof}

% \begin{definition}
% Given an oriented $n$-manifold $M$, a chart $(\U, x)$ and a differential $n$-form
% $\omega = f \rmd x^1 \wedge \dots \wedge \rmd x^n \in \Omega^n(M)$, then we define its integral
% over any open set  $\O \subseteq \U$ as:
% \[
% 	\int_\O \omega
% 		= \int_\O f \rmd x^1 \wedge \dots \wedge \rmd x^n = \int_{x(\O)} f(p)\, \rmd\lambda_n(p)
% \]
% \end{definition}
% Before we can work with this definition we have to check that it is independent of coordinates
% i.e. that for another chart $(\V, y)$ and an open set $\O \subseteq \U \cap \V$ both charts
% lead to the same result. By the transformation formula:
% \begin{align*}
% 	\int_\O \omega
% 		= \int_{\O} f \rmd x^1 \wedge \dots \rmd x^n
% 		&= \int_{x(\O)} f(x(p)) \lambda_n(p) \\
% 		&= \int_{y(\O)} f(y(p)) |\det(D(y \circ x^{-1}))| \lambda_n(p) \\
% 		&= \int_\O f |\det(D(y \circ x^{-1}))| \rmd y^1 \wedge \dots \wedge \rmd y^n \\
% 		&= \int_\O \omega 
% \end{align*}
% Where the last equality depends on the behavior of differential forms under changes of coordinates and
% only works because of our orientation assumption, which guarantees that $\det(D(y \circ x^{-1}))$ is
% positive.

% TODO: Compactness

% Therefore our definition is coordinate independent though it is still limited to open sets contained in one
% chart. To extend this to integral over the whole manifold we are going to use a fundamental tool of
% differential geometry that is often used to "glue" local definitions into global ones:
% \begin{definition}
% Let $M$ be a smooth manifold with an open cover $\{\U_\alpha\}_{\alpha \in I}$. We call a family of functions
% $\{\varphi_\alpha\}_{\alpha \in I}$ with every $\varphi_\alpha$ being a smooth function from $M$ to $\RR$ a
% \textbf{partition of unity subordinate to $\{\U_\alpha\}_{\alpha \in I}$} if
% \begin{itemize}
% 	\item The support of every $\varphi_\alpha$ is contained in $\U_\alpha$
% 	\item At every point $p \in M$ only finitely many $\varphi_\alpha$ take on a non-zero value
% 	\item $\sum_{\alpha \in I} \varphi_\alpha(p) = 1$ for every $p \in M$ (This sum is technically not
% 	well-defined as a sum of potentially infinite functions, but locally only finite of them are non-zero
% 	hence one can still make sense of it)
% \end{itemize}
% \end{definition}
% Given a partition of unity subordinate to the charts of a smooth manifold it is not too hard to define
% integration globally:
% \begin{definition}
% Let $M$ be a smooth manifold with charts $\mathfrak{U} = \{(\U_\alpha, x_\alpha)\}_{\alpha \in I}$ and
% $\{\rho_\alpha\}_{\alpha \in I}$ a partition of unity subordinate to $\mathfrak{U}$. Define the integral of
% a differential $n$-form $\omega \in \Omega^n(M)$ with compact support as:
% \[
% 	\int_M \omega = \sum_{\alpha \in I} \int_{\U_\alpha} \rho_\alpha \omega
% \]
% where $\{\rho_\alpha\}_{\alpha \in I}$ is a partition of unity subordinate to the charts of $M$.
% \end{definition}

% Though this definition works, it critically depends on the existence of a partition of unity. Luckily on 
% smooth manifolds this is always the case:
% \begin{lemma}
% 	Let $M$ be a smooth manifold with an open cover $\{\U_\alpha\}_{\alpha \in I}$. Then there exists a
% 	partition of unity subordinate to it.
% \end{lemma}
% Since the proof of this lemma is rather lengthy and technical we are going to omit it, referring the reader to
% the literature mentioned earlier.

% Defining integration for all differential forms requires us to develop some more technicalities first. Since a $k$-differential form defines
% a $k$-volume it only makes sense to integrate it on a $k$-manifold, so integrating differential $k$-forms on an $n$-manifold for $n > k$ requires
% first developing the notion of a \textbf{submanifold}, which we are going to define as the images of special maps:
% \begin{definition}
% We call a smooth map $f: M \to N$ an \textbf{embedding} if it is an homeomorphism onto its image (i.e. $f: M \to f(M)$ is a bijection with
% continuous inverse) and its tangent map $Tf: TM \to TN$ is injective for every restriction $T_p f: T_p M \to T_{f(p)} N$.
% \end{definition}
% Intuitively the first condition ensures that $f$ preserves the "dimensionality" of $M$ while the second ensures smoothness of the result.

% These maps characterize smooth submanifolds:
% \begin{definition}
% Let $M$ be a smooth manifold. We call $S \subseteq M$ a \textbf{(smooth) $k$-submanifold of $M$} if there is a $k$-manifold $N$ and an 
% embedding $f: N \to M$ such that $S = \Img(f)$.
% \end{definition}

% Then we can define:
% \begin{definition}
% Let $M$ be a smooth manifold and $\omega \in \Omega^k(M)$ a differential $k$-form. Then we define the
% integral of $\omega$ over any compact $k$-submanifold $S \subseteq M$ as
% \[
% 	\int_S \omega = \int_{f^{-1}(S)} f_* \omega
% \]
% with $f: N \to M$ being an embedding with $\Img f = S$.
% \end{definition}
% \begin{remark}
% Since differential $0$-forms are just smooth functions and compact $0$-submanifolds are just finite
% collections of discrete points the integral of a $0$-form is just a weighted sum
% \[
% 	\int_S f = \sum_{p \in S} \epsilon(p) f(p)
% \]
% where $\epsilon(p)$ is the orientation of $p$.
% \end{remark}

% That this definition is independent from the choice of embedding follows from a similar argument as the independence of the integral from a
% change of coordinates. (TODO)

% After having defined integration in such a general context there is only one technicality left to state the
% fundamental theorem of calculus in the language of differential geometry. It has got to to with the fact
% that the closed interval $[a,b]$ is not a manifold, but something we did not define yet: A smooth manifold
% with boundary. Since this concept is not going to play any further role in the rest of this thesis except
% for this motivational excourse we are going to be a bit sloppy with its definition and won't talk about all the ramifications of it.
% Without further ado:
% \begin{definition}
% A topological manifold with boundary is a metrizable and separable topological space $X$ such that every
% point $p$ has an open neighborhood that is homeomorphic to an open subset of the $n$-th half-space i.e. $\mathbb{H}^n = [0, \infty) \times \RR^{n - 1}$
% \end{definition}
% This definition translates to the smooth case without much problems, in fact all of our
% definitions so far can be extended to the case of manifolds with boundaries (for details
% on this see TODO). One defines the boundary of a manifold with boundary as all the points
% that get mapped to $\{0\} \times \RR^{n - 1}$. This is an $(n-1)$-dimensional manifold which we denote by
% $\partial S$ and in the special case of an oriented manifold the boundary has a canonical orientation
% induced by the orientation of the whole manifold.

% \begin{example}
% $[0,1]$ with the standard orientation is an oriented manifold with boundary whose
% boundary consists of the points $0$ and $1$ with $0$ being negatively oriented and
% $1$ being positively oriented.
% \end{example}

% \subsection{The exterior derivative}
% We are finally able to state the fundamental theorem of calculus in the language of
% differential geometry:
% \begin{lemma}
% Let $M$ be a smooth manifold. Given a smooth function $f: M \to \RR$ and a compact 1-submanifold
% $S \subseteq M$ it holds:
% \[
% 	\int_S \mathrm{d}f = \int_{\partial S} f
% \]
% \end{lemma}
% The fundamental theorem of calculus is just a special case of this:
% \begin{example}
% $[0,1] \subseteq \RR$ is a compact 1-submanifold with boundary of $\RR$ and the fundamental theorem of
% calculus states that for any smooth function $f: \RR \to \RR$:
% \[
% 	\int_{[0,1]} \rmd f = f(1) - f(0)
% 		= \int_{{-\{0\}}\,\cup\,{+\{1\}}} f
% 		= \int_{\partial [0,1]} f
% \]
% \end{example}
% The advantage of this formulation of this statement is that it can be generalized to higher dimensions
% to define a "derivative" of higher differential forms i.e. we want a linear map $\rmd$ (which we are
% going to call \textbf{exterior derivative}), that maps $k$-forms to $(k + 1)$-forms such that for any
% $\omega \in \Omega^k(M)$ and any compact $(k + 1)$-submanifold $S$ with boundary:
% \[
% 	\int_S \rmd \omega = \int_{\partial S} \omega
% \]
% While this is actually enough to specify $\rmd$ (which one can see properly executed
% in \cite{yang_exterior_nodate}) following this line of reasoning is a bit too long for this thesis, which is
% why we are going to define $\rmd$ (the "exterior derivative") in more algebraic terms
% and then note that this property is fulfilled. But nonetheless it is helpful to have
% this geometric interpretation in mind, not just as a property that the exterior derivative
% happens to fulfill, but as a definition.

% But now for the more workable though abstract definition:
% \begin{definition}
% Let $M$ be a smooth manifold and $\omega \in \Omega^k(M)$ a differential $k$-form.
% Then we define $\rmd \omega$ as the unique differential $k+1$-form such that for any $p \in M$ and
% chart $(x, \U)$ around $p$:
% \[
% 	\rmd \omega(p)
% 		= \sum_{i_1 < \dots < i_k} \sum_{j = 1}^n \left(\frac{\partial}{\partial x^j}\omega_{i_1 \dots i_k}(p) \right) \rmd x^j \wedge \rmd x^{i_1} \dots \wedge \rmd x^{i_k}
% \]
% \end{definition}
% Since this only defines the exterior derivative on the level of local coordinates it has
% to be ensured that $\rmd \omega$ is well-defined i.e. coordinate independent.

% TODO

% - Elemental Properties of exterior derivative
% - Refer back to introduction

\section{De Rham Cohomology}
We can now finally combine and generalize our introductory examples in the definition of de Rham cohomology:
% At this time it might be tempting to define the de Rham cohomology just as the kernel of $\rmd$, but that this
% works in dimension zero is just a happy little accident – one sees that by considering our next case: The
% conservativity of vector fields. In the introduction we talked about an interconnection between a space being
% simple connected and the existence of non-conservative vector fields with vanishing curl. Originally we defined a
% vector field $f$ on $\RR^3$ as just a map from $\RR^3$ to $\RR^3$. One can turn such a vector field into a
% differential 1-form as follows:
% \begin{align*}
% 	\omega = f_1 \dx{1} + f_2 \dx{2} + f_3 \dx{3}
% \end{align*}
% We can calculate the exterior derivative of this vector field:
% \begin{align*}
% 	\rmd \omega &=
% 		  \left(\pdx[f_1]{2} \right) \dx{2}\dx{1}
% 		+ \left(\pdx[f_1]{3} \right) \dx{3}\dx{1}
% 		+ \left(\pdx[f_2]{1} \right) \dx{1}\dx{2}
% 		+ \left(\pdx[f_2]{3} \right) \dx{3}\dx{2} \\
% 		&\phantom{=}\,+ \left(\pdx{1} f_3\right) \dx{1}\dx{3}
% 		+ \left(\pdx{2} f_3\right) \dx{2}\dx{3} \\
% 		&= \left( \pdx[f_2]{1} - \pdx[f_1]{2} \right) \dx{1}\dx{2}
% 			+ \left( \pdx[f_3]{2} - \pdx[f_2]{3} \right) \dx{2}\dx{3} \\
% 		&\phantom{=}\,+ \left( \pdx[f_1]{3} - \pdx[f_3]{1} \right) \dx{3}\dx{1}
% \end{align*}
% This vanishes if and only if the curl of $f$ vanishes, hence we are once against interested in the kernel of $\rmd$,
% but we also want to conclude the "boring" cases, where $f$ is actually conservative. These are the cases where
% there is a potential $F$ with derivative $\nabla F = f$. But taking the vector fields with curl as a 1-form one
% may rephrase this as there being a 0-form $F$ such that $\rmd F = \omega$.  Thus we are actually interested in
% the kernel of $\rmd$ modulo the image of $\rmd$.

% None of this prior discussion depended on any special cases, hence we may define:
\begin{definition}
Let $M$ be a smooth manifold. We define the \textbf{de Rham cohomology of $M$} as the graded
algebra:
\[
	H^*_\dr(M) \coloneqq \bigoplus_{k = 0}^\infty \ker(\mathrm{d}^k)/\Img(\mathrm{d}^{k - 1})
\]
\end{definition}

As already discussed, $H^0_\dr(M)$ consists of the locally constant functions on $M$, hence we can already
calculate it for any space:
\begin{theorem}
For any smooth manifold $M$:
\[
	H^0_\dr(M) = \bigoplus_{\pi_0(M)} \RR
\]
where $\pi_0(M)$ consists of the connected components of $M$.
\end{theorem}

Another helpful statement also follows immediately from the definition:
\begin{lemma}
Let $M$ be a smooth $n$-manifold. Then for every $k > n$:
\[
	H^k_\dr(M) = 0
\]
\end{lemma}
\begin{proof}
For $k > n$ the only skew-symmetric $k$-form on a $n$-dimensional vector space is $0$, hence
$\Omega^k(M) = 0$, which implies $H^k_\dr(M) = 0$.
\end{proof}

Explicit computation is now also possible:
\begin{lemma}
	The de Rham cohomology of $\RR$ is:
	\begin{align*}
		H^k_\dr(\RR) = \begin{cases}
			\RR &\quad\text{if } k = 0 \\
			0   &\quad\text{otherwise}
		\end{cases}
	\end{align*}
\end{lemma}
\begin{proof}
Since $\RR$ is connected $H^0(\RR) = \RR$ and for $k > 1$ it is already known that $H^k(\RR) = 0$. It remains
to show that $H^1(\RR)$ is also trivial by using the fundamental theorem of calculus. Since $\RR$ has global
canonical coordinates $(\RR, x)$ every 1-form $\omega \in \Omega^1(\RR)$ can be written as
\[
	\omega = f \rmd x
\]
Then define:
\[
	F(x) \coloneqq \int_0^x f \,\rmd \lambda_1
\]
By the fundamental theorem of calculus this is a smooth map on $\RR$ and its exterior derivative is:
\[
	dF(x) = \frac{\partial}{\partial x} F \rmd x =  f \rmd x = \omega
\]
Hence every 1-form on $\RR$ is exact, ensuring that $H^1(\RR) = 0$.
\end{proof}
\subsection{The Mayer-Vietoris Sequence}
With our current knowledge it seems impossible to calculate the de-Rham cohomology for anything but the
simplest of spaces since it involves solving differential equations. Luckily there is major computational
tool making these calculations much easier by connecting the cohomology of a space to the cohomologies of
its components. Concretely, given two open set $\U, \V$ that cover a smooth manifold $M$, there is a long
exact sequence, called the \textbf{Mayer-Vietoris sequence} between the cohomologies of
$\U, \V, \U \amalg \V$ and $M$.

But before we can define this sequence we have to talk about how smooth maps $f: M \to N$ induce maps
between $\Omega^*(M)$ and $\Omega^*(N)$:
\begin{definition}
Let $M,N$ be smooth manifolds, $f: M \to N$ a smooth map. Then we define
$f^*: \Omega^*(N) \to \Omega^*(M)$ as the linear function that maps $k$-forms to $k$-forms and is given by
\[
	(f^* \omega)(p)(v_1, \dots, v_n) = \omega(f(p))(Tf(v_1), \dots, Tf(v_n))
\]
We call $f^* \omega$ the \textbf{pullback of $\omega$ via $f$}.
\end{definition}
This commutes with function composition:
\begin{theorem}
For smooth maps $f: M \to N, g: N \to O$ it holds
\[
	(f \circ g)^* = g^* \circ f^*
\]
\end{theorem}
\begin{proof}
Proving this amounts to just writing down the definition of the pullback:
\begin{align*}
	(f \circ g)^*\omega)(p)(v_1, \dots, v_n) 
		&= \omega((f \circ g)(p))(T(f \circ g)(v_1), \dots, T(f \circ g)(v_n))\\
		&= \omega(f(g(p)))(Tf(Tg(v_1)), \dots, Tf(Tg(v_n)))\\
		&= ((g^* \circ f^*)\omega)(p)(v_1, \dots, v_n)
\end{align*}
\end{proof}
As an interesting consequence of this statement we get that the pullback of any diffeomorphism $f: M \to N$ is
an isomorphism since $f^* \circ \left(f^{-1}\right)^* = (f \circ f^{-1})^* = \mathrm{Id}^*$, which by writing
down its definition, is easily seen to be the identity map on differential forms.

This also commutes with the wedge product:
\begin{theorem}
Let $f: M \to N$ a smooth map, then for any $\alpha, \beta \in \Omega^*(N)$:
\[
	f^* (\alpha \wedge \beta)
		= f^* \alpha \wedge f^* \beta
\]
\end{theorem}
\begin{proof}
By (bi-)linearity and since both pullbacks and wedge products are local it is enough to prove this for
terms $f \rmd x^I, g \rmd x^J$ with $I = (i_1, \dots, i_k), J = (j_1, \dots, j_\ell)$ being multi-indices.
Let $\phi: M \to N$ be a smooth map, then
\begin{align*}
	\phi^* (f \rmd x^I \wedge g \rmd x^J)(p)(v_1, \dots, v_{k + \ell})
		&= (f \rmd x^I \wedge g \rmd x^J)(\phi(p))(T\phi(v_1), \dots, T\phi(v_{k + \ell})) \\
		&= (f \rmd x^I)(\phi(p))(T\phi(v_1), \dots, T\phi(v_{k}))\\
		&\phantom{=\,} {} \cdot (g \rmd x^J)(\phi(p))(T\phi(v_{k + 1}), \dots, T\phi(v_{k + \ell})) \\
		&= (\phi^*(f \rmd x^I) \wedge \phi^* (g \rmd x^J))(p)(v_1, \dots, v_{k + \ell})
		 \qedhere
\end{align*}
\end{proof}
	
\begin{remark}
For any $n$-manifold $M$ every open subset $\U \subseteq M$ is also an $n$-manifold with the subset topology
by restricting the charts of $M$ to $\U$. Similarly the disjoint union of two smooth manifolds can also be
turned into a smooth manifold with the disjoint union topology by taking the disjoint union of all charts.

Any differential $k$-form on $\U \amalg \V$ then determines $k$-forms on $\U$ and $\V$ and vice versa,
resulting in a natural isomorphism between $\Omega^*(\U) \oplus \Omega^(\V)$ and $\Omega^*(\U \amalg \V)$.
\end{remark}

This exact sequence is derived from the inclusion maps:
\begin{center}
% https://tikzcd.yichuanshen.de/#N4Igdg9gJgpgziAXAbVABwnAlgFyxMJZABgBpiBdUkANwEMAbAVxiRAFkQBfU9TXfIRQBGclVqMWbADrSAqgAJZAYwhoATtCXSAat14gM2PASIAmMdXrNWiELMUq6abXq7iYUAObwioAGaaALZIFiA4EEiiIAxYYLYgcBCxUCDUABYwdKmIYEwMDNQ4dFgMbJDxaYnpWP44SAC00dZSdlgA+rJuBoEQIYhhEVHUcDV1wzF0AEYwDAAK-CZCIOpYXun1VpIJHQ76AcETQ4hkEjZsAFbcFFxAA
\begin{tikzcd}
M & \U \coprod \V \arrow[l, "j"] & \U \cap \V \arrow[l, "i_\V", shift left] \arrow[l, "i_\U"', shift right]
\end{tikzcd}
\end{center}
With $j$ being the disjoint union of the inclusions from $\U$ and $\V$ into $M$ and $i_\U$ being the
inclusion of $\U \cap \V$ into $\U$ (and then into the disjoint union), similar with $i_\V$. All of these
are smooth and hence induce a sequence of restrictions of differential forms:
\begin{equation}
% https://tikzcd.yichuanshen.de/#N4Igdg9gJgpgziAXAbVABwnAlgFyxMJZABgBpiBdUkANwEMAbAVxiRAB12B5AWxgHM6APQBUACgCyAShABfUuky58hFAEZyVWoxZtOvAcPGcAqlIAEnAEZZ+ENMziXufQaLGcAajPmLseAiIAJk1qemZWRA4XQ3dTZwBjOjRnbzktGCh+eCJQADMAJwgeJA0QHAgkEJAGLDBIkDgIWqgQagALGDpWxDAmBgZqHDosBjZIerbG9qw8nFKwnQasAH0vUSmGOisYBgAFJQDVEALbdvnfEELihfLKxGq4GbmkAFoyrZ39w5U2BhgXosImxVqYNpdriVEGQ7rdwroogArcEUWRAA
\begin{tikzcd}
\Omega^*(M) \arrow[r, "j^*"] & \Omega^*(\U) \bigoplus \Omega^*(\V) \arrow[r, "i_\V^*"', shift right] \arrow[r, "i_\U^*", shift left] & \Omega^*(\U \cap \V)
\end{tikzcd}
\end{equation}
This sequence being exact can be seen as the statement that the cohomology of a space is the cohomology of
its parts, modulo their overlap.

This seems reasonable, but has to be modified a bit, since the current sequence of restrictions would map
a differential form  $\omega$ on $M$ to $2\omega$ on $\U \cap \V$ which of course does not have to be zero. 
Adding some signs fixes this problem:
\begin{definition}[Short Mayer-Vietoris Sequence]
Let $M$ be a smooth manifold and $\U, \V \subseteq M$ an open cover of $M$. Then we define the
\textbf{short Mayer-Vietoris Sequence} as
\begin{equation}
	% https://tikzcd.yichuanshen.de/#N4Igdg9gJgpgziAXAbVABwnAlgFyxMJZARgBoAGAXVJADcBDAGwFcYkQAdDgeQFsYA5vQB6AKgAUAWQCUIAL6l0mXPkIoATBWp0mrdlz6CRErgFVpAAi4AjLAIhoWcKz35Cx4rgDVZCpdjwCIgBmLRoGFjZETlcjDzMXAGN6NBcfeUUQDADVInIwnUj2cgz-FSCUABYCiL1okr8s5UC1ZE1ibVqokE8OCDd6UhccemZfTOzy1tCO8N1urhHmCwBaF36jeW0YKAF4IlAAMwAnfqQyEBwIJE0QRiwwbrgIe6gQGgALGHo3xDBmRiMGgjLCMdiQR7vEBwD5YQ44JD5Qp1EBYMQAfW8qwsaNEmI4pihjHo1hgjAACs1ctFGDB4aUQCczogkVdznMitEAFZiBlM3hIUKXa4sxr8m7AkWVMWnAWIACskqQADZgfRQexeCk4Gy5JQ5EA
	\begin{tikzcd}
	0 \arrow[r] & \Omega^*(M) \arrow[r, "j^*"] & \Omega^*(\U) \bigoplus \Omega^*(\V) \arrow[r, "i^*_\V - i^*_\U"] & \Omega^*(\U \cap \V) \arrow[r] & 0 \\
	            &                              & {(\omega, \tau)} \arrow[r, maps to]                              & \tau - \omega                  &  
	\end{tikzcd}
	% \label{cd:de-rham:mv-ses}
\end{equation}
\end{definition}
The exactness of this sequence is important enough to be stated as a theorem:
\begin{theorem}
The short Mayer-Vietoris sequence is exact.
\end{theorem}
%TODO: I need to talk about partitions of unity somewhere
\begin{proof}
$j^*(\omega)$ can only be zero if the restriction of $\omega$ to both $\U$ and $\V$ is zero. And since
$\U, \V$ cover $M$ only zero gets mapped to zero by $j^*$, thus it is an injective map.

The kernel of $i^*_\V - i^*_\U$ consists exactly of the differential forms that agree on $\U \cap \V$,
hence precisely those which are the restriction of a differential form $\omega$ on $M$ and therefore
$\Img(j^*) = \ker(i^*_\V - i^*_\U)$.

It remains to show that $i^*_\V - i^*_\U$ is surjective, which we are going to do by
constructing a right-inverse. Let $\omega \in \Omega^*(\U \cap \V)$ and $\rho_\U, \rho_V$ a
partition of unity subordinate to $\U, \V$. Now $\rho_\mathcal{V} \omega$ (taken as pointwise
multiplication) can be zero-extended to $\U$, as can $\rho_\mathcal{U} \omega$ to $\V$. We then
have:
\begin{equation*}
	(i^*_\V - i^*_U)(-\rho_\V \omega, \rho_\U \omega) = (\rho_\U + \rho_\V)\omega = \omega
\end{equation*}
Having a right inverse implies surjectivity, therefore the whole sequence is exact.
%TODO: Picture
\end{proof}

To turn this into a sequence of cohomologies we need to define a way to let maps on differential forms induce
maps on cohomology. While this isn't possible in the case of general maps, one can do it for pullbacks by
smooth maps:
\begin{definition}
Let $f: M \to N$ be a smooth map. We define the map $\tilde{f}^*: H^*_\text{DR}(N) \to H^*_\text{DR}(M)$ as
\[
	\tilde{f}^*([\alpha]) = [f^*(\alpha)]
\]
\end{definition}
Note that similar as with the pullback $\reallywidetilde{f^* \circ g^*} = \tilde{f}^* \circ \tilde{g}^*$ and
$\reallywidetilde{\mathrm{Id}}^* = \mathrm{Id}$, hence by the same argument as earlier, isomorphism of the
space of differential forms descend to isomorphisms of cohomologies and in particular diffeomorphic manifolds
have isomorphic cohomology. 

It still has to be proven that $\tilde{f}^*$ is well-defined, but we are going to delay this proof until we
have put de Rham cohomology in a more general algebraic framework. But having defined how these maps descend
onto cohomology we can now turn the Mayer-Vietoris sequence into one of cohomologies:
\begin{theorem}[Mayer-Vietoris sequence]
	Let $M$ be a smooth manifold and $\U, \V \subseteq M$ open subsets, such that $M = \U \cup \V$.

	The short Mayer-Vietoris sequence
	% https://q.uiver.app/#q=WzAsMyxbMCwwLCJcXE9tZWdhXiooTSkiXSxbMSwwLCJcXE9tZWdhXiooXFxVKSBcXGJpZ29wbHVzIFxcT21lZ2FeKihcXFYpIl0sWzIsMCwiXFxPbWVnYV4qKFxcVSBcXGNhcCBcXFYpIl0sWzAsMSwial4qIl0sWzEsMiwiaV9cXFZeKiAtIGlfXFxVXioiXV0=
	\[\begin{tikzcd}
		{\Omega^*(M)} & {\Omega^*(\U) \bigoplus \Omega^*(\V)} & {\Omega^*(\U \cap \V)}
		\arrow["{j^*}", from=1-1, to=1-2]
		\arrow["{i_\V^* - i_\U^*}", from=1-2, to=1-3]
	\end{tikzcd}\]
	induces a long exact sequence sequence in cohomology, called \textbf{Mayer-Vietoris sequence}:
	\begin{equation}
		% https://tikzcd.yichuanshen.de/#N4Igdg9gJgpgziAXAbVABwnAlgFyxMJZARgBoAGAXVJADcBDAGwFcYkQAJAPQGsAKALIBKEAF9S6TLnyEUAJgrU6TVu278AOhoCqQgARaARlgDmENCzh71fLQDUR4ydjwEiAZkU0GLNok68tjoGGgDG9GghDmISIBguMkTkXsq+7FpQEDgITnFSrrIkpMRKPqr+3MA8egDUesSigo6x8dJu8sWlKn4BVbX1jVq6IcZmFsxWldV1DUHRua0FHp3e3WpcfTODwVrhkfbNzm2FACwrqeUgGVk5SjBQJvBEoABmAE4QALZIZCA4EEgFCBGFgwD04BAQVAQDQABYwejQxBgZiMRg0HD0LCMdiQMEwkBwWFYF44JDJYH0QwwRgABXyiX8jBgpJirw+30QFP+P1y7y+SE8fwBXL5HKQADYMSKAOxigWIM7CpAAVnlnJV0slokooiAA
		\begin{tikzcd}[column sep=small]
			\usetikzlibrary{fadings}
			&\arrow[
				d,
				"\dots"{description},
				dash pattern=
					on 0.5pt
					off 8pt
					on 1pt
					off 7pt
					on 2pt
					off 6pt
					on 4pt
					off 5pt
					on 6pt
					off 4pt
					% on 8pt
					% off 3pt
					% on 10pt
					% off 1pt
					on 100pt,
				rounded corners,
				to path={
					([xshift=3ex]\tikztostart.west)
					-- (\tikztostart.west)
					-- ([xshift=-3ex]\tikztostart.west)
					-| ([xshift=-3ex]\tikztotarget.west)
					-- (\tikztotarget)
				}
				]
			&\\[-1.6em]
			& H^k_\dr(M)
				\arrow[r]
			& H^k_\dr(\U) \bigoplus H^k_\dr(\V)
				\arrow[r]
				% \arrow[u, ""{coordinate, name=U}]
				\arrow[d, phantom, ""{coordinate, name=Z}]
			& H^k_\dr(\U \cap \V)
				\arrow[
					dll,
					"d^*"{pos=1.0, description},
					rounded corners,
					to path={
						-- ([xshift=2ex]\tikztostart.east)
						|- (Z) [near end]\tikztonodes
						-| ([xshift=-2ex]\tikztotarget.west)
						-- (\tikztotarget)
					}]
			&\\
            & H^{k + 1}(M)
				\arrow[r]
			& H^{k + 1}(\U) \bigoplus H^{k + 1}(\V)
				\arrow[d, phantom, ""{coordinate, name=D}]
				\arrow[r]
			& H^{k + 1}(\U \cap \V)
				\arrow[
					d,
					""{description},
					dash,
					dash pattern=
						on 0.5pt
						off 8pt
						on 1pt
						off 7pt
						on 2pt
						off 6pt
						on 4pt
						off 5pt
						on 6pt
						off 4pt
						% on 8pt
						% off 3pt
						% on 10pt
						% off 1pt
						on 100pt,
					rounded corners,
					to path={
						([xshift=-6ex]\tikztotarget.east)
						-| ([xshift=2ex]\tikztostart.east)
						-- (\tikztostart.east)
					}
				]
			&\\[-1.3em]
			& {} & {} & {}
		\end{tikzcd}
	\end{equation}
\end{theorem}

We are going to prove this statement by first developing the more general algebraic framework of 
cohomologies, called homological algebra. Homological algebra gives us the ability to study abstract cochain
complexes (which can be seen as a generalization of the differential forms on a manifold) and their cohomology,
which is going to result in a purely algebraic way to prove the exactness of the Mayer-Vietoris sequence.

Without further ado, lets define its basic building blocks:
\begin{definition}[Cochain Complex]
A \textbf{cochain complex} is a graded vector space $A^*$ together with maps
\begin{align*}
	\rmd^k: A^k \to A^{k + 1}
\end{align*}
which together define a map $\rmd: A^* \to A^*$ in the obvious way, that fulfills:
\begin{align*}
	\rmd \circ \rmd = 0
\end{align*}
The map $\mathrm{d}$ will usually be called \textbf{differential} or
\textbf{coboundary operator} of $A^*$.
\end{definition}
The obvious example of this is the space of differential forms on a manifold with the exterior derivate as
its differential.

Given an abstract cochain complex, one defines its cohomology in the obvious way:
\begin{definition}[Cohomology]
Let $A^*$ be a cochain complex. We define the \textbf{cohomology} of $A^*$ as the graded
vector space:
\begin{align*}
H^k(A^*) \coloneqq \ker(\rmd^k) / \Img(\rmd^{k - 1})
\end{align*}
The elements of $\ker(\rmd^k)$ are called $k$-cocycles of $A^*$ and the elements of $\Img(\rmd^{k - 1})$
$k$-coboundaries of $A^*$.
\end{definition}
De Rham cohomology is an obvious example of this construction.

No algebraic discussion would be complete without talking about the interesting maps between our objects.
In the case of cochain complexes we are mostly interested in those maps that descend to homology. These
are going to be the cochain maps:
\begin{definition}[Cochain Map]
Let $(A^*, \rmd_A), (B^*, \rmd_B)$ be cochain complexes and $f^k: A^k \to B^k$ a vector
space homomorphism for every $k$. We call $f^k$ \textbf{cochain map} if it commutes with the differentials
i.e. if the following diagram commutes:
\[\begin{tikzcd}
	{A^*} & {A^{* + 1}} \\
	{B^*} & {B^{* + 1}}
	\arrow["{\mathrm{d}_A}", from=1-1, to=1-2]
	\arrow["{f^*}"', from=1-1, to=2-1]
	\arrow["{f^{* + 1}}"', from=1-2, to=2-2]
	\arrow["{\mathrm{d}_B}"', from=2-1, to=2-2]
\end{tikzcd}\]
(this introduces a common abuse of notation: Writing $A^*, A^{* + 1}$ to mean that these can be
replaced by $A^k, A^{k + 1}$ for any $k$.
)
\end{definition}
These maps now induce maps on cohomology in a familiar way:
\begin{definition}
Let $(A^*, \rmd_A), (B^*, \rmd_B)$ be cochain complexes and $f^*: A^* \to B^*$ a cochain map.
We define the map $\tilde{f}^*: H^*(A^*) \to H^*(B^*)$ as
\[
	\tilde{f}^*([\alpha]) = [f^*(\alpha)]
\]
\end{definition}
Note that by abuse of notation we will often use $f^*$ for both maps, so long as the meaning is clear from
context.

It still has to be proven that this induced map is actually well-defined:
\begin{proof}
We have to proof that given two $a, \tilde{a} \in \ker(\rmd_A^k)$ such that $[\tilde{a}] = [a]$ it holds
that $[f^*(a)] = [f^*(\tilde{a})]$. Since $[\tilde{a}] = [a]$ there exists $\hat{a} \in A^{k - 1}$ such that
$a = \tilde{a} + \rmd_A \hat{a}$. Then:
\[
	[f^*(a)]
		= [f^*(\tilde{a} + \rmd_A \hat{a})]
		= [f^*(\tilde{a}) + \rmd_B f^*(\hat{a})]
		= [f^*(\tilde{a})]
\]
Therefore $\tilde{f}^*$ is well-defined.
\end{proof}
As an immediate consequence we can now proof that pullbacks of smooth maps descend to cohomology by proving
that these maps are chain maps!
\begin{lemma}
Let $M, N$ be smooth manifolds and $f: M \to N$ a smooth map. Then the map $f^*: \Omega^*(N) \to \Omega^*(M)$
is a cochain map.
\end{lemma}
\begin{proof}
We have to show that $f^*$ commutes with the exterior derivative i.e. $f^* \rmd = \rmd f^*$. We are going to
show this using induction on the degree of differential forms.

To show this we first need to explicitly show how $Tf_p$ acts on an tangent vector $v$. To do this we
first have to pick local coordinates $(\U, x)$ around $p$ and $(\V, y)$ around $f(p)$, then we denote the
components of $v$ with respect to $x$ as $v^1, \dots, v^m$. Then we can calculate:
\[
	T f_p(v)
		= D(y \circ f \circ x^{-1})(p)(v)
		= \begin{pmatrix}
			\sum_{j = 1}^m \frac{\partial}{\partial x^j} f_1(p) v^j \\
			\vdots \\
			\sum_{j = 1}^m \frac{\partial}{\partial x^j} f_n(p) v^j
		\end{pmatrix}
\]
where $f_1, \dots, f_n$ denote the components of $f$ with respect to $y$ and
$\frac{\partial}{\partial x^i} f_j$ is defined as $\partial_i (f_j \circ x)$.

Now we can properly start the proof, let $\alpha \in \Omega^0(N)$ be a 0-form and $(\U, x), (\V, y)$
coordinates as above. By noting that $\rmd y^i$ is the $i$-th component with respect to $y$, we can calculate:
\begin{align*}
	f^* \rmd \alpha (p)(v)
		&= f^*\left( \sum_{j = 1}^n \frac{\partial}{\partial y^j} \alpha \rmd y^j\right)(p)(v) \\
		&= \sum_{j = 1}^n \left( \frac{\partial}{\partial y^j} \alpha\right)(f(p)) \rmd y^j(Tf_p(v)) \\
		&= \sum_{j = 1}^n \left( \frac{\partial}{\partial y^j} \alpha\right)(f(p)) 
			\left( \sum_{i = 1}^m \frac{\partial}{\partial x^i} f_j(p) v^i \right) \\
		&= \sum_{i = 1}^m \sum_{j = 1}^n \left(\frac{\partial}{\partial y^j} \alpha\right)(f(p))
			\frac{\partial}{\partial x^i}f_j(p) v^i
\end{align*}
In the reverse direction, by the chain rule:
\begin{align*}
	\rmd f^* \alpha (p)(v)
		&= \sum_{i = 1}^m \frac{\partial}{\partial x^i} (\alpha \circ f)(p) \rmd x^i(v) \\
		&= \sum_{i = 1}^m \sum_{j = 1}^n
			\left(\frac{\partial}{\partial y^j}  \alpha \right)(f(p))
			\frac{\partial}{\partial x^i} f_j(p)
			v^i
\end{align*}
Hence $\rmd$ and $f^*$ commute on 0-forms. Now assume that $\rmd$ and $f^*$ commute on $k-1$-forms. Because of
linearity proving that they also commute on $k$-forms only requires showing that for terms of the form
$\alpha = g \rmd x^{i_1} \wedge \dots \wedge \rmd x^{i_k}$. Recalling that pullbacks commute with the wedge
product it holds:
\begin{align*}
	\rmd f^* \alpha 
		&= \rmd f^*(g \rmd x^{i_1} \wedge \dots \wedge x^{i_k}) \\
		&= \rmd( f^*(g \rmd x^{i_1} \wedge \dots \wedge x^{i_{k - 1}})
			\wedge f^* (\rmd x^k)) \\
		&= \rmd f^* (g \rmd x^{i_1} \wedge \dots \wedge x^{i_{k - 1}}) \wedge f^*(\rmd x^k) \\
		&\phantom{=\,} + (-1)^{k - 1} f^*(g \rmd x^{i_1} \wedge \dots \wedge x^{i_{k - 1}}) \wedge \rmd f^*(\rmd x^{i_k}) \\
		&= f^* \rmd (g \rmd x^{i_1} \wedge \dots \wedge x^{i_{k - 1}}) \wedge f^*(\rmd x^k) \\
		&\phantom{=\,} + (-1)^{k - 1} f^*(g \rmd x^{i_1} \wedge \dots \wedge x^{i_{k - 1}}) \wedge f^*
			\underbrace{\rmd (\rmd x^{i_k})}_{= 0} \\
		&= f^* \rmd (g \rmd x^{i_1} \wedge \dots \wedge \rmd x^{i_k}) \\
		&= f^* \rmd \alpha
\end{align*}
Hence $f^*$ and $\rmd$ commute on every differential form.
\end{proof}

The notion of (short) exact sequences seamlessly translates to cochain complexes, finally allowing us to
state our theorem in all its generality:
\begin{theorem}
Let
% https://q.uiver.app/#q=WzAsNSxbMSwwLCJBXioiXSxbMCwwLCIwIl0sWzIsMCwiQl4qIl0sWzMsMCwiQ14qIl0sWzQsMCwiMCJdLFsxLDBdLFswLDIsImZeKiJdLFsyLDMsImdeKiJdLFszLDRdXQ==
\[\begin{tikzcd}
	0 & {A^*} & {B^*} & {C^*} & 0
	\arrow[from=1-1, to=1-2]
	\arrow["{f^*}", from=1-2, to=1-3]
	\arrow["{g^*}", from=1-3, to=1-4]
	\arrow[from=1-4, to=1-5]
\end{tikzcd}\]
be a short exact sequence of chain complexes. Then there exists a connecting morphism
$\rmd^*: C^* \to A^{* + 1}$ such that the sequence
% https://q.uiver.app/#q=WzAsNixbMSwwLCJIXiooQV4qKSJdLFsyLDAsIkheKihCXiopIl0sWzMsMCwiSF4qKENeKikiXSxbNCwwLCJIXnsqICsgMX0oQV4qKSJdLFs1LDAsIlxcZG90cyJdLFswLDAsIlxcZG90cyJdLFswLDEsImZeKiJdLFsxLDIsImdeKiJdLFsyLDMsIlxcbWF0aHJte2R9XioiXSxbMyw0XSxbNSwwXV0=
\[\begin{tikzcd}[column sep=small]
	\dots & {H^*(A^*)} & {H^*(B^*)} & {H^*(C^*)} & {H^{* + 1}(A^*)} & \dots
	\arrow[from=1-1, to=1-2]
	\arrow["{f^*}", from=1-2, to=1-3]
	\arrow["{g^*}", from=1-3, to=1-4]
	\arrow["{\mathrm{d}^*}", from=1-4, to=1-5]
	\arrow[from=1-5, to=1-6]
\end{tikzcd}\]
is exact.
\end{theorem}
\begin{proof}
We first define a candidate for the morphism, prove that it is well-defined and then that it indeed makes
the sequence exact. Let $[c] \in H^k(C^*)$. Because $g^k$ is surjective one can pick $b \in B^k$ such that
$g^k(b) = c$. As $g^*$ is a cochain map we can put these in a commutative diagram with $\rmd_B b$ and
$\rmd_C c$ (which is by definition zero):
% https://q.uiver.app/#q=WzAsNCxbMCwwLCJiIl0sWzAsMSwiXFxtYXRocm17ZH1fQiBiIl0sWzEsMSwiMCJdLFsxLDAsImMiXSxbMCwxLCJcXG1hdGhybXtkfV9CIiwyXSxbMSwyLCJnXntrICsgMX0iLDJdLFswLDMsImdeayJdLFszLDIsIlxcbWF0aHJte2R9X0MiXV0=
\[\begin{tikzcd}[cramped]
	b & c \\
	{\mathrm{d}_B b} & 0
	\arrow["{g^k}", from=1-1, to=1-2]
	\arrow["{\mathrm{d}_B}"', from=1-1, to=2-1]
	\arrow["{\mathrm{d}_C}", from=1-2, to=2-2]
	\arrow["{g^{k + 1}}"', from=2-1, to=2-2]
\end{tikzcd}\]
Thus $g(\mathrm{d}_B b) = 0$ and by exactness there is a unique $a \in A^{k + 1}$ such that
$f(a) = \rmd_B b$. We then define $\rmd^*[c]$ as $[a]$.

Since $a$ depends on choices of representative $c$ and preimage $b$ one has to prove that $\rmd^*$ is
nonetheless well-defined. We are going to start with proving that the choice of $b$ does not matter by choosing
$\tilde{b}$ such that $g^k(\tilde{b}) = c$. Then $g^k(b - \tilde{b}) = 0$ and by exactness
there exists a unique $\hat{a}$ such that $f^k(\hat{a}) = b - \tilde{b}$. Putting this into a commutative:
% https://q.uiver.app/#q=WzAsNixbMSwwLCJiIC0gXFx0aWxkZXtifSJdLFsxLDEsIlxcbWF0aHJte2R9X0IoYiAtIFxcdGlsZGV7Yn0pIl0sWzIsMSwiMCJdLFsyLDAsIjAiXSxbMCwwLCJcXGhhdHthfSJdLFswLDEsIlxcbWF0aHJte2R9X0FcXGhhdHthfSJdLFswLDEsIlxcbWF0aHJte2R9X0IiLDJdLFsxLDIsImdee2sgKyAxfSIsMl0sWzAsMywiZ15rIl0sWzMsMiwiXFxtYXRocm17ZH1fQyJdLFs0LDAsImZeayJdLFs0LDUsIlxcbWF0aHJte2R9X0EiLDJdLFs1LDEsImZee2sgKyAxfSIsMl1d
\[\begin{tikzcd}[column sep=small]
	{\hat{a}} & {b - \tilde{b}} & 0 \\
	{\mathrm{d}_A\hat{a}} & {\mathrm{d}_B(b - \tilde{b})} & 0
	\arrow["{f^k}", from=1-1, to=1-2]
	\arrow["{\mathrm{d}_A}"', from=1-1, to=2-1]
	\arrow["{g^k}", from=1-2, to=1-3]
	\arrow["{\mathrm{d}_B}"', from=1-2, to=2-2]
	\arrow["{\mathrm{d}_C}", from=1-3, to=2-3]
	\arrow["{f^{k + 1}}"', from=2-1, to=2-2]
	\arrow["{g^{k + 1}}"', from=2-2, to=2-3]
\end{tikzcd}\]
This shows that $\rmd^*[g^k(b) - g^k(\tilde{b})] = [\rmd_A \hat{a}] = [0]$, thus both $b$ and $\tilde{b}$
result in the same equivalence class $[a]$.

Continuing with the independence on the choice of $c$ let $[\tilde{c}] \in H^k(C^*)$ such that
$[\tilde{c}] = [c]$. Then $[c - \tilde{c}] = 0$ which implies the existence of $\hat{c} \in C^{k - 1}$ such
that $\rmd_C \hat{c} = c - \tilde{c}$. By surjectivity of $g^{k - 1}$ there is then $\hat{b} \in B^{k - 1}$
such that $g^{k-1}(\hat{b}) = \hat{c}$. These can be put into a commutative diagram:
% https://q.uiver.app/#q=WzAsNyxbMSwxLCJcXG1hdGhybXtkfV9CIFxcaGF0e2J9Il0sWzEsMiwiMCJdLFsyLDIsIjAiXSxbMiwxLCJjIC0gXFx0aWxkZXtjfSJdLFsyLDAsIlxcaGF0e2N9Il0sWzEsMCwiXFxoYXR7Yn0iXSxbMCwyLCIwIl0sWzAsMSwiXFxtYXRocm17ZH1fQiIsMl0sWzEsMiwiZ157ayArIDF9IiwyXSxbMCwzLCJnXmsiXSxbMywyLCJcXG1hdGhybXtkfV9DIl0sWzQsMywiXFxtYXRocm17ZH1fQyJdLFs1LDQsImdee2sgLSAxfSJdLFs1LDAsIlxcbWF0aHJte2R9X0IiLDJdLFs2LDEsImZee2sgKyAxfSIsMl1d
\[\begin{tikzcd}
	& {\hat{b}} & {\hat{c}} \\
	& {\mathrm{d}_B \hat{b}} & {c - \tilde{c}} \\
	0 & 0 & 0
	\arrow["{g^{k - 1}}", from=1-2, to=1-3]
	\arrow["{\mathrm{d}_B}"', from=1-2, to=2-2]
	\arrow["{\mathrm{d}_C}", from=1-3, to=2-3]
	\arrow["{g^k}", from=2-2, to=2-3]
	\arrow["{\mathrm{d}_B}"', from=2-2, to=3-2]
	\arrow["{\mathrm{d}_C}", from=2-3, to=3-3]
	\arrow["{f^{k + 1}}"', from=3-1, to=3-2]
	\arrow["{g^{k + 1}}"', from=3-2, to=3-3]
\end{tikzcd}\]
Since it has been showed that $\mathrm{d}^*$ does not depend on the choice of $b$ this implies
$\rmd^*([c - \tilde{c}]) = 0$, thus $\rmd^*$ is independent of the choice of representative $c$ and in
particular well-defined.

It remains to show that the resulting sequence is exact. We start by proving that
$\Img(\tilde{f}^k) = \ker(\tilde{g}^k)$ i.e. that this equality descend to cohomology. It is easy
to show that $\Img(\tilde{f}^k) \subseteq \ker(\tilde{g}^k)$ since
\begin{align*}
	(g \circ f)([a]) = [(g \circ f)(a)] = [0] = 0
\end{align*}
To show that $\ker(\tilde{g}^k) \subseteq \Img(\tilde{f}^k)$ let $[b] \in H^k(B^*)$ such that
$g^k([b]) = [g^k(b)] = [0]$. This is the case if and only if $g^k(b) = \rmd_C c$ for some
$c \in C^{k - 1}$. By surjectivity of $g^{k - 1}$ there exists $\hat{b} \in B^{k - 1}$ such that
$g^{k - 1}(\hat{b}) = c$. Since $g^*$ is a chain map the 
following diagram commutes:
% https://q.uiver.app/#q=WzAsNCxbMCwwLCJcXGhhdHtifSJdLFsxLDAsImMiXSxbMSwxLCJnXmsoYikiXSxbMCwxLCJcXG1hdGhybXtkfSBcXGhhdHtifSJdLFswLDEsImdee2sgLSAxfSJdLFsxLDIsIlxcbWF0aHJte2R9X0MiXSxbMCwzLCJcXG1hdGhybXtkfV9CIiwyXSxbMywyLCJnXmsiLDJdXQ==
\[\begin{tikzcd}
	{\hat{b}} & c \\
	{\mathrm{d} \hat{b}} & {g^k(b)}
	\arrow["{g^{k - 1}}", from=1-1, to=1-2]
	\arrow["{\mathrm{d}_B}"', from=1-1, to=2-1]
	\arrow["{\mathrm{d}_C}", from=1-2, to=2-2]
	\arrow["{g^k}"', from=2-1, to=2-2]
\end{tikzcd}\]
Therefore $g^k(\mathrm{d}_B \hat{B}) = g^k(b)$ and in particular $g^k(b - \rmd_B \hat{b}) = 0$,
which implies by exactness of the original sequence that there exists $a \in A^k$ such that
$f^k(a) = b - \rmd_B \hat{b}$. Since $f^*$ is a chain map the following diagram commutes:
% https://q.uiver.app/#q=WzAsNCxbMCwwLCJhIl0sWzEsMCwiYiAtIFxcbWF0aHJte2R9X0IgXFxoYXR7Yn0iXSxbMSwxLCIwIl0sWzAsMSwiXFxtYXRocm17ZH1fQSBhIl0sWzAsMSwiZl5rIl0sWzEsMiwiXFxtYXRocm17ZH1fQiJdLFswLDMsIlxcbWF0aHJte2R9X0EiLDJdLFszLDIsImZee2sgKyAxfSIsMl1d
\[\begin{tikzcd}
	a & {b - \mathrm{d}_B \hat{b}} \\
	{\mathrm{d}_A a} & 0
	\arrow["{f^k}", from=1-1, to=1-2]
	\arrow["{\mathrm{d}_A}"', from=1-1, to=2-1]
	\arrow["{\mathrm{d}_B}", from=1-2, to=2-2]
	\arrow["{f^{k + 1}}"', from=2-1, to=2-2]
\end{tikzcd}\]
And since $f^{k + 1}$ is injective it follows that $\mathrm{d}_A a = 0$. Therefore $[a] \in H^k(A^*)$ and
by construction:
\[
	\tilde{f}^k([a]) = [f^k(a)] = [b - \rmd_B \hat{b}] = [b]
\]
Hence $\Img(f^k) = \ker(g^k)$.

Next we prove that $\Img(g^k) = \ker(\rmd^*)$. Let $[b] \in H^k(B^*)$, calculating $\rmd^*[g^k(b)]$ requires
us to choose a preimage of $g^k(b)$ under $g$, apply $\rmd_B$ to it and then find the unique preimage of this
under $f$. For the first step $b$ itself is the obvious choice, then $\rmd_B b = 0$ because
$[b] \in H^k(B^*)$ and this has $0$ as its unique preimage, hence $\rmd^*[g^k(b)] = 0$.

In the other direction, if $\mathrm{d}^* [c] = [0]$, then the calculation of $\rmd^*$ looks like this:
% https://q.uiver.app/#q=WzAsNSxbMCwxLCJcXG1hdGhybXtkfV9BIGEiXSxbMCwwLCJhIl0sWzEsMSwiXFxtYXRocm17ZH1fQiBiIl0sWzEsMCwiYiJdLFsyLDAsImMiXSxbMSwwLCJcXG1hdGhybXtkfV9BIiwyXSxbMCwyLCJmXntrICsgMX0iLDJdLFszLDIsIlxcbWF0aHJte2R9X0IiXSxbMyw0LCJnXmsiXV0=
\[\begin{tikzcd}
	a & b & c \\
	{\mathrm{d}_A a} & {\mathrm{d}_B b}
	\arrow["{\mathrm{d}_A}"', from=1-1, to=2-1]
	\arrow["{g^k}", from=1-2, to=1-3]
	\arrow["{\mathrm{d}_B}", from=1-2, to=2-2]
	\arrow["{f^{k + 1}}"', from=2-1, to=2-2]
\end{tikzcd}\]
Since $g^k \circ f^k = 0$ it follows that $g^k(b - f^k(a)) = c$ and since $f^*$ is a chain
map $\rmd_B f^k(a) = \rmd_B b$, thus:
\begin{align*}
\rmd_B(b - f^k(a)) = 0
\end{align*}
Hence $[c] \in \Img(\tilde{g}^k)$ and in particular $\Img(\tilde{g}^*) = \ker (\rmd^*)$.

The final step consists of proving that $\Img(\mathrm{d}^*) = \ker(f^*)$. Let $[c] \in H^k(C^*)$. By
construction of $\rmd^*$ there exist $a \in A^{k + 1}$ and $b \in B^k$ such that $\rmd^*[c] = [a]$ and
$f^{k + 1}(a) = \rmd_B b$. Therefore 
\[
	\tilde{f}^{k + 1}(\rmd^*[c])
		= \tilde{f}^{k + 1}([a])
		= [f^{k + 1}(a)]
		= [\rmd_B b]
		= [0]
\]
and in particular $\Img(\rmd^*) \subseteq \ker(\tilde{f}^*)$. To prove the reverse inclusion let
$[a] \in \ker(\tilde{f}^{k + 1})$, which implies that there is a $b \in B^k$ such that
$f^{k + 1}(a) = \rmd_B b$. The construction of $\rmd^*$ then implies
\[
	\rmd^*([g^{k}(b)]) = [a]
\]
therefore $\Img(\rmd^*) = \ker(\tilde{f}^*)$ and the sequence is exact.
\end{proof}

\begin{remark}
The attentive reader might object that this does not really proof the exactness of the Mayer-Vietoris since
the short Mayer-Vietoris contains $\Omega^*(\U) \oplus \Omega^*(\V)$, but the long one contains
$H^*(\Omega^*(\U)) \oplus H^*(\Omega^*(\V))$ not $H^*(\Omega^*(\U) \oplus \Omega^*(\V))$. But by writing
down these cohomologies one sees that they actually are the same.
\end{remark}

As abstract all this algebra might seem, this proof gives us not only the existence of $\rmd^*$, but also
a concrete recipe to construct $\rmd^* [c]$, which is quite valuable in practice.

To show its concrete usefulness, we are going to use the Mayer-Vietoris Sequence to calculate the de-Rham
Cohomology of $S^1$, including finding a concrete basis of $H^1(S^1)$.

We cover $S^1$ by two open sets $\U, \V \subseteq S^1$ as shown in TODO. These sets are diffeomorphic to $\RR$
and thus their cohomology is already known:
\[
	H^k_\dr(\U) = H^k_\dr(\V) = \begin{cases}
		\RR &\text{if } k = 0 \\
		0   &\text{otherwise}
	\end{cases}
\]
It is easy to show that the cohomology of the disjoint union of two smooth manifolds is the direct sum of the
respective cohomologies and therefore
\[
	H^k_\dr(\U \amalg \V) = \begin{cases}
		\RR \oplus \RR &\text{if } k = 0 \\
		0              &\text{otherwise}
	\end{cases}
\]
This is also the cohomology of their intersection, since $\U \cap \V$ is diffeomorphic to two disjoint copies
of $\RR$.

The Mayer-Vietoris sequence of these spaces is then
% https://q.uiver.app/#q=WzAsMTIsWzAsMCwiMCJdLFsxLDAsIkheMChTXjEpIl0sWzIsMCwiSF4wKFxcVSBcXGFtYWxnIFxcVikiXSxbMywwLCJIXjAoXFxVIFxcY2FwIFxcVikiXSxbNCwwLCJIXjEoU14xKSJdLFs1LDAsIjAiXSxbMSwxLCJcXFJSIl0sWzIsMSwiXFxSUiBcXG9wbHVzIFxcUlIiXSxbMywxLCJcXFJSIFxcb3BsdXMgXFxSUiJdLFswLDEsIjAiXSxbNCwxLCI/Il0sWzUsMSwiMCJdLFswLDFdLFsxLDIsImpeKiJdLFsyLDMsImleKiJdLFszLDQsIlxcbWF0aHJte2R9XioiXSxbNCw1XSxbNiw3XSxbNyw4XSxbOSw2XSxbOCwxMF0sWzEwLDExXSxbMSw2LCJcXGNvbmciLDMseyJzdHlsZSI6eyJib2R5Ijp7Im5hbWUiOiJub25lIn0sImhlYWQiOnsibmFtZSI6Im5vbmUifX19XSxbMiw3LCJcXGNvbmciLDMseyJzdHlsZSI6eyJib2R5Ijp7Im5hbWUiOiJub25lIn0sImhlYWQiOnsibmFtZSI6Im5vbmUifX19XSxbMyw4LCJcXGNvbmciLDMseyJzdHlsZSI6eyJib2R5Ijp7Im5hbWUiOiJub25lIn0sImhlYWQiOnsibmFtZSI6Im5vbmUifX19XSxbNCwxMCwiXFxjb25nIiwzLHsic3R5bGUiOnsiYm9keSI6eyJuYW1lIjoibm9uZSJ9LCJoZWFkIjp7Im5hbWUiOiJub25lIn19fV1d
\[\begin{tikzcd}
	0 & {H^0(S^1)} & {H^0(\U \amalg \V)} & {H^0(\U \cap \V)} & {H^1(S^1)} & 0 \\
	0 & \RR & {\RR \oplus \RR} & {\RR \oplus \RR} & {?} & 0
	\arrow[from=1-1, to=1-2]
	\arrow["{j^*}", from=1-2, to=1-3]
	\arrow["\cong"{marking, allow upside down}, draw=none, from=1-2, to=2-2]
	\arrow["{i^*}", from=1-3, to=1-4]
	\arrow["\cong"{marking, allow upside down}, draw=none, from=1-3, to=2-3]
	\arrow["{\mathrm{d}^*}", from=1-4, to=1-5]
	\arrow["\cong"{marking, allow upside down}, draw=none, from=1-4, to=2-4]
	\arrow[from=1-5, to=1-6]
	\arrow["\cong"{marking, allow upside down}, draw=none, from=1-5, to=2-5]
	\arrow[from=2-1, to=2-2]
	\arrow[from=2-2, to=2-3]
	\arrow[from=2-3, to=2-4]
	\arrow[from=2-4, to=2-5]
	\arrow[from=2-5, to=2-6]
\end{tikzcd}\]
where we have removed all superfluous cohomologies that are known to be trivial.

While it would be possible to calculate $H^1(S^1)$ just by dimension counting (a task the reader is gladly
invited to do) it is instructive to look closer at all these maps. $H^0(S^1)$ consists of the constant
functions on $S^1$. The constant functions on $S^1$ are pulled back by $j^*$ to the globally constant functions
on $\U \amalg \V$, which form a 1-dimensional subspace of $H^0(\U \amalg \V)$ i.e. the locally constant
functions on $\U \amalg \V$. These are then mapped by $i^*$ to the difference of their pullbacks by the
inclusions. This difference is constant on $\U \cap \V$ and thus the image of $i^*$ consists of the globally
constant functions, which form a 1-dimensional subspace of $H^0(\U \cap \V)$. The last map of this sequence
is of course the most interesting. Since the image of $i^*$ consists of the globally constant functions on
$\U \cap \V$, the functions that don't vanish under $\rmd^*$ are the locally constant ones that differ between
$\U$ and $\V$ e.g. $\mathbb{1}_\U$ (the indicator function of $\U$). To calculate $\rmd^*[\mathds{1}_\U]$,
we first us a partition of unity $\rho_\U, \rho_\V$ to calculate the right inverse of $\mathds{1}_\U$ under
$i^*$ as in the proof of exactness of the short Mayer-Vietoris sequence:
$(-\rho_\V \mathds{1}_\U, \rho_\U \mathds{1}_\U) = (-\rho_\V, 0)$. Explicitly calculating the exterior
derivative of this requires a choice of local coordinates, but luckily on
$S^1$ there is a convenient coordinate system consisting of the angle $\theta$. While this angle is only
specified up to multiples of $2\pi$ and hence is not a global coordinate system, we can almost treat it like
one. In particular one can always talk about the differentiation with respect to $\theta$: 
\[
	\rmd (-\rho_\V, 0) = (- \frac{\rho_\V}{\partial \theta} (\theta), 0)
\]
As seen in TODO this is a bump form on $\U$ and since this generates the image of $i^*$ i.e. $H^1(S^1)$
we can conclude that $H^1(S^1)$ is 1-dimensional.
